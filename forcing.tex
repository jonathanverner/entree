\cfoot{}\rhead{\thepage}
\lhead{{\scshape Forcing} $\qquad$ {\tiny \today } }

% \noindent{\Large{\scshape\bfseries Entreé to Generic Extensions and Forcing in Set Theory}} \\[0.1cm]
%
% \noindent {\scshape Bohuslav Balcar}, {\small CTS, J{\' \i}lsk{\' a} 1, Praha 1,
% 	Czech Republic, {\ttfamily balcar@cts.cuni.cz} } \\[0.1cm]
% \noindent {\scshape Jonathan Verner}, {\small KTIML MFF UK, {\ttfamily jonathan.verner@matfyz.cz}\\[0.1cm]
% {\tiny \today } \\[0.5cm]

%\maketitle

\thispagestyle{empty}

%%%%%%%%%%%%%%%%%%%%%%%%%%%%%%%%%%%%%%%%%%%%%%%%%%%%%%%%%%%%%%%%%%%%%%%%%%%
%%%%%%%%%%%%%%%%%%%%%%%%%%%%%%%%%%%%%%%%%%%%%%%%%%%%%%%%%%%%%%%%%%%%%%%%%%%
\section{Forcing notion and Generic filters}

{\tiny \today } \\[0.5cm]


We now come to the important definition.

\begin{definition}{\bf Forcing Notion.}\label{forcing-notion}
A \emph{forcing notion} is a synonym for an ordering (see Definition \ref{ordering}).
In this context elements of a forcing notion $P$ are called conditions and for two conditions $p,q\in P$ we say that
$p$ is \emph{stronger} than $q$ if $p\leq q$. If two conditions are compatible we write $p\parallel q$ and we write
$p\perp q$ if they are incompatible or disjoint.
\end{definition}

\begin{note}
Beware that some authors use the so called `eastern notation' and they say that a condition $p$ is stronger, i.e. carries more information, than $q$ if $p\geq q$. When reading a book or article on forcing, one should check which convention is used.
\end{note}

Recall that for a forcing notion $(P,\leq)$ we have defined its separative modification $(P,\leq_{sp})$ which
gives rise to the separative quotient $(P,\leq_{sp})/\hskip-2mm\approx$ and this in turn is
isomorphic to a dense subset of a complete
Boolean algebra we have denoted by $RO(P)$. Note that when dealing with properties which are interesting from the forcing point
of view, it is usually not important whether we take the forcing notion, its separative modification or the complete Boolean algebra $RO(P)$.
%Thus, e.g., $(P,\leq)$ is ccc (or $\sigma$-centered, etc.)
%iff $(P,\leq_{sp})$ is iff $(P,\leq)/\hskip-2mm\approx_{sp}$ is iff $RO(P)$ is.
The reason lies in the preservation of the disjointness relation.

Since it is formally irrelevant whether one works with an ordering
or a complete Boolean Algebra one may wonder whether there is really \emph{any}
difference. It turns out that orderings are easier to work with in
practical applications, while complete Boolean algebras are useful for
the theoretical work.

% What is better concerning forcing, dealing with ordering or with
% complete Boolean algebra? We can say that an ordering is more
% convenient for practical use while complete Boolean algebras are
% useful for theoretical reasoning.

\subsection{${}$ \hspace{-1em}Comparing forcing notions.}

\begin{definition}\label{compare} Given two forcing notions $P,Q$ we say that
 \begin{itemize}
  \item[(i)] $P\lessdot Q$ if $RO(P)$ can be completely embedded into $RO(Q)$,
  \item[(ii)] the orderings $P$ and $Q$ are \emph{forcing equivalent},
	$P\sim Q$ if the algebras $RO(P)$ and $RO(Q)$ are isomorphic.
 \end{itemize}
\end{definition}

The $\lessdot$ relation is transitive and $\sim$ is an equivalence relation. A special case of the $\sim$ relation is the case when
the two forcing notions have isomorphic dense subsets. However note that $P\lessdot Q$ and $Q\lessdot P$ \emph{does not},
in general, imply $P\sim Q$. We also have to be prepared that sometimes it is not so easy
to determine whether two forcing notions are in these relations.

\medskip

We introduce two criteria for $P\lessdot Q$.
\subsection{${}$ \hspace{-1em}Criterion 1.}\label{criterium1}
Let $P,Q$ be two forcing notions and $f : P \rightarrow Q$ be a regular mapping, i.e.
%If $f$ is regular, i.e.
\begin{itemize}
 \item[(i)] $f$ is homomorphism of orderings that
 \item[(ii)] preserves disjointness relation and
 \item[(iii)] each maximal antichain $A$ in $P$ is mapped
	onto a maximal antichain in $Q$,
\end{itemize}
then $P\lessdot Q$.

\medskip

\begin{proof}
 {\scshape (Hint)} The mapping $f$ preserves disjointness. Let
$B = RO(Q)$. Then there is a homomorphism
$h$ of $(P,\leq_1)$ into $(B^+,\leq)$ preserving disjointness
and such that $\bigvee_B h[P] = \mathbf 1$. The complete subalgebra
of $B$ completely generated by $h[P]$ is $RO(P)$.
\end{proof}

\begin{example}
 \begin{itemize}
  \item[(i)] Let $P \subset (Q,\leq)$ be such that every maximal antichain
	in $(P,\leq)$ is maximal antichain in $Q$. Then $P\lessdot Q$, and
	the identity on $P$ is a regular embedding.
  \item[(ii)] Let $P,Q$ be forcing notions with greatest elements
	$\mathbf 1_P$ and $\mathbf 1_Q$, respectively. Then
	$P \times \{\mathbf 1_Q\}$ and $\{\mathbf 1_P\} \times Q$
	are regular suborderings of $P \times Q$ (see Definition \ref{product}).
 \end{itemize}
\end{example}

\subsection{${}$ \hspace{-1em}Criterion 2.}\label{criterium2}
Let $h:Q\to P$ satisfy
 \begin{itemize}
  \item[(i)] $h$ is a homomorphism,
  \item[(ii)] $(\forall q\in Q)(\forall p^\prime\leq h(q))(\exists q^\prime)(q^\prime\parallel q\ \&\ h(q^\prime)\leq p^\prime)$,
  \item[(iii)] and $h[Q]$ is dense in $P$.
 \end{itemize}
 Then $P\lessdot Q$.

\medskip

\begin{proof}
 {\scshape (Hint)} We can assume that $Q$ is dense in $B = RO(Q)$
and $P$ is dense $C = RO(P)$. Let $f:C \to B$, where
$f(c) = \bigvee_B \{q\in Q \ : \ h(q) \leq c \}$. Then $f$ is a
complete embedding of $C$ into $B$.
\end{proof}

% \begin{lemma}
% Let $P\subseteq Q$ be two forcing notions satisfying
%  \begin{itemize}
%   \item[(i)] for all $p_1,p_2\in P$ if $p_1$ and $p_2$ are compatible in $Q$ then they are compatible in $P$.
%   \item[(ii)] every maximal antichain in $P$ is maximal in $Q$.
%  \end{itemize}
%  Then $P\lessdot Q$.
% \end{lemma}
%
% \begin{lemma}Let $h:Q\to P$ satisfy
%  \begin{itemize}
%   \item[(i)] If $q_1\leq q_2$ then $h(q_1)\leq h(q_2)$.
%   \item[(ii)] $(\forall q\in Q)(\forall p^\prime\leq h(q))(\exists q^\prime)(q^\prime\parallel q\ \&\ h(q^\prime)\leq p^\prime)$
%   \item[(iii)] $h[Q]$ is dense in $P$.
%  \end{itemize}
%  Then $P\lessdot Q$.
% \end{lemma}

\subsection{${}$ \hspace{-1em}Projections.} Let $C$ be a complete subalgebra
of a cBA $B$. The \emph{upward projection} $upr : B \to C$ is an onto mapping defined
by $upr(b) \ = \ \bigwedge_{C} \{c \in C : c \geq b \}$, for each $b \in B$.
The element $upr(b)$ is the least element in $C$ which is larger than $b$.
For any $b_1,b_2 \in B$ $upr(b_1 \vee b_2) = upr(b_1) \vee upr(b_2)$.
On the other hand we only have $upr(b_1 \wedge b_2) \leq upr(b_1) \wedge upr(b_2)$.

The mapping $upr : B^+ \to C^+$ satisfies all conditions for $h$ in
criterion 2.

\subsection{${}$ \hspace{-1em}The Generic filter.}

Recall that a filter $F$ on an ordered set $P$ is an upward closed, centred
subset of $P$ (see Definition \ref{various}). The following notion of
a \emph{generic filter} is the key to understanding the whole theory of
forcing. It is also probably the most confusing definition for beginners.

% If, after reading the definition, the reader thinks that such objects
% cannot exists, she is right --- at least in some sense. We will have to
% admit that the generic filter is only a semiset (see Definition \ref{semisets}).
% We will show, how to obtain, or generate, other semisets form the generic object.


% \begin{definition}\label{filter} {\bf Filter.}
% A subset $F\subseteq P$ of a forcing notion $P$ is called a \emph{filter} on $P$ if
% \begin{itemize}
%   \item[(i)] $F$ is upward closed, i.e. $x\in F$ and $x\leq y$ imply $y\in F$.
%   \item[(ii)] $F$ is downward directed, i.e. $(\forall x,y\in F)(\exists z\in F)(z\leq x\ \&\ z\leq y)$.
% \end{itemize}
% \end{definition}

% It is clear that an increasing union of filters is again a filter. Also a filter is centered and any centered subsystem of $F$ can be extended to a filter. In particular, the forcing notion $P$ is $\sigma$-centered iff it can be written as a countable union
% of filters.

\begin{definition}\label{generic}{\bf Generic filter.}
A filter $G$ on a forcing notion $P$ is called a \emph{generic filter} if it satisfies a
condition of `genericity':
%\placeformula
$$ %\startformula
(\forall D\subseteq P,\ D\ \mbox{dense})(\exists d\in D)(d\in G).
$$ %\stopformula
\end{definition}


% \begin{note}
% It is not hard to see that a generic filter on $(P,\leq)$ uniquely determines a generic filter on $(P,\leq_{sp})$, $(P,\leq_{sp})/\hskip-2mm\approx_{sp}$ and $RO(P)$ and hence also on any forcing notion $Q$ such that $Q\lessdot P$. We will use this fact freely. Also the generic filter is $\approx_{sp}$ saturated, i.e. if $x\in G$ and $y\approx_{sp}x$ then $y\in G$.
%
% Note that if $C$ is a complete subalgebra of a cBA $B$ and $G$ is a generic filter on $(B^{+},\leq)$ then
% $G\cap C$ is a generic filter on $(C^{+},\leq)$.
% \end{note}
%
% \begin{note} In the following, we will often use the term \emph{$M$-generic filter}, where $M$ is a model of $ZFC$ or a suitable fragment. Such a filter need not
% intersect all dense subsets, but only those, which are elements of $M$. Thus a \emph{generic filter} is a $V$-generic filter
% \end{note}

The following are equivalent characterisations of genericity and follow
from the Fact \ref{dense_equivalents}.

\begin{fact}\label{generic-equivalence}
For a filter $G$ on $P$ the following conditions are equivalent.
\begin{itemize}
  \item[(i)] $G$ is generic,
  \item[(ii)] whenever $X\subseteq P$ is predense $G\cap X\neq\emptyset$,
  \item[(iii)] whenever $A\subseteq P$ is a maximal antichain $G\cap A\neq\emptyset$.
\end{itemize}
\end{fact}

Remember that when we talk about $F$ being a filter on a Boolean algebra $B$
we in fact mean that $F$ is a filter on the canonical ordering $(B^+,\leq)$, in particular
$F$ does not contain $\mathbf 0_B$.

\begin{fact}\label{set-complete}
Let $B$ be a cBA, $G$ a filter  on $B$. Then $G$ is a generic filter on $B$ if and only if
\begin{itemize}
 \item[(i)] for any $b \in B^+$ either $b \in G$ or $-b \in G$, i.e.
	 $G$ has the \emph{ultrafilter property} and
 \item[(ii)] for any set $X \subset G$ the meet $\bigwedge_B X \in G$, i.e.
	$G$ is \emph{set-complet}.
\end{itemize}
% $G$ has ultrafilter property, i.e. for any $b \in B^+$
% either $b \in G$ or $-b \in G$ and $G$ is a set-complete,
% i.e. for any set $X \subset G$ the meat $\bigwedge_B X \in G$.
%
% $ then the following are also equivalent to (i)
% \begin{itemize}
%   \item[(iv)] $G$ is an \emph{ultrafilter}, i.e. for any $p\in P$ either $p\in G$ or $-p\in G$.
%   \item[(v)] $G$ is \emph{set-complete}, i.e. for any set $X\subseteq G$, the infimum $\bigwedge X$ is an element of $G$.
% \end{itemize}
\end{fact}

%In the general case the last condition is only \emph{implied} by (i):

\begin{proof}
 For any $b \in B^+$, the family $\{b,-b\}$ is predense in $B$
and
so is the family $\{\bigwedge X\} \cup \bigcup \{ \{x\}^\perp : x \in X \}$.

\smallskip

For the opposite direction let $A\subset B^+$ be a maximal antichain. If
$A \cap G = \emptyset$, then $(\forall a \in A)$ $-a \in G$ and thus
$\mathbf 0 = \bigwedge \{ -a : a \in A \} \in G$, a contradiction.
\end{proof}

% \begin{lemma}
% A generic filter on $(P,\leq)$ is 'set complete', that is for any set $X\subseteq G$
% there is a $z\in G$ which is below all the elements of $X$ in the separative modification $\preceq$ of $\leq$, i.e.
% $(\forall x\in X)(z\preceq x)$.
% \end{lemma}
%
% \begin{corollary}
% If a generic filter on $P$ is a set, then it is generated by an atom of $P$.
% \end{corollary}
%
%
% \begin{proof} Consider the following sets: $D_1=\{y\in P:(\exists x\in X)(y\perp x)$ and $D_2=D_1^\perp$. Then
% $D_1\cup D_2$ is predense and $D_1\cap G=\emptyset$ since $G$ is a filter. Because $G$ is generic, we can
% find a $z\in D_2\cap G$. For any $x\in X$ we have that $\{z\}^\perp \supseteq \{x\}^\perp $, which is equivalent to
% $z\preceq x$ (see Definition \ref{separative-modification}). To see the corollary notice that any $z\in G$ which
% is $\preceq$-below all of $G$ must be an atom of $P$.
% \end{proof}

\begin{corollary}
 If $C$ is a complete subalgebra of a cBA $B$ and $G$ is a generic filter on $B$,
then $G \cap C$ is a generic filter on $C$.
\end{corollary}

\subsection{${}$ \hspace{-1em} Comment.}\label{generic_images}
Let $G$ be a generic filter on $(P,\leq)$. Then $G$ is
	$\approx_{sp}$ saturated, i.e. if $x \in G$ and $x \approx_{sp} y$
	then $y \in G$. This follows from the fact that
	$x\approx_{sp} y\rightarrow\{x\}^\perp=\{y\}^\perp\rightarrow\{y\}\cup\{x\}^\perp$
	is predense in $(P,\leq)$.
	Thus $G$ is also a generic filter on $(P,\leq_{sp})$.



	Consider now the separative quotient $(P,\leq_{sp})/\hskip-2mm\approx$
	and the relation $r_1 = \{\langle x,[x]_{\approx_{sp}} \rangle : x \in P \}$.
	Then $r_1[G] = G_1$ is a generic filter on the separative quotient
	and moreover $G = r_1^{-1}{}'' G_1$.
	Moreover, considering $B = RO(P)$, with a homomorphism
	$h : P \to B$ on a dense set, let $r_2 \subseteq P \times B$
	be a relation defined by $\langle p,a \rangle \in r_2 \equiv a \geq h(p)$.
	Then $r_2[G] = G_2$ is a generic filter on the cBA $B$ and $G = r_2^{-1}{}'' G_2$.


\medskip

We can see that a generic filter on $P$ determines, simply via images by set relations,
a generic filter on any ordering $Q$, for which $Q \lessdot P$ (see Definition \ref{compare}).


\medskip

\subsection{${}$ \hspace{-1em} Problem.} It follows directly from
Fact \ref{set-complete} that a generic filter needs to extend
an atom to be `set-complete'. In other words there are no non-trivial generic
filters. Moreover all ordered sets we discussed so far are atomless
and so there are \emph{no} generic filters on them.

\subsection{${}$ \hspace{-1em} What now?} We need to
break out of our universe of sets and admit semisets into our
reasoning. It is not impossible, that a generic filter might
exist as a semiset.


\subsection{${}$ \hspace{-1em} An analogy.}
Going back in time, imagine, that there are no numbers
beyond real numbers for us. However this means, that
some simple quadratic equations have no solution. At this
point, we break out of our universe of real numbers and accept
a single imaginary object $\sqrt{-1}$. Then, starting with this object
and our universe of numbers $\mathbb R$,
we are able to build an extension of $\mathbb R$
--- the field of complex numbers --- and extend algebraic operations
to this new field. %In such extension we are loosing the ordering
What we get is an algebraically closed field, i.e. not only
algebraic equations with real coefficients now have a solution but also equations with complex coefficients.

One can look at a generic filter as a kind of imaginary object,
in our terminology, a proper semiset.

Let us broaden our minds and admit that our universe is not absolute,
but that there are possible extensions in which some proper
semisets of our universe $V$ become sets. %and we get an extension of our universe.
A generic filter is the gate to such extensions in a smiliar way as $\sqrt{-1}$ was
to the the extension of $\mathbb R$.

An extension of a universe of course not only contains all sets from the original universal
class $V$, but also admits an extension of the relation $\in$ to new born sets.

As the field of complex numbers is the smallest field that contains
$\mathbb R$ and $\sqrt{-1}$, in what follows we will try to
convince you that one can consider the smallest extension of
our universe $(V,\in)$ that contains a $V-$generic filter $G$. Such an extension,
usually called the generic extension,
will satisfy the axioms of set theory and will be denoted by $V[G]$.

\begin{definition}\label{universe_extension}{\bf Extension.}
A transitive universe $(M,\in)$ is called an extension of
a universe $(V,\in)$ if, in $M$, all the axioms of ZF are satisfied,
$M \supseteq V$ and both universes have the same ordinal numbers.
\end{definition}

When talking about an extension, the starting universe
is often referred to as the \emph{ground model} and is denoted by $(V,\in)$. Note
that the notion of an extension is more general than the notion of a generic
extension.

% What is the role of semisets here? Consider any proper extension
% $M$ of $(V,\in)$, i.e. $V \subsetneq M$, $V$ is a transitive
% subclass of $M$, hence all the ordinals in extension are the
% ordinals of $V$, i.e. ordinal numbers are absolute.
% Following the assumption $M - V \not = \emptyset$,
% by the axiom of foundation (which hold in $M$) there is $\in$-minimal
% element $\sigma \in M-V$.

\smallskip

\subsection{Why generic filters?}
The driving motive behind the definition of a generic filter is the
quest for simplicity: we want the extension to ``essentially depend
on a single semiset''. This, together with the natural requirement
that the extension is closed under simple set operations, already
gives us the generic filter. This paragraph will explain this in
more detail.

Consider an extension $V\subseteq M$ of the ground model $V$ as in \ref{universe_extension}.
The class $M$ is closed under standard binary operations, namely the
cartesian product $x\times y$ and the Boolean operations of union $x\cup y$
and set difference $x\setminus y$ and also intersection, since $x\cap y=x\setminus(x\setminus y)$.
Moreover it is also closed under the operation $rng(z)=\{x:(\exists x)(\langle x,y\rangle\in z\}$.

The following notation for semisets of $M$ over $V$ is convenient:
$$
Sm_{M,V}(\sigma)\equiv\sigma\in M\ \&\ (\exists a\in V)(\sigma\subseteq a).
$$
When there is no danger of confusion, we drop the subscripts $M,V$.
The basic properties of semisets of $M$ over $V$ are the following:

\begin{itemize}
 \item[(i)] $(\forall x\in V)(Sm(x))$,
 \item[(ii)] if $M\setminus V\neq\emptyset$ then there is a proper semiset
 \item[(iii)] the class of semisets is closed under the cartesian product,
 union and difference of sets and the range operation
 \item[(iv)] and it is closed under taking images via ground model relations, i.e. if $Sm(\sigma)$ and $r\in V$ then $Sm(r[\sigma])$.
\end{itemize}

Only number (ii) needs comments: Using the axiom of
foundation valid in $M$ find an $\in$-minimal element $\sigma$ of
$M\setminus V$. The image of $\sigma$ under the ground model rank function
$rk$ cannot be cofinal in $On$ because $rk[\sigma]$ is an element of $M$
by the axiom of replacement and $M$ and $V$ have the same ordinals.

Note however, that not all elements of $M$ are semisets over the ground model.
For example the singleton $\{x\}$ consisting of some new set $x\in M\setminus V$
is \emph{not} a semiset.

We now give a precise definition of dependence between semisets and also
introduce the notion of their similar
ity.

\begin{definition}\label{similarity}
We say that $\sigma$ is \emph{dependent} on $\rho$, denoted $Dep(\rho,\sigma)$, if there
is a relation $r$ from the ground model such that $\sigma = r[\rho]$.
Two semisets $\sigma,\rho$ are said to be \emph{similar}, denoted $Sim(\sigma,\rho)$, if both are
dependent on the other.
\end{definition}

Fix a nonempty semiset $\sigma\subseteq a$ of $M$ over $V$. We can consider the class $Sm_{\sigma,M,V}$
of semisets depending on $\sigma$:

\begin{fact}
 \begin{itemize}
  \item[(i)] Each ground model set depends on $\sigma$, i.e. $(\forall x\in V)(Dep(\sigma,x))$ and
  \item[(ii)] if $\rho_0,\rho_1$ depend on $\sigma$ then so does $\rho_0\cup\rho_1$.
 \end{itemize}
\end{fact}
\begin{proof}
 For (i) given any $x\in V$ take some $y\in\sigma$ and the relation $\{y\}\times x$ shows that
 $x$ depends on $\sigma$. For two, given witnesses $r_0,r_1$ to the dependence of $\rho_0,\rho_1$ on $\sigma$,
 the relation $r_0\cup r_1$ shows that $\rho_0\cup\rho_1$ depends on $\sigma$.
\end{proof}

\begin{proposition}
The set $\rho=\{b\subseteq a:b\in V,\ b\cap \sigma\neq\emptyset\}$ depends on $\sigma$.
\end{proposition}
\begin{proof}
This is witnessed by the relation $r=\{\langle x,b\rangle: x\in b, b\subseteq a, b\in V\}$.
\end{proof}

A natural question to ask is: When is the class $Sm_{\sigma,M,V}$ closed under set difference, i.e.
given $\rho\in Sm_{\sigma,M,V}$ when is $\mathcal{P}(a)\setminus\rho$ also a member of $Sm_{\sigma,M,V}$?

\begin{proposition}[Balcar,Vopěnka] The semiset $\pw(a)\setminus\rho$ depends on $\sigma$ iff there is an ground model
 ordering $\leq$ on $a$ such that $\sigma$ is a generic filter on $(a,\leq)$.
\end{proposition}

\begin{corollary} The class $Sm_{\sigma,M,V}$ is a ring iff $\sigma$ is a generic filter on some ordered
 set from the ground model.
\end{corollary}

\begin{proof}[Proof of proposition] Let $s\in V$ be a relation such that $s[\sigma]=\pw(a)\setminus\rho=\{x\subseteq a:x\cap\sigma=\emptyset\}$.
We can assume that $s\subseteq a\times\pw(a)$. Define
$$
\bar{s}=\{\langle x,y\rangle:x\in dom(s)\ \&\ y\in\bigcup s[\{x\}]\}.
$$
It is easy to see that
\begin{itemize}
 \item[(i)] if $x\in\sigma$ then $\bar{s}[\{x\}]\cap \sigma=\emptyset$ and
 \item[(ii)] for any $b\subseteq a,b\in V$ if $b\cap \sigma=\emptyset$ then
 there is an $x\in\sigma$ such that $b\subseteq\bar{s}[\{x\}]$.
\end{itemize}

Let $d=(\bar{s}\cup\bar{s}^{-1})\setminus id_a$. This is a symmetric and antireflexive
relation belonging to $V$. Finally let
$$
x\leq y\equiv d[\{x\}]\supseteq d[\{y\}],\quad \mbox{for}\ x,y\in a
$$

\emph{Claim.} The relation $d$ has the following properties:
 \begin{itemize}
  \item[(iii)] $(\forall x\in a)(x\in\sigma\equiv d[\{x\}]\cap\sigma=\emptyset)$ and
  \item[(iv)] $\langle x,y\rangle\in d$ implies $x\perp y$ in the ordering $(a,\leq)$.
 \end{itemize}
(iii) is straightforward to check and for (iv) suppose to the contrary that there is a $z\leq x,y$.
Then $d[\{z\}]\supseteq d[\{x\}]\cup d[\{y\}]$ and $y\in d[\{x\}]$ so that $\langle z,z\rangle\in d$
a contradiction.

\emph{Claim.} $\sigma$ is a generic filter on $(a,\leq)$ over $V$.
\emph{Filter.} If $x\in\sigma$ and $y\geq x$ then $d[\{x\}]\supseteq d[\{y\}]$ so $d[\{y\}]\cap\sigma=\emptyset$
so by (iii) $y\in\sigma$. Next, given $x,y\in\sigma$ we know that $(d[\{x\}]\cup d[\{y\}])\cap\sigma=\emptyset$,
so, by (ii), there is $z\leq x,y$.

\emph{Genericity.} Let $b\subseteq a$ be a dense set in $(a,\leq)$. If $b\cap\sigma\neq\emptyset$, then
we are done. Otherwise, by (ii) and (iii), there is $z\in\sigma$ such that $d[\{z\}]\supseteq b$ and, by (iv), $z$
is disjoint with all elements of $b$ which contradicts the fact that $b$ is dense.

The other implication in the proposition follows from the proof of the corollary.
\end{proof}

\begin{proof}[Proof of Corollary]
 It remains to check that the semisets depending on a generic fiter $\sigma$ on $(a,\leq)$ are closed
 under set difference and intersection. Let $\pi=r[\sigma]\subseteq b$ and $\tau=s[\sigma]$. Then
 $$
 \pi\cap\tau = b\setminus((b\setminus\pi)\cup(b\setminus\tau))
 $$
 and
 $$
 \pi\setminus\tau = b\setminus((b\setminus\pi)\cup\tau).
 $$
 It suffices to show that $b\setminus\pi$ depends on $\sigma$. But for this we can let
 $$
 \bar{r}=\{\langle x,y\rangle:x\in(r^{-1}[\{y\}])^\perp, y\in b\}
 $$
 and we have $\bar{r}[\sigma]=b\setminus\pi$: For $y\in b\setminus\pi$ we know that $r^{-1}[\{y\}]\cap\sigma=\emptyset$ so,
 since $r^{-1}[\{y\}]\cup (r^{-1}[\{y\}])^\perp$ is predense, there is $x\in\sigma$ such that $\langle x,y\rangle\in\bar{r}$.
 On the other hand if $y\in\pi$ then $r^{-1}[\{y\}\cap\sigma\neq\emptyset$ and no $x\in(r^{-1}[\{y\}])^\perp$ is a member of $\sigma$,
 hence $y\not\in r[\sigma]$.
\end{proof}



Recall that in Comment \ref{generic_images} we used images of a generic filter
to show that a generic on the order $(P,\leq)$ produces a generic on the complete
Boolean algebra $RO(P)$. In fact those generic filters are similar. Moreover
if $P\lessdot Q$ then given a generic $G$ on $Q$ the embedding naturally determines
a filter $\bar{G}$ on $P$ and $Dep(G,\bar{G})$ so $V[\bar{G}]\subseteq V[G]$ and
$Sm_{V[\bar{G}],V}\subseteq Sm_{V[G],V}$.

\smallskip


% \begin{note}
% The last corollary shows, that to obtain generic filters, we must break out of our universe of sets.
% Section \ref{forcing-relations} is devoted to explaining what this means and how it can be done.
% \end{note}

\subsection{${}$ \hspace{-1em}Slogan.}
Since we will meet different universes of sets, we have to be
more precise in using the term generic filter. We shall therefore say that
\begin{center}
 `$G$ is generic filter on $P$ over $N$'
\end{center}
or
\begin{center}
 `$G$ is an $N$-generic filter on $P$'
\end{center}
\noindent to express that the ordering $P$ is an element of $N$ and the generic filter $G$
intersects all dense subsets of $P$ which \emph{belong to $N$} (see
Definition \ref{generic}). Note that whenever $P$ is atomless then $G \not \in N$.

\subsection{${}$ \hspace{-1em}Consistency.}
The easiest way to see that adding a generic filter $G$ to
a ground model $N$ produces a model of ZF (ZFC), or that $G$ can be consistently
added to $N$, is to pretend that $N$ is a countable transitive model.
Let $P$ be an ordering, $P \in N$. Since $N$ is
countable there are only countably many dense subsets of $P$ in $N$
and we can use Rasiowa~-~Sikorski Theorem \ref{rasiowa-sikorski}\ % in $V$
to obtain a filter $G$ that intersects all those dense sets.
The smallest model $N[G]$ of ZF (ZFC) containing $N$ and the set $G$ is the
generic extension of $N$ and $G$ is a generic filter on $P$ over $N$.

% If one, thanks to some unexplainable metaphysical urging, does not want to assume the
% existence of a countable model of ZFC, one need not worry. The general
% idea sketched here for a countable model of $ZFC$ will also work for a countable
% model of some sufficiently large fragment of ZFC, see \ref{QaA}.

\subsection{How to describe a generic extension $V[G]$?}\label{first-names} The theory of forcing does this
via names that are available already in the ground model $V$. These
names are then evaluated using the generic filter $G$ as a parameter.
We show one quick approach now, while in a subsequent chapter we describe
the more standard approach.

Let $(P,\leq)$ be a forcing notion. In $V$ we define a hierarchy of names.
A name will be some relation $r\in V, dom(r)\subseteq P$ together with a rank $\alpha$.
We denote the names of rank $1$ as
$$
R_1=\{(r,1):dom(r)\subseteq P\}
$$
and inductively define
$$
R_\alpha=\left\{(r,\alpha):r\in R_1\&rng(r)\subseteq\bigcup_{\beta<\alpha} R_\beta\right\}.
$$
Formally, names are pairs, but we shall say that $r$ is a name of rank $\alpha$
instead of talking about a name $(r,\alpha)$.

Let $G$ be $V$-generic filter $G$ on $P$. The evaluation of a name $r$ will depend
on its rank. If $r$ is of rank $1$ then
$$
r/_G=r[G].
$$
Inductively, if $r$ is of rank $\alpha$, we let
$$
r/_G=\{s/_G:s\in r[G]\}
$$
Finally we let
$$
V_\alpha[G]=\{r/_G:r\ \mbox{is of rank}\ \alpha\}\ V[G]=\bigcup_{\alpha<On} V_\alpha[G]
$$
The way $V[G]$ is built up is essentially by simulating the operation of power set.
We shall illustrate this on a countable transitive model $M$ and an $M$-generic filter $G$.
Let $\vartheta=M\cap On$, then by induction to $\vartheta$ we define
$$
\pw^0(M)=M,\ \pw^{\alpha+1}(M)=\pw(\pw^\alpha(M))\ \mbox{and}\ \pw^\alpha(M)=\bigcup_{\beta<\alpha}\pw^\beta(M),\ \mbox{for}\ \alpha\ \mbox{limit}.
$$
We will have that $M_\alpha[G]\subseteq \pw^\alpha(M)$. Also note that $\pw^\vartheta(M)$ will
be very big but $M=M_\vartheta[G]$ will be a countable subset of $\pw^\vartheta(M)$, since there
are only countably many $M$-names.

In subsequent chapters we will see that a generic extension $M=V[G]$ is characterised by the following property
$$
(\forall \sigma \in M) \ ( \sigma \subseteq V \ \rightarrow \
	\exists r \in V \ \mbox{such that} \ r[G] = \sigma, \ \mbox{hence} \ \sigma \subseteq rng(r) \in V),
$$
i.e. a generic extension $M$ is determined already by $M_1[G]$.

The next chapter will elaborate examples of forcing notions and the properties
of semisets in their generic extensions. This is not a coincidence, since
the whole extension is already determined by its semisets, although there of course
are new sets, which are not semisets over the ground model.

It will be helpful to know that each semiset has a name of rank $1$, i.e.
$$
Sm_{V,V[G]}=\{r/_G:r\ \mbox{is of rank}\ 1\}.
$$

%%%%%%%%%%%%%%%%%%%%%%%%%%%%%%%%%%%%%%%%%%%%%%%%%%%%%%%%%%%%%%%%%%%%%%%
%%%                          END                                    %%%
%%%%%%%%%%%%%%%%%%%%%%%%%%%%%%%%%%%%%%%%%%%%%%%%%%%%%%%%%%%%%%%%%%%%%%%


