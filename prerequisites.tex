\cfoot{}\rhead{\thepage}
\lhead{{\scshape Prerequisites} $\qquad$ {\tiny \today } }

% \noindent{\large{\scshape\bfseries %Entre{\' e}
% An Exposition to Generic Extensions and Forcing in Set Theory}} \\[0.1cm]
%
% \noindent {\scshape Bohuslav Balcar}, {\small CTS, J{\' \i}lsk{\' a} 1, Praha 1,
% 	Czech Republic, {\ttfamily
% 	%\href{mailto:balcar@cts.cuni.cz}
% 	{balcar@cts.cuni.cz} } } \\[0.1cm]
% \noindent {\scshape Tom{\' a}{\v s} Paz{\' a}k}, {\small CTS, J{\' \i}lsk{\' a} 1, Praha 1,
% 	Czech Republic, {\ttfamily
% 	%\href{mailto:pazak@cts.cuni.cz}
% 	{pazak@cts.cuni.cz} } } \\[0.1cm]
% \noindent {\scshape Jonathan Verner}, {\small KTIML MFF UK, Praha,
% 	Czech Republic {\ttfamily
% 	%\href{mailto:jonathan.verner@matfyz.cz}
% 	{jonathan.verner@matfyz.cz}} }\\[0.1cm]
% {\tiny \today } \\[0.5cm]

%\maketitle

\thispagestyle{empty}

\pagenumbering{arabic}
%%%%%%%%%%%%%%%%%%%%%%%%%%%%%%%%%%%%%%%%%%%%%%%%%%%%%%%%%%%%
%%%%%%%%%%%%%%%%%%%%%%%%%%%%%%%%%%%%%%%%%%%%%%%%%%%%%%%%%%%%
\section{Set Theory prerequisites}
{\tiny \today } \\[0.5cm]

When we talk about Set Theory we mean the first order predicate theory in a language with equality containing a binary predicate $\in$
with the standard axioms. Formulas describing properties of sets are built up from symbols for variables, predicates ($\in,=$),
logical connectives and quantifiers ($\forall,\exists$). If $\phi(x_0,\ldots,x_n)$ is such a formula, then $\{\langle x_0,\ldots, x_n\rangle:
\phi(x_0,\ldots,x_n)\}$ is the class defined by this formula. E.g. $V=\{x:x=x\}$ is the class of all sets.

\subsection{${}$ \hspace{-1em} Axioms of Zermelo-Fraenkel.}

\begin{itemize}
\item[1.]\emph{Extensionality.} If $x$ and $y$ have the same elements, then
$x=y$ $$(\forall x)(\forall y)((\forall z)(z\in x\leftrightarrow z\in y)\rightarrow x=y)$$
\item[2.]\emph{Pairing.} For any $a$ and $b$ there exists a set $\{a, b\}$ that
contains exactly $a$ and $b$
$$(\forall a)(\forall b)(\exists x)((\forall z)(z\in x\leftrightarrow (z=a\vee z=b)))$$
\item[3.]\emph{Separation (Comprehension).} If $\varphi$ is a formula (with parameter $p$),
then for any set $x$ and $p$ there exists a set $y = \{u \in x : \varphi(u, p)\}$,
i.e. a set that contains all those $u \in x$ that satisfy $\varphi$. Formally, given a formula $\varphi$ with $u$ and $p$ as  free
variables, the formula
$$(\forall p)(\forall x)(\exists y)(\forall u)(u\in y\leftrightarrow y\in x\ \&\ \varphi(u,p))$$
is an axiom of set theory.
\item[4.]\emph{Union.} For any $x$ there exists a set $y =\bigcup x$, the \emph{union}
of all elements of $x$
$$(\forall x)(\exists y)(z\in y\leftrightarrow (\exists u)(u\in x\ \&\ y\in u))$$
\item[5.]\emph{Power Set.} For any $x$ there exists a set $y = \pw (x)$, the
set of all subsets of $x$
$$(\forall x)(\exists y)(\forall z)(z\in y\leftrightarrow (\forall u)(u\in z\rightarrow u\in x))$$
\item[6.]\emph{Infinity.} There exists an infinite set
$$(\exists x)(\emptyset\in x\ \&\ (\forall z)(z\in x\rightarrow z\cup\{z\} \in x))$$
\item[7.]\emph{Replacement.} The image of a set under a (class) mapping is a set. Formally if $\varphi(x,y)$ is a formula then
$$\big[(\forall x)(\forall y)(\forall z)((\varphi(x,y)\ \&\ \varphi(x,z)) \rightarrow z=y)\big]\rightarrow
\big[(\forall a)(\exists b)(y\in b\leftrightarrow (\exists x\in a)\varphi(x,y))\big]$$
 is an axiom.
\item[8.]\emph{Foundation.} Every nonempty set has an $\in$-minimal element
$$(\forall x\neq\emptyset)(\exists y)(y\in x\ \&\ (\forall z)(z\in y\rightarrow z\not\in x))$$
\item[9.]\emph{Choice.} Every set has a choice function
$$(\forall x)(\exists f, \mbox{function})(\forall z)(z\in x\ \&\ z\neq\emptyset\rightarrow f(z)\in z)$$
\end{itemize}

\subsubsection{Comments.}
\begin{itemize}
%  \item[(i)] We introduced only vague formulation of axioms, but each of the
% 	axioms of ZFC can be expressed as a formula, e.g. Foundation can
% 	be stated in following form:
% 	$$
% 	(\forall a \not = \emptyset)\ (\exists x \in a) \ x \cap a = \emptyset.
% 	$$
\item[(i)] In the formal statements of the axiom schemas we tacitly assume that the formulas which serve as
           parameters do not contain ``conflicting free variables''.
\item[(ii)] Some authors include an axiom postulating the existence of a set, i.e. $(\exists x)(x=x)$.
            We do not include it here since it is provable from the axioms of predicate logic with equality.
            Moreover, this is only relevant for fragments which do not include the axiom of infinity.
\item[(iii)] Once we have any set, say $a$, using the third axiom we can prove the existence of the empty set $\{x\in a:x\neq x\}$,
            customarily denoted as $\emptyset$.

 \item[(v)] Formally, our language admits varibles ranging over sets only, however it is
            often useful to talk about collections which are too large to be sets, e.g.
            the universal class $V$. To overcome this difficulty we shall regard formulas
            involving class variables as shorthands. Any formula $\varphi$ naturally determines
            the collection $F=\{x:\varphi(x)\}$ of all sets satisfying $\varphi$. In this
            situation $x\in F$ is a shorthand for $\varphi(x)$. so, e.g. the formula
            $(\forall x\in F)\psi$ is shorthand for $(\forall x)(\varphi\rightarrow\psi)$.
\item[(vi)] The list of axioms is not the shortest possible, e.g. each instance of the axiom of comprehension follows from an instance of the axiom of
            replacement. Also pairing can be proved from the other axioms. However since the list is infinite anyway, this makes \emph{no} difference
            whatsoever.


\end{itemize}


\smallskip

It is customary to let ZF denote Set Theory with axioms 1.--8. and ZFC Set Theory ZF with choice,
i.e. with axiom 9. Occasionally we shall write ZF$^-$ and ZFC$^-$ when we omit axiom 5.

\begin{notation}
Throughout the text we will use the standard notation. For definiteness given a relation $r$
and a class $A$ we let $r[A]=\{y:(\exists a\in A)(\langle a,y\rangle\in r\}$. Similarly for
a function $f$ we let $f[A]=\{y:(\exists a\in A)(f(a)=y)\}$, $f^{-1}[A]=\{x:f(x)\in A\}$.
Moreover the domain of a function (or relation) $f$ will be denoted by $dom(f)$ and the
range of $f$, i.e. $\{y:(\exists x)(f(x)=y)\}$, will be denoted by $rng(f)$.
The corresponding notions for a relation will also be used.
% The ordered pair $\langle x,y\rangle$ will defined according to Kuratowski as $\{\{x\},\{x,y\}\}$ but
% we note that for our purposes any other sensible definition would also do.
\end{notation}

\begin{definition}\label{ts}
{\bf Transitive set.} A set $a$ is \emph{transitive} if every element of $a$
 is a subset of $a$, i.e. $(\forall x\in a)(x\subseteq a)$.
\end{definition}

\begin{fact} Every set $a$ has a \emph{transitive closure}, that is, the smallest transitive
 set containing $a$ as an element. It is denoted by $\mbox{tcl}(a)$.
\end{fact}
\begin{proof} (Hint.) By the axiom of union we can define a (class) function $F$ such that
$F^0(x)=x$, $F(x)=\bigcup x$ and we let $Tcl(a)=\bigcup_{n<\omega} F^n(\{a\})$, i.e.
$\mbox{tcl}(A)=\{A\}\cup A\cup \bigcup A\cup \bigcup\bigcup A\cdots$ is arrived at by iterating
the union operation. See also example \ref{transitive_closure}.
\end{proof}

\begin{definition}\label{ordnumber}{\bf Ordinal number.} A set $\alpha$ is an ordinal number if
 \begin{itemize}
  \item[(i)] $\alpha$ is a transitive set and
  %\item{(ii)}  every element of $\alpha$ is transitive
  \item[(ii)] the $\in$ relation is a linear ordering on $\alpha$, i.e. for all $\beta,\gamma\in\alpha$
               either $\beta\in\gamma$ or $\gamma\in\beta$ or $\beta=\gamma$.
 \end{itemize}
 In other words an ordinal number is the set of all smaller ordinal numbers well-ordered by $\in$.
 The class of all ordinal numbers is denoted by $On$. We write $\alpha<\beta$ to mean $\alpha\in\beta$.
\end{definition}

%The standard definition above is due to John von Neumann.
\begin{definition}[Maximo-lexicographical ordering] The class $On^2$ of pairs of ordinals can be well ordered in a particularly nice way. We say
 that $\langle \alpha_1,\beta_1\rangle\leq_{MLex}\langle\alpha_2,\beta_2\rangle$ iff
 $$
 \max \{\alpha_1,\beta_1\} < \max \{\alpha_2,\beta_2\}\quad \vee\quad \left(\max \{\alpha_1,\beta_1\} = \max \{\alpha_2,\beta_2\}\ \&\
                                                                     \min\{\alpha_1,\beta_1\}<\min\{\alpha_2,\beta_2\}\right)
 $$
 This ordering is a set-like well order, i.e. for each pair $p\in On^2$ the class of pairs which are $\leq_{MLex}$ below $p$ is a set.
\end{definition}


\subsection{${}$ \hspace{-1em} Approximation of the universe.}

% The universal class $V=\{x:x=x\}$ satisfies all axioms of Set Theory, but it is not a set. However there are transitive sets,
% which satisfy a large fragment of the axioms, although not necessarily all of them. The ones most commonly used are
% the two hierarchies of $V_\alpha$'s and $H(\kappa)$'s.

We shall show that there are transitive sets which --- in a sense --- approximate the universe $V$. The ones most commonly used are
the two hierarchies of $V_\alpha$'s and $H(\kappa)$'s.

\begin{definition}{\bf The hierarchy of $V_\alpha$'s.} Using transfinite recursion to iterate the power set operation we define $V_\alpha$
for $\alpha\in On$ as follows:
$$ %\startformula
V_{\emptyset}=\emptyset,\quad V_{\alpha+1}=\pw(V_\alpha),\quad V_\alpha = \bigcup_{\beta<\alpha} V_\beta,\ \mbox{for limit}\ \alpha.
$$ %\stopformula
\end{definition}
\begin{fact} $V=\bigcup\{V_\alpha:\alpha\in On\}$. \\
Note that this fact is equivalent to the axiom of foundation (axiom 8.).
\end{fact}

\begin{definition}\label{rank}{\bf Rank.} To each set $x$ we assign an ordinal number --- its \emph{rank} $\tau(x)$:
 $$ %\startformula
  \tau(x)=min\{\alpha:x\subseteq V_\alpha\}.
 $$ %\stopformula
  It is easily seen that $\tau(x)=sup\{\tau(y)+1:y\in x\}$ for any set $x$.
\end{definition}

\begin{fact}\label{valphaFact} Suppose $\alpha<\beta$ are ordinal numbers. Then
 \begin{itemize}
  \item[(i)] $V_\alpha$ is a transitive set.
  \item[(ii)] $V_\alpha\in V_\beta$.
  \item[(iii)] $On\cap V_\alpha = \alpha$, hence $\alpha\subseteq V_\alpha$ but
	$\alpha\not\in V_\alpha$.
 \end{itemize}
\end{fact}
\begin{proof} (i) Given $x\in V_\alpha$, let $\beta$ be minimal such that $x\in V_\beta$. Clearly $\beta=\gamma+1\leq\alpha$. By the definition of
$V_\beta$ we have that $x\in \mathcal{P(V_\gamma)}$ so $x\subseteq V_\gamma\subseteq V_\beta\subseteq V_\alpha$. (ii) is clear since
$x\in\mathcal{P}(x)$ (iii) We show by induction that $\alpha\subseteq V_\alpha$ and $\alpha\not\in V_\alpha$: this is clear for $\emptyset$. Suppose we
have proven this for all $\beta<\alpha$. If $\alpha$ is limit then by the inductive hypothesis clearly $\alpha\subseteq V_\alpha$ and
easily $\alpha\not\in V_\alpha$. If $\alpha=\beta+1$, then $\alpha=\beta\cup\{\beta\}$ and since $\beta\subseteq V_\beta\subseteq V_\alpha$
by the inductive hypothesis and hence $\beta\in V_\alpha$ by the definition of $V_\alpha$. Similarly $\alpha\not\in V_\alpha$.
\end{proof}

% \begin{definition} Given a set (or class) $M$ and a formula $\varphi$ of set theory, we define the \emph{relativization} $\varphi^M$ of
%  $\varphi$ to $M$ by induction on the complexity of $\varphi$ as follows:
%  \begin{itemize}
%   \item[(i)] If $\varphi$ is atomic, then $\varphi^M$ is $\varphi$,
%   \item[(ii)] if $\varphi$ is of the form $(\forall x)\psi$ then $\varphi^M$ is $(\forall x)(x\in M\rightarrow \psi^M)$ and
%   \item[(iii)] if $\varphi$ is of the form $\neg\psi$ then $\varphi^M$ is $\neg(\psi^M)$.
%  \end{itemize}
%  If $\varphi$ is a closed formula, we say that $M$ satisfies $\varphi$ iff $\varphi^M$ is provable in ZFC. In this case
%  we say that $M$ is a \emph{syntactical model} for $\varphi$.
% \end{definition}

\begin{proposition}\label{valphaApprox}
\begin{itemize}
 \item[]
 \item[(i)] If $\alpha>\omega$ is a limit ordinal number, then $V_\alpha$ satisfies all axioms
 of set theory except for the (full) replacement schema. The class of ordinal numbers in the sense of $V_\alpha$ is
 $\alpha=V_\alpha\cap On$.
 \item[(ii)] $V_\omega$ consists of hereditarily finite sets and
	$(V_\omega,\in)$ is a model of $\mbox{ZF}_{\mbox{fin}} \ $,
	the theory of finite sets. It turns out that it is ``equivalent''
	to the standard model of Peano Arithmetic.
\end{itemize}
\end{proposition}




\begin{definition}\label{H-kappa}{\bf The $H(\kappa)$ hierarchy.} Assume AC. If $\kappa$ is an infinite cardinal number we define
$H(\kappa)$ as follows:
$$ %\startformula
 H(\kappa) = \{x:|\mbox{tcl}(x)|<\kappa\}.
$$ %\stopformula
The set $H(\kappa)$ consists of sets which are hereditarily of cardinality  less then
$\kappa$. The universe $H(\omega_1)$ of hereditarily countable
sets is sometimes denoted by $HC$. Note that $H(\kappa)$ is a set for each $\kappa$.
\end{definition}

In ZFC we again have the following:

\begin{proposition}{\rm (AC)}
 \begin{itemize}
  \item[(i)] Each $H(\kappa)$ is a transitive set.
  \item[(ii)] $V=\bigcup\{H(\kappa):\kappa\ \mbox{is a cardinal number}\}$
 \end{itemize}
\end{proposition}

\begin{definition}\label{strong-limit} Recall, that a cardinal number $\kappa$ is \emph{strongly limit} if
if $2^\lambda<\kappa$ for any $\lambda<\kappa$. An uncountable regular strongly limit cardinal $\kappa$
is called \emph{inaccessible}.
\end{definition}
\begin{proposition}{\rm (AC)} If $\kappa$ is inaccessible or $\kappa=\omega$ then
    $$H(\kappa)=V_\kappa$$
\end{proposition}

Note that the existence of an inaccessible cardinal is not provable in ZFC, however $H(\kappa)=V_\kappa$
for any fixed point of the function $\beth(\kappa)=2^{<\kappa}$.

\begin{proposition}{\rm (AC)}\label{Hkappa2}
\begin{itemize}
 \item[(i)] If $\kappa>\omega$ is a regular cardinal, then $(H(\kappa),\in)$ satisfies all axioms of set theory
 except for the power set axiom 5., i.e. it is a model of $ZFC^-$. (cf. Proposition \ref{valphaApprox}.)
 \item[(ii)] The role of the class of ordinal numbers in the structure $(H(\kappa),\in)$ is played by $\kappa$. %(cf. \in{fact}[valphaFact],iii.)
 \item[(iii)] If a set $b$ is in $H(\kappa)$, then $\pw(b)\subseteq H(\kappa)$.
 \item[(iv)] If $b\in H(\kappa)$ then $\pw(b)\in H(\kappa)$ iff $2^{|b|}<\kappa$.
\end{itemize}
\end{proposition}
\begin{note} Even thought $H(\kappa)$ need not satisfy the full power-set axiom, (iv) tells us that
if we restrict ourselves to sets of ``small'' cardinality --- viz. of cardinality $\lambda$ such that $2^\lambda <\kappa$ ---
the power-set axiom holds.
\end{note}

\subsection{${}$ \hspace{-1em}$H(\kappa)$ as models of fragments of ZFC}
For the purpose of proving
consistency results a model of full ZFC is not needed. More precisely: every proof is a finite
chain (finite in the met\hbox{a-t}heory) of consecutive arguments. Given any finite fragment of ZFC,
there are unboundedly many cardinal numbers $\kappa$ such that $(H(\kappa),\in)$ is a model of
this fragment --- this is the essence of the reflection principle.

Now of course one does not specify the exact fragment of ZFC one needs. The main point
is that it is, at least theoretically, possible to do this. This is the content of the oft
seen phrase

\begin{center}
 ``take sufficiently large $\kappa$''
\end{center}
which in fact means: take $\kappa$ large enough so that $(H(\kappa),\in)$ satisfies
enough axioms to guarantee all the arguments and constructions which we need.

\subsection{${}$ \hspace{-1em}Well-founded relations.}

\begin{definition}\label{ext}
A binary relation $R$ on a set $A$ is \emph{well-founded}
iff each nonempty subset of $A$ contains at least one
 $R$-minimal element, that is for any $\emptyset\neq a\subseteq A$ there is an $x\in a$ with
 $$ %\startformula
  Ext_R(x) \cap a = \emptyset,\quad\hbox{where}\quad Ext_R(x)=\{ y:\langle y,x\rangle\in R\}.
 $$ %\stopformula
\end{definition}

A strict wellordering is an example of a well-founded relation. Well-founded relations
are a generalization of wellorders which are themselves a generalization of the $\in$
relation, viz the axiom of foundation. We shall show (see \ref{WF-recursion}), that recursive
constructions can be carried out along general well-founded relations much in the same way as along $\in$.

\begin{definition}\label{DC}
{\bf Dependent Choice.} The axiom of dependent choice is a weaker
version of the axiom of choice. It says, that
whenever some relation $R$ satisfies
$$ %\startformula
  (\forall x\in dom(R))(\exists y)(yRx)
$$ %\stopformula
then there is an $R$-descending sequence $\langle x_n:n<\omega\rangle$,
i.e. a sequence satisfying $x_{n+1}R x_n$ for each $n<\omega$.
The axiom of dependent choice is abbreviated DC.
\end{definition}

\begin{fact} If we assume DC then a relation $R$ is well-founded iff there are no infinite $R$-decreasing chains.
\end{fact}


% \begin{definition}
% A relation $R$ is \emph{set-like} iff for every set $x$ the class $Ext_R(x)$ is a set.
% \end{definition}

% Well-founded set-like relations are generalizations of the $\in$ relation. The theorems on transfinite induction and recursion can be
% generalized in this setting to induction and recursion over set-like well-founded relations.

% \begin{example}
% \begin{itemize}
%  \item[(i)]  The lexicographical ordering $<_1$ on $On\times On$ is an example of a relation which is well-founded but is not set-like.
%  \item[(ii)] The maximo-lexicographical ordering $<_2$ on $On\times On$ is a well-founded set-like relation:
% %\startformula
% %\startalign[n=3,align={left,middle,left}]
% \begin{align}
% {} & \langle \alpha_1,\beta_1\rangle \leq_2 \langle \alpha_2,\beta_2\rangle & \equiv& \max(\alpha_1,\beta_1) < \max(\alpha_2,\beta_2)\ \hbox{or}\ \\
%                                                                           &&&(\max(\alpha_1,\beta_1) = \max(\alpha_2,\beta_2)\ \&\ %\\ &&&
%                                                                           \min(\alpha_1,\beta_1)< \min(\alpha_2,\beta_2) )
% \end{align}
%  %\stopformula
% \end{itemize}
% \end{example}

\subsection{${}$ \hspace{-1em}Well-founded induction and recursion.}


% \begin{theorem}\label{WF-induction}
% {\rm (Well-founded induction).}
% Suppose $R$ is a well-founded %and set-like
% relation on $A$. Let $X \subset A$ contain the $R$-minimal
% element of $A$ and $X$ satisfies ``the induction
% hypothesis'' $Ext_R(a) \subset X \rightarrow a \in X$. Then
% $X=A$.
% % Furthermore suppose $\phi$ is a formula such that whenever for each $a\in A$ if
% % $(\forall b\in Ext_R(a))\varphi(b) \rightarrow \varphi(a)$. Then $(\forall a\in A)\varphi(a)$.
% \end{theorem}

\begin{theorem}\label{WF-recursion}
{\rm (Well-founded recursion).} Suppose $R$ is a well-founded and set-like
relation on $X$ and $G$ is a mapping defined on $V$.
Then there exists a unique function $F$ defined on $X$ which satisfies
$$ %\startformula
 %F(a) = G(F\upharpoonright Ext_R(a) )\quad\hbox{for each}\ a\in A.
 F(x) = G(\{F(y):y R x \}), \ \mbox{for each}\ x \in X.
$$ %\stopformula
\end{theorem}

% \begin{corollary}
% It follows, as a special case, that there exists a unique function $F_1$ such that
%  $$ %\startformula
%  F_1(a) = G(\{F_1(y):b R a \}).
% $$ %\stopformula
% \end{corollary}

\subsection{${}$ \hspace{-1em}Rank function.}
Every well-founded relation $R$ on $X$ has an associated
\emph{rank function} $\rho_R:X\to On$ which
is defined using well-founded recursion as:
$$ %\startformula
 \rho_R(x)=\hbox{sup}\{\rho_R(y)+1:y R x\},\ \sup \emptyset = 0.
$$ %\stopformula

\smallskip

If $X$ is a set, then the range of $\rho_R$ is an ordinal number and is called the \emph{height} of $R$ on $X$.
If $R$ is a well-ordering, then the height of $R$ is also called the \emph{order type} of $R$.

\smallskip

We shall now introduce the often used concept of a tree and show an application of the rank function on ill-founded trees.


\begin{definition}\label{tree}
{\bf Tree.} A nonempty set $T$ together with a partial order $\leq$ forms a \emph{tree} iff
for each $t\in T$ the set of its predecessors $pred_T(t)=\{s\in T:s< t\}$ is well-ordered by $<$.
\begin{itemize}
 \item[(i)] Similarly to predecessors, we denote $succ_T(t)$ the set of all immediate successors of node $t$ in $T$.
 \item[(ii)] For each $t\in T$ we define the \emph{height} $h_T(t)$ of $t$ as the order type of $(pred_T(t),<)$.
 \item[(iii)] The $\alpha$-th level of $T$, denoted by $T_\alpha$,  is defined as $T_\alpha=\{t\in T:h_T(t)=\alpha\}$.
 \item[(iv)] The height of $T$ is defined to be the rank of $<$ and is equal to $\min\{\alpha:T_\alpha=\emptyset\}$.
 \item[(v)] A maximal linearly ordered subset of $T$ is called a \emph{branch} of $T$. The set of all branches of a tree $T$ will be denoted by $[T]$.
\end{itemize}

\smallskip

By a \emph{subtree} $T' \subset T$ we understand a nonempty subset $T'$ of $T$
which is closed under initial segments, i.e. $(\forall t \in T')$ if $s \leq t$ then $s \in T'$.
\end{definition}


\begin{example}\label{Seq}
Trees of height $\omega$ with a single root, i.e. a smallest element, have a nice representation.
For a set $X$ define $Seq(X)={}^{<\omega}X=\bigcup\{{}^nX:n<\omega\}$,
the set of all finite sequences of
elements of $X$. In case $X=\omega$ we write just $Seq$. A set
$T\subseteq Seq(X)$ closed under initial segments,
i.e. if $f\in T$ and $n\in dom f$ then $f\upharpoonright n\in T$,
is called a tree on $X$. If $X=\kappa$, we say $T$ is a
\emph{$\kappa$-ary tree}, if $X=2$, $T$ is \emph{binary tree}.
If $T$ is a tree on $X$ then  $(T,\subseteq)$ is a tree.

Given a tree $T$ on $X$ we shall identify branches with
elements of ${}^\omega X$ and let $[T]=\{f\in {}^\omega X:(\forall n<\omega)(f\upharpoonright n\in T)\}$
be the set of all branches of $T$.

\end{example}


\begin{definition} A tree is called \emph{well-founded} if the reverse order $>$ is well-founded, otherwise it is \emph{ill-founded}. Beware that we take the
\emph{reverse} order $>$, since $<$ is well-founded by definition!
\end{definition}

\begin{fact}{\rm {(ZF)} }
\begin{itemize}
 \item[(i)] Assume DC. Then a tree is ill-founded iff it has an infinite branch.
 \item[(ii)] Assume DC. A tree of height omega with finite levels must have an infinite branch.
 \item[(iii)] A well-founded binary tree is finite
\end{itemize}


\end{fact}


\begin{example} If $T$ is a well-founded $\omega$-ary tree, then the reverse
inclusion relation on $T$ is
 well-founded, so it has an associated rank function $\rho_T$.
We can then define the rank $r(T)$
 of $T$ to be $\rho_T(\emptyset)$.
\end{example}

\begin{example}{\rm {(DC)} }
If $T$ is a well-founded $\omega$-ary tree, then $r(T)<\omega_1$ and for any $\alpha<\omega_1$
 there is a well-founded $\omega$-ary tree $T^\alpha$ with $r(T^\alpha)=\alpha$.

\medskip

To see that $r(T)<\omega_1$ note that $|T|$ is countable. For the opposite direction
 we proceed by induction.
A tree of rank $1$ is just $\{\emptyset\}$. Suppose $\alpha<\omega_1$ and we have constructed a tree $T^\beta$ of rank $\beta$ for each $\beta<\alpha$.
To construct a tree of rank $\alpha$, take a tree $T$ of height $2$ with $|T_1|=\omega$ and glue the $T^\beta$'s to the branches of $T$ to
get $T^\alpha$.
%
% $we have a tree $T$
% A tree of finite rank is easy to build. A tree of rank $\omega$ is just an infinite union of branches of increasing length.
%
% if $\alpha<\omega_1$, choose for each limit $0<\beta\leq\alpha$ a strictly increasing sequence $\beta_n$
% with $\beta=\bigcup\{\beta_n:n<\omega\}$. We define by induction on $\alpha$ a function $f$ on the full $\omega$-ary tree
% ${}^{<\omega}\omega$ as follows: $f(\emptyset)=\alpha$. If $f(t)=\beta+1$ then $f(t\conc n)=\beta$ for each $n<\omega$.
% If $f(t)=\beta$ and $\beta$ is limit and nonzero, then $f(t\conc n)=\beta_n$. If $f(t)=0$ then $f(t\conc n)=0$.
% Finally we let $T_\alpha=\{t\in{}^{<\omega}\omega:f(t)>0\vee f(t)=0\ \&\ (\forall s\subset t)(f(s)\neq 0)\}$. It
% is easily seen that $r(T_\alpha)=\alpha$ and that $f\upharpoonright T_\alpha$ is the canonical rank function $\rho_{T_\alpha}$.
\end{example}

The following example will be useful in chapter \ref{chapter_generic} when we will compare
transitive models of ZFC.

\begin{example}\label{transitive_closure}
Given a set $A$, let $Tc(A)$ be the set of all $\in$-descending sequences beginning in
A. The set $Tc(A)$ together with inclusion is a tree. By the axiom of foundation
the height of $(Tc(A),\subseteq)$ is at most $\omega$ and the tree has no infinite
branches. For convenience we shall add the empty sequence to $Tc(A)$ so that $Tc(A)$
has a single root.
\end{example}

Given any well-founded tree $T$ we can define the following two functions on $T$:
\begin{eqnarray*}
 f_T(t)  & = &  \{ \ f(t') : \ t' \in \mbox{succ}_T (t) \ \}, \\
 g_T(t)  & = &  \sup\{ \ g(t') + 1 : \ t' \in \mbox{succ}_T (t) \ \}, \
	\mbox{where} \ g(\mbox{leaf}) = 0.
\end{eqnarray*}
and note that $f_T(s)=\emptyset$ if $s$ is a leaf of $T$.

\begin{theorem} The tree $Tc(A)$ codes the set $A$ in the sense that
$f_{Tc(A)}(\emptyset) = \{A\}$. Moreover given any tree $T$ isomorphic to $Tc(A)$,
$f_T(root(T))=\{A\}$, $g_T(root(T)) = g_{Tc(A)}(\emptyset)=rk(A)+1=\min\{\alpha:A\subseteq V_\alpha\}$
and $rng(f_T)=Tcl(\{A\})$.
\end{theorem}


%
% Any set $X$ determines a tree $(Tc(X),\subseteq)$ consisting of
% finite sequences of sets.
% \begin{itemize}
%  \item[(i)] The empty sequence (sequence of length $0$) belongs to
% 	$Tc(X)$.
%  \item[(ii)] The sequence $\langle X \rangle$ of length $1$ belongs to
% 	$Tc(X)$.
%  \item[(iii)] If sequence $\langle X_0,X_1,X_2,\dots, X_n \rangle$ belongs
% 	to $Tc(X)$, then sequence $\langle X_0,X_1,X_2,\dots, X_n, X_{n+1}\rangle$
% 	belongs to $Tc(X)$ provided $X_{n+1} \in X_n$.
% \end{itemize}
% The sequences are ordered by inclusion. By the axiom of foundation,
% the height of $Tc(X)$ is at most $\omega$ and each leaf of the tree
% is an empty set. It is easy to see, that \emph{transitive closure}
% $\mbox{tcl}(X)$ of set $X$ is equal to $\bigcup \{rng(A) : A \in Tc(X)\}$.
%
% \smallskip
%
% Take any tree $(T,\leq)$ isomorphic to $Tc(X)$. Using WF-recursion
% we can define the following functions on $T$.
% \begin{eqnarray*}
%  f(t)  & = &  \{ \ f(t') : \ t' \in \mbox{succ}_T (t) \ \}, \\
%  g(t)  & = &  \sup\{ \ g(t') + 1 : \ t' \in \mbox{succ}_T (t) \ \}, \
% 	\mbox{where} \ g(\mbox{leaf}) = 0.
% \end{eqnarray*}
% Then we have that $f(\mbox{root}) = \{X\}$ and
% $g(\mbox{root}) = \min \{\alpha : x \in V_\alpha \}$.
% \end{example}

% \begin{example}\label{Cantor_Baire}
% Fundamental examples of trees are
% sets of all finite
% sequences of of natural numbers ${}^\omega \omega$ denoted as
% $\mbox{Seq}$
% or finite sequences of $\{0,1\}$ denoted as
% and $\mbox{Seq}(2) = {}^\omega 2$, where the tree ordering is the inclusion.
% $\mbox{Seq}(2)$ is an example of binary tree and
% $\mbox{Seq}$ of $\omega$-ary tree.
% By the definition
% $\mbox{Seq}(2)$ is a subtree of $\mbox{Seq}$.
% The sets of functions ${}^\omega \{0,1\}$ and ${}^\omega \omega$
% can be equipped with natural topology.
%  $$ %\startformula
% d(f,g)=\frac{1}{k+1}, \ \mbox{where} \ k=\min\{n:f(n)\neq g(n)\}.
% $$ %\stopformula
% Then the topological space ${}^\omega \omega$, known as \emph{Baire space} and
% denoted by $\mathcal N$, is homeomorphic to the irrational numbers,
% while ${}^\omega \{0,1\}$ with the very same topology
% is known as \emph{Cantor space}, denoted by $\mathcal C$ and is a compact subspace
% of $\mathcal N$. ${\mathcal N}$ is a classical example of a non-compact Polish space.
% Note that $[Seq] = {}^\omega \omega$ and $[Seq(2)] = {}^\omega \{0,1\}$.
% \end{example}

We now turn to an important theorem that will be useful later on, but first we need a definition.

\begin{definition}
A relation $R$ on a set $A$ is extensional if for $a\neq b\in A$ we have that $Ext_R(a)\neq Ext_R(b)$, where
$Ext_R(a)=\{x\in A: xRa\}$ (see \ref{ext}).
\end{definition}

Note that the axiom of extensionality is equivalent to saying that $\in$ is extensional on $V$.

\begin{theorem}\label{mostowski}
{\scshape (Mostowski collapse).} Suppose $R$ is a well-founded
%, set-like relation on a class
relation on a set $A$.
\begin{itemize}
\item[(i)] There is a transitive set $M$ and a surjective mapping $F:A\to M$ such that $xRy\rightarrow F(x)\in F(y)$.
\item[(ii)] Moreover if $R$ is extensional on $A$, then the mapping $F$ is an isomorphism.
\end{itemize}
\end{theorem}

\begin{proof}
(hint) Using well-founded recursion, define $F(a)=\{F(b):b\in A\ \&\ bRa\}$. To show that $F$ is as required, use well-founded induction.
\end{proof}

% \startlemma Suppose that $M,N$ are transitive models of ZF and one of them satisfies the Axiom of choice.
%  Suppose, moreover, that they have the same subsets of ordinal numbers, i.e. $\pw^M(Ord^M)=\pw^N(Ord^N)$.
%  Then $M=N$.
% \stoplemma
% \begin{proof} Without loss of generality assume $M\models AC$. First we show that $M\subseteq N$. Let $X\in M$ and, since $M\models AC$
% there is bijection $f\in M$ of some ordinal $\delta\in M$ onto the transitive closure of $X$.
% Define a relation $E$ on $\delta$ as follows: $\alpha E\beta\equiv f(\alpha)\in f(\beta)$. Then, by our
% assumption, $E\in N$. Also $E$ is extensional and well-founded in $M$ and, by absoluteness, also in $N$. Applying
% Mostowski collapsing theorem, there is a transitive set $T\in N$ such that $(E,\delta)\equiv (T,\in)$. So also
% $(T,\in)\equiv (Tcl(X),\in)$. But since both $T$ and $Tcl(X)$ are transitive, necessarily $T=Tcl(X)$. So
% $X\in N$.
%
% Now we show $M=N$ by transfinite induction. Assume $X\in N$ and $X\subseteq M$. Find $\alpha\in Ord$ such
% that $X\subseteq V_\alpha^M$ and let $f\in M$ be a bijection from $V_\alpha^M$ into $Ord$. Now since
% $f[X]$ is a set of ordinals and hence $f[X]\in M$ we know that $X=f^{-1}[f[X]]\in M$.
% \end{proof}


\subsection{${}$ \hspace{-1em}Elementary substructures.}

We will focus on countable substructures of the uncountable structures
$(H(\kappa),\in)$ and universal algebras
$A=\langle A,\{f_i:i\in I\}\rangle$. We shall tacitly assume that the set of operations is at most
countable and each of them is finitary. Suppose we have such an algebra $A$.
A nonempty subset $X\subseteq A$ closed under all these operations determines a
subalgebra and whenever $Y\subseteq A$ is infinite, there is an $X\subseteq A$
which is closed under the operations,
contains $Y$ and $|Y|=|X|$. We assume the axiom of choice throughout this section.

\bigskip
Let us first recall the classical definition of a club set on $\omega_1$. The cardinal number
$\omega_1$ ordered by $\in$ carries a natural topology derived from the order. A club on $\omega_1$
is an unbounded subset of $\omega_1$ which is closed in this topology, i.e. is closed under
taking suprema. A standard argument shows that a countable intersection of club sets is again
a club set. A subset of $\omega_1$ is called stationary iff it intersects each club set. The following
is a classical theorem about stationary subsets, which was subsequently generalized by R. Solovay to
higher cardinalities.

\begin{theorem}[Fodor] Any stationary subset of $\omega_1$ can be partitioned into $\omega_1$-many
 disjoint stationary subsets of $\omega_1$.
\end{theorem}

In the context of forcing it turns out that the following generalization of club sets to
different structures is very convenient.

\begin{definition}\label{club}
{\bf Club.} Let $A$ be an infinite set. A family $C\subseteq[A]^{\omega}$ is
\emph{closed unbounded}, or \emph{club}
for short, in $[A]^\omega$ if it is
\begin{itemize}
  \item[(i)] \emph{unbounded,} i.e.
	$(\forall Y\in[A]^\omega)(\exists X\in C)(Y\subseteq X)$ and
  \item[(ii)] \emph{closed,} i.e. for any increasing chain $\{X_n:n<\omega\}$ of elements of
	$C$ the union $\bigcup\{X_n:n<\omega\}$ is again an element of $C$.
\end{itemize}
\end{definition}

\begin{note} In our definition we have only assumed that $A$ is infinite. If it is countable, however, the notion is not very interesting, in particular
the singleton $\{A\}$ is a club.
\end{note}

\begin{note} Observe that if $C$ is a (classical) club on $\omega_1$ then it is also a club in $[\omega_1]^{\omega}$
(more precisely $\{\alpha:\alpha\in C\ \&\ \omega\leq\alpha\}$ is a club in $[\omega_1]^{\omega}$).
\end{note}

\begin{fact}
Suppose $A$ is an infinite set. It is relatively easy to prove that the intersection
of a countable system of clubs in $[A]^{\omega}$ is again
a club in $[A]^{\omega}$. It follows that the system of clubs in $[A]^{\omega}$
generates a $\sigma$-complete filter.
\end{fact}
%
% \begin{fact}{} It is relatively easy to prove the following basic properties. Suppose $A$ is infinite. Then
% \begin{itemize}[i]
% \item[()] The intersection of a countable system of clubs in $[A]^{\omega}$ is a club in $[A]^{\omega}$.
% \item[()] The system of clubs in $[A]^{\omega}$ generates a filter.
% \end{itemize}
% \end{fact}

The following proposition is an algebraic version of the well-known L\"owenheim~-~Skolem theorem from model theory.

 \begin{proposition}\label{algebraic-club-A}
Suppose $A=\langle A,\{f_n:n<\omega\}\rangle$ is an algebra with infinite underlying set $A$.
Then the family of all  countable subalgebras forms a club in $[A]^{\omega}$.
 \end{proposition}

The converse of the proposition also holds (see e.g. \cite{JechST}):

 \begin{proposition}\label{algebraic-club-B}
Suppose that $C$ is a club in $[A]^{\omega}$, then there are operations
$\{f_n:n < \omega\}$ on $A$ such that
 $C$ contains all countable subalgebras of $A$ with given operations.
 \end{proposition}


Having defined closed unbounded families, we can define stationary families.

\begin{definition}\label{stationary}
{\bf Stationary set.} A family $S\subseteq [A]^{\omega}$ is \emph{stationary} if it intersects each club,
i.e. if $C$ is a club in $[A]^{\omega}$, then $S\cap C\neq\emptyset$.
\end{definition}

It follows from propositions \ref{algebraic-club-A}, \ref{algebraic-club-B} that:

% \begin{corollary} A set $S\subseteq [A]^\omega$ is stationary iff for every system $\{f_n:n<\omega\}$ of finitary operations on $A$,
% $S$ contains a set which is closed under these operations.
% \end{corollary}

Consider now the structure $(H(\kappa),\in)$.

\begin{definition}\label{elementary-substructure}
{\bf Elementary substructure.}
Let $X\subseteq H(\kappa)$ be a countable set. We shall say that
$X$ is an \emph{elementary substructure} of $H(\kappa)$, denoted $(X,\in)\preceq (H(\kappa),\in)$, if for each property expressed
by a formula of Set Theory $\varphi(v_1,\ldots,v_n)$ with free variables $v_1,\ldots, v_n$, the following holds:
%\placeformula
$$ %\startformula
(\forall x_1,\ldots,x_n\in X)( (X,\in)\models \varphi[x_1,\ldots,x_n] \equiv (H(\kappa),\in)\models \varphi[x_1,\ldots,x_n] )
$$ %\stopformula
This implies, in particular, that sentences (i.e. closed formulas) of Set Theory are valid in $(X,\in)$ iff they are valid in $(H(\kappa),\in)$.
\end{definition}

\noindent{\bfseries\scshape Relativity principle.}
The concept of an elementary substructure is akin to the Galileo's relativity principle in physics. This principle states that, using physical
experiments, we cannot decide which frame of reference we are in. Similarly, using Set Theoretical properties, we cannot distinguish
whether we are in the elementary substructure or the superstructure. In other words, the validity of statements about elements of $X$
cannot distinguish between $X$ and $H(\kappa)$.


% \begin{note} Note that when deciding the validity of formulas in $X$ the quantifiers range only over $X$ whereas in $H(\kappa)$ they range
% over all of $\kappa$. In particular existential formulas are 'easier to satisfy' in $H(\kappa)$ whereas universal formulas are ``easier to
% satisfy'' in $X$.
% \end{note}

\begin{fact}\label{elementary-fact}
For any $X\preceq H(\kappa)$ the following basic facts are true:
\begin{itemize}
  \item[(i)] $\omega\subseteq X$ since each natural number is definable in $H(\kappa)$ and
  \item[(ii)] $\omega\in X$, since $\omega$ is also definable in $H(\kappa)$,
\end{itemize}
more generally
\begin{itemize}
  \item[(iii)] If $Y\in X$ and $Y$ is countable, then $Y\subseteq X$.
\end{itemize}
\end{fact}

\begin{note}
Countable elementary substructures typically are not transitive sets, and for $\kappa>\omega_1$ are provably not transitive.
             It is hence perfectly possible that some uncountable set is an \emph{element} of the countable substructure. In fact,
             $\omega_1\in X$ whenever $X$ is an elementary substructure of $H(\kappa)$ for $\kappa>\omega_1$.
\end{note}

As with algebras, we have the following often used and important theorem.

\begin{theorem} Suppose $\kappa>\omega$. Then
\begin{itemize}
  \item[(i)] for any countable subset $Y\subseteq H(\kappa)$, there is an
	$X\preceq H(\kappa)$ such that $Y\subseteq X$ and
  \item[(ii)] the set of countable elementary substructures of $H(\kappa)$
	forms a club in $[H(\kappa)]^\omega$.
\end{itemize}
\end{theorem}

Stationary sets on $[A]^\omega$ are important in the theory of proper forcing. For more information on elementary substructures
and their usage e.g. in topology see \cite{DowST}. Here we mention an application in set theory

\begin{theorem} It $\kappa$ is an inaccessible cardinal, then $V_\kappa = H(\kappa)$ and
 $V_\kappa$ is a model of ZFC. Moreover there exists a cardinal number $\lambda<\kappa$
 with countable cofinality such that $V_\lambda$ is a model of ZFC.
\end{theorem}
\begin{proof} By induction find $\alpha_n<\omega$ and elementary substructures $X_n\preceq H(\kappa)$
such that $X_n\subseteq V_{\alpha_n}\subseteq X_{n+1}$. Then $\lambda=\sup\{\alpha_n:n<\omega$ is as required.
\end{proof}


\subsection{${}$ \hspace{-1em} Elementary substructures as proof tools.}
If we want to prove that a given set $a$ has property $\varphi$
the following argument is sometimes useful. We consider a ``sufficiently
large $\kappa$'' such that $a\in H(\kappa)$. Then, taking an elementary
substructure $(X,\in,a)\prec (H(\kappa),\in,a)$, it is sometimes
easier to prove that
$$(X,\in,a) \vDash \varphi (a),$$
By elementarity we get $(H(\kappa),\in,a)\vDash\varphi(a)$ and, since
$\kappa$ was ``sufficiently large'' then also $\mbox{ZFC}\vdash\varphi(a)$.

\subsection{${}$ \hspace{-1em} Interpretations \& finitary consistency proofs.}

We shall now sketch an effective method for proving relative consistency. This
method was used by G\"{o}del to prove the relative consistency with ZF of CH and the
Axiom of choice.

First we introduce the notion of \emph{interpretation} of the language of Set Theory in
Set Theory itself. The language of Set Theory is determined by two basic binary predicates:
equality ($=$) and membership ($\in$). Using these two predicates we may
define, as is customary, further predicates (e.g. $\cup,\emptyset,\omega,\mathbb{R},\ldots$).

To interpret this language in Set Theory, we need to define a class $U$ and two binary relations,
which we shall suggestively call $=^*$ and $\in^*$. We also require that the following basic conditions
are satisfied, i.e. provable in Set Theory,

\begin{itemize}
 \item[(a)] $(\exists x)U$,
 \item[(b)] $\mathcal{=^*}$ is an equivalence relation on $\{x:U(x)\}$ and
 \item[(c)] $(\forall x,y,z)(U(x)\ \&\ U(y)\ \&\ U(z)\rightarrow \big((x\in^*y\ \&\ x=^*z\rightarrow z\in^*y)\ \&\
								      (x\in^*y\ \&\ y=^*z\rightarrow x\in^*z)\big))$.
\end{itemize}

Then we can translate any formula $\varphi$ of Set Theory into the corresponding formula $\varphi^*$ ---
the interpretation of $\varphi$ --- by replacing all occurances of $=$ with $=^*$, $\in$ with $\in^*$ and
$\forall x$ with $(\forall x\in U)$. The precise definition is, of course, by
induction on the complexity of $\varphi$:

\begin{itemize}
 \item[] If $\varphi$ is $x\in y$, then $\varphi^*$ is $x\in^* y$,
 \item[] if $\varphi$ is $x = y$ then $\varphi^*$ is $x=^*y$ and
 \item[] if $\varphi$ is $\neg(\psi)$ then $\varphi^*$ is $\neg(\psi^*)$,
 \item[] if $\varphi$ is $\psi\rightarrow\chi$ then $\varphi^*$ is $\psi^*\rightarrow\chi^*$ and
 \item[] if $\varphi$ is $(\forall x)\psi$ then $\varphi^*$ is $(\forall x)(U\rightarrow\psi^*)$.
\end{itemize}

We shall say that $(U,=^*,\in^*)$ is a \emph{syntactical model} of a theory $T$ if any axiom of
$T$ translates into a provable formula. We say that the model \emph{satisfies} a sentence $\varphi$ if $\varphi^*$
is provable.

\begin{metatheorem} Suppose $(U,=^*,\in^*)$ is a syntactical model of Set Theory which satisfies
some sentence $\varphi$. Then $\varphi$ is consistent with Set Theory.
\end{metatheorem}
\begin{proof} We assume Set Theory is consistent, otherwise the theorem is clear.
Given a sentence $\psi$ provable in Set Theory by induction on the lenght of its proof we
show that it is satisfied by $U,=^*,\in^*$. Now if $\varphi$ were inconsistent,
then $\neg\varphi$ would be provable so also $(\neg\varphi)^*\equiv\neg(\varphi^*)$ would be
provable. Since we assumed $\varphi^*$ is provable we have shown that Set Theory itself is inconsistent
a constradiction.
\end{proof}

A common special case of interpretations is that we only define the class $U$ and let $=^*,\in^*$
be just $=,\in$. In that case the translated formula $\varphi^*$ is often denoted as $\varphi^U$.

\begin{example} If $U\neq\emptyset$ is a transitive class and $\varphi$ is the axiom of extensionality, then
 $\varphi^U$ is provable. Let us call this interpretation transitive. A formula with $\varphi$ with
 free variables $x_0,\ldots,x_n$ is called \emph{absolute} for an interpretation if
 $$
 (\forall x_0,\ldots,x_n\in U)(\varphi^*\leftrightarrow\varphi).
 $$
 Transitive interpretations have the advantage that many formulas are absolute.
 For example $z=\{x,y\}$ is absolute so for any $z\in U$ we have ``(is a pair)$^U$''
 iff $z$ is a pair. This allows us to simplify many arguments in the case of transitive interpretations.
\end{example}

\begin{example}[G\"{o}del] Let $L$ be the class of constructible sets (see e.g. \cite{JechST}).
$L$ is a transitive class.Then if $\varphi$ is an axiom of ZF, then $\varphi^L$ is provable.
Moreover GCH$^L$ and AC$^L$ are provable. It follows that the continuum hypothesis and the
axiom of choice are both consistent with ZF.
\end{example}

\subsection{${}$ \hspace{-1em}Semisets.}
We now introduce a working term which will be used later on.
Please keep in mind that this concept is not generally known in the wide
Set-Theory circles. Nevertheless it is a simple notion and appears to be quite handy for
generic extension. For more details and in-depth discussion of the concept
see \cite{VoHa:72}. For our purposes we shall not assume --- or need --- any
knowledge of this book.

As a motivation for the definition consider the axiom of power set.
Conventionally it is understood to say that the collection of all parts of a set
forms a set. The following interpretation, which leads to the definition of a
semiset, is equally valid: The axiom of powerset says that the collection of
all parts of a set \emph{which are themselves a set} forms a set.

\begin{definition}\label{semiset}
{\bf Semiset.} A \emph{semiset} is a part of a set. A semiset which is not
a set is \emph{proper}.
\end{definition}

Note that each set is a semiset and in our Set Theory there are no proper semisets.
We will now describe two natural scenarios which lead to this notion.


\subsubsection{Motivating semisets.}
{\bf I.} Take an infinite cardinal $\lambda$ and consider $H(\kappa)$ where
$\kappa= (2^\lambda)^+$. We know that
$(H(\kappa),\in)$ is a model of ZFC${}^-$ (see Proposition \ref{Hkappa2}, (i)).
In our special case,
$\pw(\lambda)\in H(\kappa)$ so the power set axiom holds for
$\lambda$. Let $X$ be a countable elementary substructure of $H(\kappa)$
with $\lambda\in X$.

Then $(X,\in)$ is a model for ZFC${}^-$, $\lambda,\pw(\lambda)\in X$ and
$\in$ is a well-founded extensional relation
on $X$. Applying the Mostowski collapse (see Theorem \ref{mostowski})
to obtain a transitive set $M$ and an isomorphism $F$ of $(X,\in)$
and $(M,\in)$. $(M,\in)$ is again a countable model of ZFC${}^-$.
All subsets of $F(\lambda)$ which are not elements
of $M$ are proper semisets from the point of view of $M$.

\medskip
\hskip-\parindent{\bf II.} Another context, where semisets arise, are nonstandard models of ZFC. For example consider the ultrapower $V^\omega/{\mathcal U}$
where ${\mathcal U}$ is a nontrivial ultrafilter on $\omega$. In this structure there are proper semisets which are even parts of a natural number:
the class of standard natural numbers is an example of such a semiset. It has to be proper, since it has no maximal element. Conversely there
are parts of the natural numbers with no minimal elements, so, again, they must be proper semisets.



% \subsubsubject{A last comment}
% The method of forcing invented by P. Cohen is a method which, given suitable proper semisets, allows to extend the universe of sets into a
% new universe where these semisets become regular sets. Moreover the method is flexible in the sense that it allows to control the behavior
% of the new sets. This extension is called the \emph{generic} extension and the method involves the usage of names.

% \subsection{${}$ \hspace{-1em}Questions\ \&\ Answers.}\label{QaA}
%
% ${}$  \medskip

% {\scshape Question.} Why separation is an axiom schema instead of an axiom?
%
% \smallskip
%
% {\scshape Answer.} For each formula $\varphi$, the schema introduces a single axiom. It follows that there are really infinitely many axioms.
% It can be shown, that the theory $ZF$ cannot be finitely acclimatized, i.e. no finite list of axioms is equivalent to $ZF$. However, the collection of axioms is still rather simple, formally speaking, it is recursive. This is important in logic.
%
% \medskip

% {\scshape Question.} What are classes?
%
% \smallskip
%
% {\scshape Answer.} Every formula $\varphi$ of Set Theory naturally determines the collection of all sets satisfying $\varphi$. This collection
% is usually denoted by
% $$ %\startformula
% \{x:\varphi(x)\},
% $$ %\stopformula
% and called a class. We can extend most of the usual set operation to these objects even though, in many cases, e.g. $V=\{x:x=x\}$, $On=\{\alpha:\alpha\ \mbox{is an ordinal}\}$, $E=\{\langle x,y\rangle:x\in y\}$, these objects are {\bf not} sets. Separation axiom implies that the intersection of a class with a set is always a set. Using classes, we can e.g. write $(H(\kappa),E\cap H(\kappa)^2)$ instead of $(H(\kappa),\in)$.
%
% \medskip

% {\scshape Question.} What is the weird symbol $\models$?
%
% \smallskip
%
% {\scshape Answer.} Short answer: satisfaction, a relation between
% relation structures and formalised formulas, well-known from Model Theory.
%
% \medskip
%
% {\scshape Question.} What about the countable model of ZFC?
%
% \smallskip




% \bigskip
%
% For an in-depth discussion of these questions see your local logic guru.

%%%%%%%%%%%%%%%%%%%%%%%%%%%%%%%%%%%%%%%%%%%%%%%%%%%%%%%%%%%%%%%%%%%%%%%
%%%                          END                                    %%%
%%%%%%%%%%%%%%%%%%%%%%%%%%%%%%%%%%%%%%%%%%%%%%%%%%%%%%%%%%%%%%%%%%%%%%%