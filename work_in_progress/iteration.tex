\subsection{iteration}

% \subsubsection{Martin's axiom}
%
% Take $\kappa$ s.t., $\kappa^{<\kappa}=\kappa$. Consider $\mathcal S=\{(P,\leq):P\subseteq \kappa, |P|<\kappa\}$ ordered by regular embedding.
% If $B$ is a generic on $S$ then forcing with $B$ gives us MA.


\subsubsection{Preservation theorems}
\begin{definition} An ordering $P$ is \emph{$n$-linked}, $n<\omega$, if for any $X\in[P]^{\leq n}$ there is
$p\in P$ compatible with each $x\in X$. \emph{Linked} is short for $2$-linked. An ordering has the \emph{Knaster} property if any uncountable subset of the ordering contains an uncountable linked subset.
\end{definition}

\begin{obs} If $P$ has the Knaster property then given $n<\omega$ any uncountable subset contains
 an uncountable $n$-linked subset.
\end{obs}
\begin{proof} Easy, inductively apply the Knaster property to the witnesses for $2$-linkedness.
\end{proof}

\begin{proposition} If $P$ is an ordering with the Knaster property and $P\force\dot{Q}$ has the Knaster
 property then $P*\dot{Q}$ has the Knaster property.
\end{proposition}
\begin{proof}
 Let $\{\langle p_\alpha,\dot{q}_\alpha\rangle:\alpha<\omega_1\}\subseteq P*\dot{Q}$.
 By strengthening the $p_\alpha$'s we may assume that each $p_\alpha$ decides the statement

 \begin{displaymath}
  \{\dot{q}_\alpha:\alpha< \omega_1\}\ \hbox{is uncountable}
 \end{displaymath}

 By the above observation we may assume that the $p_\alpha$'s are $4$-linked. We also assume that all the
 $p_\alpha$'s decide the statement in the same way. There are two cases:

 {\bf Case 1.} Each $p_\alpha$ forces that $\{\dot{q}_\alpha:\alpha<\omega_1\}$ is uncountable. Working
 in $RO(P)$ (which also has the Knaster property) let $b=\bigvee\{p_\alpha\}$. Then $b$ forces that
 $\{\dot{q}_\alpha:\alpha<\omega_1\}$ is uncountable and since $P\force\dot{Q}$ has the Knaster property,
 there is a name $\dot{X}$ such that
 \begin{displaymath}
  b\force \dot{X}\subseteq\{\dot{q}_\alpha:\alpha<\omega_1\}\ \&\ |\dot{X}|=\omega_1\ \&\ \dot{X}\
  \mbox{is linked}.
 \end{displaymath}
 By induction to $\omega_1$ choose
 $\{\langle r_\alpha,\dot{s}_\alpha\rangle :\alpha<\omega_1\}\subseteq
  \{\langle p_\beta,\dot{q}_\beta\rangle:\beta<\omega\}$
 such that $r_\alpha\force \dot{s}_{\alpha+1}\in\dot{X}$. This is possible since each $p_\alpha$
 forces $\dot{X}$ to be uncountable. We claim that
 $\{\langle r_{\alpha+1},\dot{s}_{\alpha+1}\rangle:\alpha<\omega_1\}$
 is linked. Choose $\alpha,\beta<\omega_1$. Since we assumed the $p_\alpha$'s are $4$-linked we may find
 $p\in P$ be below $r_\alpha,r_{\alpha+1},r_\beta,r_{\beta+1}$.
 Then $p\force \dot{s}_{\alpha+1},\dot{s}_{\beta+1}\in \dot{X}$ so $p$ knows they are compatible so there
 is $\dot{t}\in \dot{Q}$ such that $p\force \dot{t}\leq \dot{s}_{\alpha+1},\dot{s}_{\beta+1}$. Then
 $\langle p,\dot{t}\rangle\leq\langle r_{\alpha+1},\dot{s}_{\alpha+1}\rangle,\langle r_{\beta+1},dot{s}_{\beta+1}\rangle$ and this finishes the proof of case 1.

 {\bf Case 2.} There is $R=\{\dot{r}_n:n<\omega\}\subseteq\dot{Q}$ such that
 $\dot{p}_\alpha\force\dot{q}_\alpha\in R$. Again work in $RO(P)$ and find, below each $p_\alpha$
 an antichain $b_{\alpha,n}$ such that $b_{\alpha,n}=||\dot{q}_\alpha=\dot{r}_n||$.
 By the Knaster property for $RO(P)$ there is an uncountable set $I\subseteq\omega_1\times\omega$
 such that for each $(\alpha,n),(\beta,m)\in I$ the corresponding $b_{\alpha,n}$ and $b_{\beta,m}$
 are compatible. Since $I$ is uncountable we may assume that the second coordinates are all equal to
 some $n_0<\omega$ and then $b_{\alpha,n_0}\leq p_\alpha$ and $b_{\alpha,n_0}\force \dot{q}_\alpha=\dot{r}_n$
 so they witness that $\langle p_\alpha,\dot{q}_\alpha\rangle$ are pairwise compatible.

\end{proof}

