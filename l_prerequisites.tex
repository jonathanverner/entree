\documentclass[a4paper,11pt,oneside]{mybook}
\usepackage{latexsym}
\usepackage{amsmath}
\usepackage{amsfonts}
\usepackage{amsthm}
\usepackage{amssymb}
\usepackage{fncylab}

%fonty
\usepackage{charter}
\usepackage{euler}

% tiskove zrcadlo
\topmargin-40pt
\headheight15pt
\headsep10pt
\topskip0pt
\textheight700pt
\oddsidemargin-20pt
\evensidemargin-20pt
\textwidth500pt

\usepackage{graphicx}

%headings
\usepackage{fancyhdr}
\pagestyle{myheadings}
\pagestyle{fancy}

%----------definitions---------------------------
\def\R{\mathbb R}
\def\cl{\text{cl}}
\def\norma#1{\parallel\!#1\!\parallel}
\def\norm#1{\parallel\!#1\!\parallel}

%math definitions
\def\sub{\text{Sub}}
\def\meas{\text{Meas}}
\def\msub{\text{mSub}}
\def\cov{\text{cov}}
\def\cal{\text{cal}}
\def\msubn{\text{mSub}_{\text{1}}}
\def\subn{\text{Sub}_{\text{1}}}
\def\fn{\text{Fn}}

\def\B{{\mathbb B}}
\def\C{{\mathbb C}}

\def\F{{{\mathcal F}}}
\def\N{{{\mathcal N}}}

\def\cont{{2^{\aleph_0}}}
\def\force{\Vdash}
\def\aleq{\leq^{*}}
\def\succ{\hbox{succ}}
\def\pw{{{\mathcal P}}}
\def\conc{{^\smallfrown}}

\def\dom{\hbox{dom}}
\def\pomega{\pw(\omega)}
\def\0{\hbox{\bf 0}}
\def\1{\hbox{\bf 1}}
\def\intr#1{%
  \hbox{int}\ #1%
}
\def\cl#1{%
  \hbox{cl}\ #1%
}



%---------numbering of the formulas and theorems---------------
% \newcounter{thm}[section]
% \numberwithin{thm}{section}

%---------numbering of the theorems------------
\swapnumbers

%\theoremstyle{plain}
\newtheoremstyle{theorem}% name
  {}%      Space above
  {}%      Space below
  {\itshape}%         Body font
  {}%         Indent amount (empty = no indent, \parindent = para indent)
  {\bfseries\scshape}% Thm head font
  {.}%        Punctuation after thm head
  {.5em}%     Space after thm head: " " = normal interword space;
        %       \newline = linebreak
  {}%         Thm head spec (can be left empty, meaning `normal')
\theoremstyle{theorem}
\newtheorem*{theorem*}{Theorem}
\newtheorem{theorem}[subsection]{Theorem}
\newtheorem{lemma}[subsection]{Lemma}
\newtheorem{proposition}[subsection]{Proposition}
\newtheorem{fact}[subsection]{Fact}
\newtheorem{claim}[subsection]{Claim}
\newtheorem{corollary}[subsection]{Corollary}
\newtheorem{problem}[subsection]{Problem}
%\numberwithin{equation}{section}
%\theoremstyle{definition}
\newtheorem{definition}[subsection]{Definition}
\newtheorem{notation}[subsection]{Notation}
%\theoremstyle{remark}
%\newtheorem{remark}[subsection]{Remark}
%\newtheorem*{note}{Note}
\newtheoremstyle{example}% name
  {}%      Space above
  {}%      Space below
  {}%         Body font
  {}%         Indent amount (empty = no indent, \parindent = para indent)
  {\bfseries\scshape}% Thm head font
  {.}%        Punctuation after thm head
  {.5em}%     Space after thm head: " " = normal interword space;
        %       \newline = linebreak
  {}%         Thm head spec (can be left empty, meaning `normal')
\theoremstyle{example}
\newtheorem{example}[subsection]{Example}
\newtheorem{remark}[subsection]{Remark}
\newtheorem*{note}{Note}

%\labelformat{subsection}{\thechapter.#1}

%vytvoreni indexu
\usepackage{makeidx}
\makeindex

\includeonly{prerequisitex.tex,orderings.tex,forcing.tex,generic.tex}

%-------------opening--------------------------
\begin{document}
\cfoot{}\rhead{\thepage}
\lhead{{\scshape Prerequisites} $\qquad$ {\tiny \today } }

\noindent{\Large{\scshape\bfseries Entreé to Generic Extensions and Forcing in Set Theory}} \\[0.1cm]

\noindent {\scshape Bohuslav Balcar}, {\small CTS, J{\' \i}lsk{\' a} 1, Praha 1,
	Czech Republic, {\ttfamily balcar@cts.cuni.cz} } \\[0.1cm]
\noindent {\scshape Jonathan Verner}, {\small KTIML MFF UK, {\ttfamily jonathan.verner@matfyz.cz}\\[0.1cm]
{\tiny \today } \\[0.5cm]

%\maketitle

\thispagestyle{empty}


\section{Set Theory prerequisites}

When we talk about Set Theory we mean the first order predicate theory in a language with equality containing a binary predicate $\in$
with the standard axioms. Formulas describing properties of sets are built up from symbol for variables, predicates ($\in,=$),
logical connectives and quantifiers ($\forall,\exists$). If $\phi(x_0,\ldots,x_n)$ is such a formula, then $\{\langle x_0,\ldots, x_n\rangle:
\phi(x_0,\ldots,x_n)\}$ is the class defined by this formula. E.g. $V=\{x:x=x\}$ is the class of all sets.

\subsection{Axioms of Zermelo-Fraenkel}


\begin{itemize}
\item[1.]\emph{Extensionality.} If $X$ and $Y$ have the same elements, then
$X=Y$.
\item[2.]\emph{Pairing.} For any $a$ and $b$ there exists a set $\{a, b\}$ that
contains exactly $a$ and $b$.
\item[3.]\emph{Separation.} If $\varphi$ is a formula (with parameter $p$),
then for any set $X$ and $p$ there exists a set $Y = \{u \in X : \varphi(u, p)\}$ that contains
all those $u \in X$ that satisfy $\varphi$.
\item[4.]\emph{Union.} For any $X$ there exists a set $Y =\bigcup X$, the \emph{union}
of all elements of $X$.
\item[5.]\emph{Power Set.} For any $X$ there exists a set $Y = \pw (X)$, the
set of all subsets of $X$.
\item[6.]\emph{Infinity.} There exists an infinite set.
\item[7.]\emph{Replacement.} If a class $F$ is a function, then for
any set $X$ there exists a set $Y = F[X] = \{F (x) : x \in X\}$.
\item[8.]\emph{Foundation.} Every nonempty set has an $\in$-minimal element.
\item[9.]\emph{Choice.} Every family of nonempty sets has a choice function.
\end{itemize}

We now give some comments: Once we have any set, say $a$, using the third axiom we can prove
that the empty class is a set: $\{x\in a:x\neq x\}$. The schema of Replacement says, roughly,
that 'small' is mapped onto 'small'.

\smallskip

It is customary to let ZF denote Set Theory with axioms 1.--8. and ZFC Set Theory ZF with choice
(i.e. with axiom 9.). Occasionally we shall write $ZF^-$ and $ZFC^-$ when we ommit axiom 5.

\begin{definition}\label{ts}
{\bf Transitive set.} A set $A$ is \emph{transitive} if every element of $A$
 is a subset of $A$ (i.e. $(\forall x\in A)(x\subseteq A)$).
\end{definition}

\begin{fact} Every set $A$ has a \emph{transitive closure}, that is, the smallest transitive
 set containing $A$ as an element. It is denoted by $Trans(A)$.
\end{fact}
\begin{proof} (hint) Use recursion and let $Trans(A)=\{A\}\cup A\cup \bigcup A\cup \bigcup\bigcup A\cdots$.
\end{proof}

\begin{definition}\label{ordnumber}{\bf Ordinal number.} A set $\alpha$ is an ordinal number if
 \begin{itemize}
  \item[(i)]   $\alpha$ is transitive,
  %\item{(ii)}  every element of $\alpha$ is transitive
  \item[(ii)] the $\in$ relation is a linear ordering on $\alpha$, i.e. for all $\beta,\gamma\in\alpha$
               either $\beta\in\gamma$ or $\gamma\in\beta$ or $\beta=\gamma$.
 \end{itemize}
 In other words an ordinal number is the set of all smaller ordinal numbers wellordered by $\in$.
 The class of all ordinal numbers is denoted by $On$. We write $\alpha<\beta$ to mean $\alpha\in\beta$.
\end{definition}

The above definition is the standard one and is originally due to von Neumann.

\subsection{Approximation of the universe}

The universal class $V=\{x:x=x\}$ satisfies all axioms of Set Theory, but it is not a set. However there are transitive sets,
which satisfy a large fragment of the axioms, although not necessarily all of them. The ones most commonly used are
the two hierarchies of $V_\alpha$'s and $H(\kappa)$'s.

\begin{definition}{\bf The hierarchy of $V_\alpha$'s.} Using transitive recursion to iterate the power set operation we define $V_\alpha$
for $\alpha\in On$ as follows:
$$ %\startformula
V_{\emptyset}=\emptyset,\quad V_{\alpha+1}=\pw(V_\alpha),\quad V_\alpha = \bigcup_{\beta<\alpha} V_\beta,\ \mbox{for limit}\ \alpha.
$$ %\stopformula
\end{definition}
\begin{fact} $V=\bigcup\{V_\alpha:\alpha\in On\}$. \\
Note that this fact is equivalent to the axiom of foundation (axiom 8.).
\end{fact}

\begin{definition}\label{rank}{\bf Rank.} To each set $x$ we assign its \emph{rank} $\tau(x)$:
 $$ %\startformula
  \tau(x)=min\{\alpha:x\subseteq V_\alpha\} \in On.
 $$ %\stopformula
  It is easily seen that $\tau(x)=sup\{\tau(y)+1:y\in x\}$ for any set $x$.
\end{definition}

\begin{fact}\label{valphaFact} Suppose $\alpha<\beta$ are ordinal numbers. Then
 \begin{itemize}
  \item[(i)] $V_\alpha$ is a transitive set.
  \item[(ii)] $V_\alpha\in V_\beta$.
  \item[(iii)] $On\cap V_\alpha = \alpha$, hence $\alpha\subseteq V_\alpha$ but $\alpha\not\in V_\alpha$.
  \item[(iv)] $V_\omega$ consists of hereditarily finite sets and $(V_\omega,\in)$ is a model of $\hbox{ZF}_{fin}$.
              It turns out that it is 'equivalent' to the standard model of Peano Arithmetic.
 \end{itemize}
\end{fact}

\begin{proposition}\label{valphaApprox} If $\alpha>\omega$ is a limit ordinal number, then $(V_\alpha,\in)$ satisfies all of the axioms
 of set theory except for the (full) replacement schema. The class of ordinal numbers of $(V_\alpha,\in)$ is
 $\alpha=V_\alpha\cap On$.
\end{proposition}


\begin{definition}\label{H-kappa}{\bf The $H(\kappa)$ hierarchy.} Assume AC. If $\kappa$ is an infinite cardinal number we define
$H(\kappa)$ as follows:
$$ %\startformula
 H(\kappa) = \{x:|Trans(x)|<\kappa\}.
$$ %\stopformula
The set $H(\kappa)$ consists of sets which are hereditarily of cardinality $<\kappa$. The universe $H(\omega_1)$ of hereditarily countable
sets is sometimes denoted by $HC$.
\end{definition}

In ZFC we again have the following:

\begin{fact} $V=\bigcup\{H(\kappa):\kappa\ \mbox{is a cardinal number}\}$.
\end{fact}

\begin{proposition}{\rm [AC]}
 \begin{itemize}
  \item[(i)] Each $H(\kappa)$ is a transitive set.
  \item[(ii)] $H(\kappa)=V_\kappa$ iff $\kappa=\omega$ or $\kappa$ is inaccessible.
 \end{itemize}
\end{proposition}

Recall, that a cardinal number $\kappa$ is inaccessible if $\kappa>\omega$ is regular and for any cardinal $\lambda<\kappa$
$2^\lambda<\kappa$ (where $2^\lambda$ is the cardinal exponentiation, i.e. $2^\lambda=|\pw(\lambda)|$).

\begin{proposition}{\rm [AC]}
\begin{itemize}
 \item[(i)] If $\kappa>\omega$ is a regular cardinal then $(H(\kappa),\in)$ satisfies all of the axioms of set theory
 except for the power set axiom 5., i.e. it is a model of $ZFC^-$. (cf. Proposition \ref{valphaApprox}.)
 \item[(ii)] The role of the class of ordinal numbers in the structure $(H(\kappa),\in)$ is played by $\kappa$. %(cf. \in{fact}[valphaFact],iii.)
 \item[(iii)] If a set $A$ is in $H(\kappa)$ then all of its subsets are also members of $H(\kappa)$
 \item[(iv)] The powerset axiom is satisfied in $H(\kappa)$ for a set $A\in H(\kappa)$ iff $\pw(A)\in H(\kappa)$. Also note that
         $\pw(A)\in H(\kappa)$ iff $2^{|A|}<\kappa$.
\end{itemize}
\end{proposition}

\subsection{Well-founded relations}

\begin{definition}\label{ext}
A binary relation $R$ on a class $A$ is \emph{well-founded}
iff each nonempty subset of $A$ contains at least one
 $R$-minimal element, that is for any $\emptyset\neq a\subseteq A$ there is an $x\in a$ with
 $$ %\startformula
  Ext_R(x) \cap a = \emptyset,\quad\hbox{where}\quad Ext_R(x)=\{ y:\langle y,x\rangle\in R\}.
 $$ %\stopformula
\end{definition}

Axiom of foundation 8. says, that the membership relation $\in$ is well-founded on the universal class $V$.

\begin{definition}\label{DC}
{\bf Dependent Choice} The axiom of dependent choice is a weaker
version of the axiom of choice. It says, that
whenever some relation $R$ satisfies
$$ %\startformula
  (\forall x\in dom(R))(\exists y)(yRx)
$$ %\stopformula
then there is an $R$-descending sequence $\langle x_n:n<\omega\rangle$,
i.e. a sequence satisfying $x_{n+1}R x_n$ for each $n<\omega$.
The axiom of dependent choice is abbreviated DC.
\end{definition}

\begin{fact} If we assume DC then a relation is well-founded iff there are no infinite $R$-decreasing chains.
\end{fact}


\begin{definition} A relation $R$ is \emph{set-like} iff for every set $x$ the class $Ext_R(x)$ is a set.
\end{definition}

Well-founded set-like relations are generalizations of the $\in$ relation. The theorems on transfinite induction and recursion can be
generalized in this settings to induction and recursion over set-like well-founded relations.

\begin{example}
\begin{itemize}
 \item[(i)]  The lexicographical ordering $<_1$ on $On\times On$ is an example of a relation which is well-founded but is not set-like.
 \item[(ii)] The maximo-lexicographical ordering $<_2$ on $On\times On$ is a well-founded set-like relation:
%\startformula
%\startalign[n=3,align={left,middle,left}]
\begin{align}
{} & \langle \alpha_1,\beta_1\rangle \leq_2 \langle \alpha_2,\beta_2\rangle & \equiv& \max(\alpha_1,\beta_1) < \max(\alpha_2,\beta_2)\ \hbox{or}\ \\
                                                                          &&&(\max(\alpha_1,\beta_1) = \max(\alpha_2,\beta_2)\ \&\ %\\ &&&
                                                                          \min(\alpha_1,\beta_1)< \min(\alpha_2,\beta_2) )
\end{align}
 %\stopformula

\end{itemize}
\end{example}

\subsubsection{Well-founded induction and recursion}


\begin{theorem}\label{WF-induction}
{\rm (Well-founded induction).}
Suppose $R$ is a well-founded and set-like relation on $A$.
Furthermore suppose $\phi$ is a formula such that whenever for each $a\in A$ if
$(\forall b\in Ext_R(a))\varphi(b) \rightarrow \varphi(a)$. Then $(\forall a\in A)\varphi(a)$.
\end{theorem}

\begin{theorem}\label{WF-recursion}
{\rm (Well-founded recursion).} Suppose $R$ is a well-founded and set-like
relation on $A$ and $G$ is a (class) function defined on $V$.
Then there exists a unique function $F$ defined on $X$ which satisfies
$$ %\startformula
 F(a) = G(F\upharpoonright Ext_R(a) )\quad\hbox{for each}\ a\in A.
$$ %\stopformula
\end{theorem}

\begin{corollary}
It follows, as a special case, that there exists a unique function $F_1$ such that
 $$ %\startformula
 F_1(a) = G(\{F_1(y):b R a \}).
$$ %\stopformula
\end{corollary}

\subsubsection{Rank function}
Every set-like well-founded relation $R$ on $X$ has an associated \emph{rank function} $\rho_R:X\to On$ which
is defined using well-founded recursion as:
$$ %\startformula
 \rho_R(x)=\hbox{sup}\{\rho_R(y)+1:y R x\}.
$$ %\stopformula

\smallskip

If $X$ is a set, then the range of $\rho_R$ is an ordinal number and is called the \emph{height} of $R$ on $X$.

\smallskip

If $R$ is a wellordering, then the height of $R$ is also called the \emph{order type} of $R$.

\smallskip

We shall now introduce the often used concept of a tree and show an application of the rank function on illfounded trees.


\begin{definition}\label{tree}
{\bf Tree.} A set $T$ together with an ordering $\leq$ forms a \emph{tree} iff
for each $t\in T$ the set of its predecessors $pred_T(t)=\{s\in T:s\leq t\}$ is well-ordered by $\leq$.

\smallskip

For each $t\in T$ we define the \emph{height} $h_T(t)$ of $t$ as the order type of
$(pred_T(t),\leq)$. The $\alpha$-th level of $T$, denoted by $T_\alpha$,  is defined as $T_\alpha=\{t\in T:h_T(t)=\alpha\}$. The height of $T$ is defined to be
the rank of $\leq$ and is equal to $sup\{\alpha+1:T_\alpha\neq\emptyset\}$. A maximal linearly ordered subset of $T$ is called a \emph{branch} of $T$.

\smallskip

Beware that a tree is called \emph{well-founded} if the reverse order $>$ is well-founded, otherwise it is \emph{ill-founded}.

\smallskip

Trees of height $\omega$ have a nice representation. For a set $X$ define $Seq(X)=\bigcup\{{}^nX:n<\omega\}$, the set of all finite sequences of elements of $X$. In case $X=\omega$ we write just $Seq$. A $T\subseteq Seq(X)$ closed under taking initial segments, i.e. if $f\in T$ and $n\in dom f$ then $f\upharpoonright n\in T$, is called a tree on $X$. If $X=\omega$, we say $T$ is an \emph{$\omega$-ary tree}, if $X=2$, $T$ is \emph{binary tree}. If $T$ is a tree on $X$ then  $(T,\subseteq)$ is a tree.
\end{definition}


\begin{fact}{\rm [DC]}
A tree is ill-founded iff it has an infinite branch.
\end{fact}


\begin{example} If $T$ is an illfounded $\omega$-ary tree, then the reverse
inclusion relation on $T$ is
 well founded, so it has an associated rank function $\rho_T$.
We can then define the rank $r(T)$
 of $T$ to be $\rho_T(\emptyset)$.
\end{example}

\begin{fact}
If $T$ is an illfounded $\omega$-ary tree, then $r(T)<\omega_1$ and for any $\alpha<\omega_1$
 there is an illfounded $\omega$-ary tree $T^\alpha$ with $r(T^\alpha)=\alpha$.
\end{fact}

\begin{proof}
To see that $r(T)<\omega_1$ note that $|T|$ is countable. On the other hand we proceed by induction.
A tree of rank $1$ is just $\{\emptyset\}$. Suppose $\alpha<\omega_1$ and we have constructed a tree $T^\beta$ of rank $\beta$ for each $\beta<\alpha$.
To construct a tree of rank $\alpha$, take a tree $T$ of height $2$ whith $|T_1|=\omega$ and glue the $T^\beta$'s to the branches of $T$ to
get $T^\alpha$.
%
% $we have a tree $T$
% A tree of finite rank is easy to build. A tree of rank $\omega$ is just an infinite union of branches of increasing length.
%
% if $\alpha<\omega_1$, choose for each limit $0<\beta\leq\alpha$ a strictly increasing sequence $\beta_n$
% with $\beta=\bigcup\{\beta_n:n<\omega\}$. We define by induction on $\alpha$ a function $f$ on the full $\omega$-ary tree
% ${}^{<\omega}\omega$ as follows: $f(\emptyset)=\alpha$. If $f(t)=\beta+1$ then $f(t\conc n)=\beta$ for each $n<\omega$.
% If $f(t)=\beta$ and $\beta$ is limit and nonzero, then $f(t\conc n)=\beta_n$. If $f(t)=0$ then $f(t\conc n)=0$.
% Finally we let $T_\alpha=\{t\in{}^{<\omega}\omega:f(t)>0\vee f(t)=0\ \&\ (\forall s\subset t)(f(s)\neq 0)\}$. It
% is easily seen that $r(T_\alpha)=\alpha$ and that $f\upharpoonright T_\alpha$ is the canonical rank function $\rho_{T_\alpha}$.
\end{proof}

We now turn to an important theorem that will be useful later on, but first we need a definition.

\begin{definition}
A relation $R$ on a class $A$ is extensional if for $a\neq b\in A$ we have that $Ext_R(a)\neq Ext_R(b)$ (see Definition \ref{ext} for the
meaning of $Ext_R(a)$.) Note that the axiom of extensionality is equivalent to saying that $\in$ is extensional on $V$.
\end{definition}

\begin{theorem}\label{mostowski}
{\rm (Mostowski collapse).} Suppose $R$ is a wellfounded, set-like relation on a class $A$.
\begin{itemize}
\item[(i)] There is a transitive class $M$ and a surjective mapping $F:A\to M$ such that $aRm\rightarrow F(a)\in F(m)$.
\item[(ii)] Moreover if $R$ is extensional on $N$, then the mapping is an isomorphism.
\end{itemize}
\end{theorem}

\begin{proof}
(hint) Using well-founded recursion, define $F(a)=\{F(b):b\in A\ \&\ bRa\}$. To show that $F$ is as required, use well-founded induction.
\end{proof}

% \startlemma Suppose that $M,N$ are transitive models of ZF and one of them satisfies the Axiom of choice.
%  Suppose, moreover, that they have the same subsets of ordinal numbers, i.e. $\pw^M(Ord^M)=\pw^N(Ord^N)$.
%  Then $M=N$.
% \stoplemma
% \begin{proof} Without loss of generality assume $M\models AC$. First we show that $M\subseteq N$. Let $X\in M$ and, since $M\models AC$
% there is bijection $f\in M$ of some ordinal $\delta\in M$ onto the transitive closure of $X$.
% Define a relation $E$ on $\delta$ as follows: $\alpha E\beta\equiv f(\alpha)\in f(\beta)$. Then, by our
% assumption, $E\in N$. Also $E$ is extensional and well-founded in $M$ and, by absoluteness, also in $N$. Applying
% Mostowski collapsing theorem, there is a transitive set $T\in N$ such that $(E,\delta)\equiv (T,\in)$. So also
% $(T,\in)\equiv (Tcl(X),\in)$. But since both $T$ and $Tcl(X)$ are transitive, necessarily $T=Tcl(X)$. So
% $X\in N$.
%
% Now we show $M=N$ by transfinite induction. Assume $X\in N$ and $X\subseteq M$. Find $\alpha\in Ord$ such
% that $X\subseteq V_\alpha^M$ and let $f\in M$ be a bijection from $V_\alpha^M$ into $Ord$. Now since
% $f[X]$ is a set of ordinals and hence $f[X]\in M$ we know that $X=f^{-1}[f[X]]\in M$.
% \end{proof}


\subsubsection{Elementary substructures}

We will focus on countable substructures of the uncountable structures $(H(\kappa),\in)$ and universal algebras
$A=\langle A,\{f_i:i\in I\}\rangle$. We shall tacitly assume that the set of operations is at most
countable and each of them is finitary. Suppose we have such an algebra $A$. A subset $X\subseteq A$ closed under the
operations determines a subalgebra and whenever $Y\subseteq A$ is infinite there is an $X\subseteq A$ which is closed under the operations,
contains $Y$ and $|Y|=|X|$. We assume the axiom of choice throughout this section.

\begin{definition}\label{club}
{\bf Club.} Let $A$ be an infinite set. A family $C\subseteq[A]^{\omega}$ is \emph{closed unbounded}, or \emph{club}
for short, in $[A]^\omega$ if it is
\begin{itemize}
  \item[(i)] \emph{unbounded,} i.e.
	$(\forall Y\in[A]^\omega)(\exists X\in C)(Y\subseteq X)$ and
  \item[(ii)] \emph{closed,} i.e.
	For any increasing chain $\{X_n:n<\omega\}$ of elements of
	$C$ the union $\bigcup\{X_n:n<\omega\}$ is again an element of $C$.
\end{itemize}
\end{definition}

\begin{note} In our definition we have only assumed that $A$ is infinite. If it is countable, however, the notion is not very interesting, in particular
the singleton $\{A\}$ is a club.
\end{note}

\begin{note} Readers that are familiar with the notion of a clubs on a cardinals may observe that if $C$ is a club on $\omega_1$ then
it is also a club in $[\omega_1]^{\omega}$ (more precisely $\{\alpha:\alpha\in C\ \&\ \omega\leq\alpha\}$ is a club in $[\omega_1]^{\omega}$).
\end{note}

\begin{fact}
Suppose $A$ is infinite. It is relatively easy to prove that the intersection of a countable system of clubs in $[A]^{\omega}$ is again
a club in $[A]^{\omega}$. It follows that the system of clubs in $[A]^{\omega}$ generates a $\sigma$-complete filter.
\end{fact}
%
% \begin{fact}{} It is relatively easy to prove the following basic properties. Suppose $A$ is infinite. Then
% \begin{itemize}[i]
% \item[()] The intersection of a countable system of clubs in $[A]^{\omega}$ is a club in $[A]^{\omega}$.
% \item[()] The system of clubs in $[A]^{\omega}$ generates a filter.
% \end{itemize}
% \end{fact}

The following proposition is an algebraic version of the well known L\"owenheim,Skolem theorem from model theory.

 \begin{proposition}\label{algebraic-club-A}
Suppose $A=\langle A,\{f_n:n<\omega\}\rangle$ is an infinite algebra. Then the family of all
 countable subalgebras forms a club in $[A]^{\omega}$.
 \end{proposition}

The converse of the proposition also holds (see e.g. \cite{jech:stt}):

 \begin{proposition}\label{algebraic-club-B}
Suppose that $C$ is a club in $[A]^{\omega}$ then there are operations $\{f_n:n\in \omega\}$ in $A$ such that
 $C$ contains all countable subalgebras of $A$ with the given operations.
 \end{proposition}


Having defined closed unbounded families we can define stationary families

\begin{definition}\label{stationary}
{Stationary set} A family $S\subseteq [A]^{\omega}$ is \emph{stationary} if it intersects each club,
i.e. if $C$ is a club in $[A]^{\omega}$ then $S\cap C\neq\emptyset$.
\end{definition}

It follows from propositions \ref{algebraic-club-A}, \ref{algebraic-club-B} that

\begin{corollary} A set $S\subseteq [A]^\omega$ is stationary iff for every system $\{f_n:n<\omega\}$ of operations on $A$
$S$ contains a set which is closed under these operations
\end{corollary}

Consider now the structure $(H(\kappa),\in)$.

\begin{definition}\label{elementary-substructure}
{\bf Elementary substructure.}
Let $X\subseteq H(\kappa)$ be a countable set. We shall say that
$X$ is an \emph{elementary substructure} of $H(\kappa)$, formally $(X,\in)\preceq (H(\kappa),\in)$, if for each property expressed
by a formula of Set Theory $\varphi(v_1,\ldots,v_n)$ with free variables $v_1,\ldots, v_n$ the following holds:
%\placeformula
$$ %\startformula
(\forall x_1,\ldots,x_n\in X)( (X,\in)\models \varphi[x_1,\ldots,x_n] \equiv (H(\kappa),\in)\models \varphi[x_1,\ldots,x_n] )
$$ %\stopformula
This implies, in particular, that sentences (i.e. closed formulas) of Set Theory are valid in $(X,\in)$ iff they are valid in $(H(\kappa),\in)$.
\end{definition}

{\scshape Relativity principle.} The concept of an elementary substructure is akin to the Gallileo's relativity principle in physics. This principle states that, using physical
experiments, we cannot decide which frame of reference we are in. Similarly, using Set Theoretical properties, we cannot distinguish
whether we are in the elementary substructure or the superstructure. In other words, the validity of statements about elements of $X$
cannot distinguish between $X$ and $H(\kappa)$.


% \begin{note} Note that when deciding the validity of formulas in $X$ the quantifiers range only over $X$ whereas in $H(\kappa)$ they range
% over all of $\kappa$. In particular existential formulas are 'easier to satisfy' in $H(\kappa)$ whereas universal formulas are ``easier to
% satisfy'' in $X$.
% \end{note}

\begin{fact}
For any $X\preceq H(\kappa)$ the following basic facts are true:
\begin{itemize}
  \item[(i)] $\omega\subseteq X$ since each natural number is definable in $H(\kappa)$ and
  \item[(ii)] $\omega\in X$, since $\omega$ is also definable in $H(\kappa)$,
\end{itemize}
more generally
\begin{itemize}
  \item[(iii)] If $Y\in X$ is countable, then $Y\subseteq X$.
\end{itemize}
\end{fact}

\begin{note}
Countable elementary substructures typically are not transitive sets, and for $\kappa>\omega_1$ are provably not transitive.
             It is hence perfectly possible that some uncountable set is an \emph{element} of the countable substructure. In fact,
             $\omega_1\in X$ whenever $X$ is an elementary substructure of $H(\kappa)$ for $\kappa>\omega_1$.
\end{note}

\begin{theorem}
As with algebras, we have the following often used and important theorem
\begin{itemize}
  \item[(i)] For any countable subset $Y\subseteq H(\kappa)$ there is an
	$X\preceq H(\kappa)$ such that $Y\subseteq X$.
  \item[(ii)] The set of countable elementary substructures of $H(\kappa)$
	forms a club in $[H(\kappa)]^\omega$.
\end{itemize}
\end{theorem}

Stationary sets on $[A]^\omega$ are important in the theory of proper forcing. For more information on elementary substructures
and their usage e.g. in topology see \cite{dow:mstt}.

\subsubsection{Semisets}
We now introduce a methodological concept which will be used later on. Reader beware: this concept is not generally known in the
Set-Theory circles, but see \cite{vopenka:semisets} for a more in-depth discussion of the concept.

\begin{definition}\label{semiset}
{\bf Semiset.} A \emph{semiset} is a class which is a part of a set, formally $Sm(X)\equiv(\exists a)(X\subseteq a)$.
It follows that each set is also a semiset. A semiset is \emph{proper} if it is not a set.
\end{definition}

In the standard set theory there are no proper semisets so we will now describe two natural situations which lead this notion.


\subsubsection{I}


Take an infinite cardinal $\lambda$ and consider $H(\kappa)$ where $\kappa= (2^\lambda)^+$. We know that
$(H(\kappa),\in)$ is a model of ZFC${}^-$ (see Proposition \ref{valphaApprox}, (i)). In our special case
$\pw(\lambda)\in H(\kappa)$ so the power set axiom holds for $\lambda$. Let $X$ be a countable elementary substructure of $H(\kappa)$
with $\lambda\in X$.

Then $(X,\in)$ is a model for ZFC${}^-$, $\lambda,\pw(\lambda)\in X$ and $\in$ is a well-founded extensional relation
on $X$. Applying the Mostowski collapse (see Theorem \ref{mostowski}) to obtain a transitive set $M$ and an isomorphism $F$ of $(X,\in)$
and $(M,\in)$. $(M,\in)$ is again a countable model of ZFC minus the power set axiom. All subsets of $F(\lambda)$ which are not elements
of $M$ are proper semisets from the point of view of $M$.


\subsubsection{II}

Another context, where semisets arise, are nonstandard models of ZFC. For example consider the ultrapower $V^\omega/{\mathcal U}$
where ${\mathcal U}$ is a nontrivial ultrafilter on $\omega$. In this structure there are proper semisets which are even parts of a natural number:
the class of standard natural numbers is an example of such a semiset. It has to be proper, since it has no maximal element. Conversely there
are parts of the natural numbers with no minimal elements, so, again, they must be proper semisets.

% \subsubsubject{A last comment}
% The method of forcing invented by P. Cohen is a method which, given suitable proper semisets, allows to extend the universe of sets into a
% new universe where these semisets become regular sets. Moreover the method is flexible in the sense that it allows to control the behavior
% of the new sets. This extension is called the \emph{generic} extension and the method involves the usage of names.

\subsubsection*{Questions\ \&\ Answers.}


${}$\indent{\scshape Question.} Why an axiom schema instead of an axiom?

\smallskip

{\scshape Answer.} For each formula $\varphi$ our schema introduces a single axiom. It follows that there are really infinitely many axioms.
It can be shown, that the theory $ZF$ cannot be finitely axiomatized, i.e. no finite list of axioms is equivalent to $ZF$. This is
unfortunate. However, the set of axioms is still rather simple, formally speaking, it is recursive. This is important in logic.

\medskip

{\scshape Question.} What are classes?

\smallskip

{\scshape Answer.} Every formula $\varphi$ of Set Theory naturally determines the collection of all sets satisfying $\varphi$. This collection
is denoted, formally incorrectly, by
$$ %\startformula
\{x:\varphi(x)\},
$$ %\stopformula
and called a class. We can extend most of the usual set operation to these objects even thought, in many cases, e.g. $V=\{x:x=x\}$, $On=\{\alpha:\alpha\ \mbox{is an ordinal}\}$, $E=\{\langle x,y\rangle:x\in y\}$, these objects are {\bf not} sets (it would lead to the famous Russell paradox). Axiom of replacement implies that the intersection of a class with a set is always a set. Using classes, we can e.g. write $(H(\kappa),E\cap H(\kappa)^2)$ instead of $(H(\kappa),\in)$.

\medskip

{\scshape Question.} What is the weird symbol $\models$?

\smallskip

{\scshape Answer.} Short answer: satisfaction. Long answer: it is a relation between models and formalized formulas.

\bigskip

For an in-depth discussion of these questions see your local logic guru.
\end{document}