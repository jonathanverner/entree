\setupcolors[state=start]       % otherwise you get greyscale
%\definecolor[headingcolor][r=1,b=0.4]
\usemodule[newmat]
\usetypescript[berry][ec]
% for the document info/catalog (reported by 'pdfinfo', for example)
\setupinteraction[state=start,  % make hyperlinks active, etc.
  title={Introduction to Forcing},
  subtitle={},
  author={Bohuslav Balcar, Jonathan Verner},
  keyword={forcing, posets, generic sets, iteration}]

% useful urls
\useURL[verner-email][mailto:jonathan.verner@matfyz.cz][][jonathan.verner@matfyz.cz]
\useURL[balcar-email][mailto:balcar@cts.cuni.cz][][balcar@cts.cuni.cz]
\setuppapersize[A4][A4]
\setuplayout[topspace=0.5cm, backspace=1cm, header=24pt, footer=36pt,
  height=middle, width=middle]

% uncomment the next line to see the layout
%\showframe

% headers and footers
\setupfooter[style=\it]
\setupfootertexts[\date\hfill Introduction to Forcing]
\setuppagenumbering[location={header,right}, style=bold]


%\setupbodyfont[12pt]            % default is 12pt
%\setupmathfont[10pt]
%\usetypescript[modern][ec]
%\setupbodyfont[modern]

\usetypescript[berry][ec] % or [8r]
\usetypescript[palatino][ec] %
\setupbodyfont[palatino,10pt]


\setuphead[section,chapter,subject][color=headingcolor]
\setuphead[section,subject][style={\bfd},
  before={\bigskip\begingroup}, after={\endgroup}]
\setuphead[subsection,subsubject][style={\bf}]
\setuphead[chapter][style={\ss\bfd}]
\setuphead[title][style={\bfd},
  before={\begingroup\setupbodyfont[10pt]},
  after={\leftline{\ss\tfa Bohuslav Balcar $\langle$\from[balcar-email]$\rangle$}
         \leftline{\ss\tfa Jonathan Verner $\langle$\from[verner-email]$\rangle$}
         \bigskip\bigskip\endgroup}]

\setupitemize[inbetween={}, style=bold]

% set inter-paragraph spacing
\setupwhitespace[medium]
% comment the next line to not indent paragraphs
\setupindenting[medium, yes]

\definenumber[counter]


\defineenumeration
  [lemma]
  [location=serried,
   text=Lemma,
   width=broad,
   after=\blank][number=counter]



\defineenumeration
  [thm]
  [width=broad,
   after=\blank]

\setupenumerations[thm]
  [headstyle=bold,
   location=serried,
   text=Theorem,
   sectionnumber=yes,
   conversion=numbers,
   way=bysection,
   number=counter]

\defineenumeration
  [definition][thm]
  [width=broad,
   text=Definition,
   after=\blank]

\defineenumeration
  [lemma][thm]
  [width=broad,
   text=Lemma,
   after=\blank]

\defineenumeration
  [fact][thm]
  [width=broad,
   text=Fact,
   after=\blank]

\definedescription
  [proof]
  [location=serried,
   text=Proof.,
   width=broad,
   after=\QED]

\definedescription
  [case]
  [location=serried,
   text=Case,
   width=broad]

\def\cf#1{%
  \hbox{cf}\ #1%
}

\def\force{\Vdash}
\def\aleq{\leq^{*}}
\def\succ{\hbox{succ}}
\def\pw{{\cal{P}}}


\starttext


%\title{Introduction to forcing}

\section{Set Theory prerequisites}

When we talk about Set Theory we mean the first order predicate theory in a language with equality containing a binary predicate $\in$
with the standard axioms\footnote{Ahoj}.



We assume the reader is familiar with naive set theory. In this sections we list, for her convenience, the axioms of set theory
and discuss some stuff we shall be needing later on.

\subsubject{Axioms of Zermelo-Fraenkel}


(taken from Jech)

\def\myitem#1{{\bf #1}}
\setupitemize[left={\determineheadnumber[section]\currentheadnumber.},right=,margin=0em,command=\myitem, stopper=]
\startitemize[n]
\item{Axiom of Extensionality.} If $X$ and $Y$ have the same elements, then
$X=Y$.
\item{Axiom of Pairing.} For any $a$ and $b$ there exists a set $\{a, b\}$ that
contains exactly $a$ and $b$.
\item{Axiom Schema of Separation.} If $P$ is a property (with parameter $p$),
then for any $X$ and $p$ there exists a set $Y = \{u \in X : P (u, p)\}$ that contains
all those $u \in X$ that have property $P$ .
\item{Axiom of Union.} For any $X$ there exists a set $Y =\bigcup X$, the union
of all elements of $X$.
\item{Axiom of Power Set.} For any $X$ there exists a set $Y = \pw (X)$, the
set of all subsets of $X$.
\item{Axiom of Infinity.} There exists an infinite set.
\item{Axiom Schema of Replacement.} If a class $F$ is a function, then for
any $X$ there exists a set $Y = F (X) = \{F (x) : x \in X\}$.
\item{Axiom of Regularity.} Every nonempty set has an $\in$-minimal element.
\item{Axiom of Choice.} Every family of nonempty sets has a choice func-
tion.
\stopitemize

The theory with axioms 1.1---1.8 is the Zermelo-Fraenkel axiomatic set theory ZF; ZFC denotes the theory ZF with the
Axiom of Choice.

\subsubject{Cumulative hierarchies}
In the theory ZF we can define the following:

\startdefinition A set $x$ is an {\bf ordinal number}, if it is wellordered (i.e. is linearly ordered and each subset has an $\in$-minimal element) by $\in$.
We shall denote ordinal numbers by lowercase greek letters $\alpha,\beta,\ldots$ from the beginning of the alphabet.

Using transfinite recursion over ordinal numbers we define the cumulative hierarchy:

\placeformula\startformula\startalign[n=4,align={left,middle,left, right}]
\NC V_0          \NC = \NC \emptyset\NC\NR
\NC V_{\alpha+1} \NC = \NC V_\alpha\cup\pw(V_\alpha)\NC \NR[eq:Valpha]
\NC V_\alpha     \NC = \NC \bigcup_{\beta<\alpha} V_\beta \NC\ \alpha\ \text{limit}\NR
\stopalign\stopformula

\stopdefinition
If we work in ZFC we can define the notion of a cardinal number:

\startdefinition A set $x$ is an {\bf cardinal number}, if it is an initial ordinal number (i.e. it is an ordinal number and
there is no 1-1 function onto $x$ from any smaller ordinal number). We denote cardinal numbers by lowercase greek letters $\kappa,\lambda,\ldots$ from the
middle of the alphabet.

Using transfinite recursion over cardinal numbers we define the rank of a set:

\startformula\startalign[n=4,align={left,middle,left}]
\NC r(\emptyset) \NC=\NC 0\NR
\NC r(x)         \NC=\NC \sup\{r(y):y\in x\}\cup\{|x|\}\NR
\stopalign\stopformula

and then define

\placeformula\startformula
H(\kappa)=\{x:r(x)<\kappa\}
\stopformula
\stopdefinition

Notice the following important fact

\startthm If $\kappa$ is a regular infinite cardinal, then the structure $\langle V_\kappa,\in\rangle$ satisfies all of the ZFC axioms except replacement
in the case when the function $F$ is a proper class. The structure $\langle H(\kappa),\in\rangle$ statisfies all of ZFC axioms
except the power set axiom which holds only for sets $x\in H(\kappa)$ with $|\pw(x)|<\kappa$.
\stopthm

\section{Generic sets --- an algebraic approach}

\section{Generic sets --- filters approach}

\section{Names for sets \& basic facts about forcing}

\section{Ordinal and cardinal numbers in generic extensions}

\startthm $On^V=On^{V[G]}$.
\stopthm

\startthm $\kappa$-closed implies preserving cardinals up to $\kappa$
\stopthm

\startthm $\kappa$-cc implies preserving cardinals above $\kappa$
\stopthm

\startthm Distributivity and new sets in the extension
\stopthm

\section{Examples of partial orders}
\subsubject{Collapse}
\subsubject{Cohen}
\subsubject{Random}
\subsubject{Sacks \& variants.}
\subsubject{Mathias \& Prikry}

\stoptext

