
\section{Examples of Forcing}
\subsection{Collapse}
\begin{definition}For two cardinal numbers $\kappa,\lambda$ we define the following partial order:
 \begin{displaymath}
  Col(\kappa,\lambda)=(\{f:\dom f\in[\kappa]^{<\kappa}, \rng f\subseteq\lambda\},\supseteq)
 \end{displaymath}
\end{definition}

\begin{prop} If $\kappa$ is a regular cardinal then $Col(\kappa,\lambda)$ is $\kappa$-closed.
\end{prop}

\subsection{Cohen}
\begin{definition} If $\kappa$ is a cardinal, we define the generalized \emph{Cantor} order:
 \begin{displaymath}
  Cantor(\kappa)=(Clopen(2^\kappa),\subseteq)
 \end{displaymath}
\end{definition}

\begin{prop} Cohen adds an independent real.
\end{prop}


\subsection{Random}
\begin{definition} The \emph{random forcing} consists of lebesgue measurable sets (e.g. on the interval $[0,1]$)
 ordered by inclusion.
\end{definition}
\begin{prop} Random is an ${}^\omega\omega$-bounding real which adds an independent real.
\end{prop}

\subsection{Tree forcings}
\subsubsection{Sacks}
\subsubsection{Miller}
\subsubsection{Bukovsk\'y $\&$ Namba}
\subsection{Mathias $\&$ Prikry}
