\begin{exercise}{Dependent Choice}
\end{exercise}

\begin{exercise}{monotone_cardinal_characteristics}
\begin{prop} Suppose $\chi$ is a (class) function assigning to each partial order $P$ a ordinal number $\chi(P)$. Suppose that this
-                     function is \emph{monotone} in the sense that $(\forall p\in P)(\chi((\leftarrow,p])\leq\chi(P))$. Then there is a partition of
-                     $P$ into factors such that $\chi$ is constant on each of the factors.
\end{prop}
\end{exercise}
\begin{exercise}{horn_tarski}
Several partial orders satisfy the following strengthening of ccc due to Horn and Tarski:
\begin{definition} An ordering $\P$ is called
\begin{itemize}
 \item [(i)] $sigma$-finite cc if $\P$ can be partitioned into countably many sets $P_n$ such that each antichain contained in $P_n$ is finite.
 \item [(ii)] $sigma$-bounded cc if $\P$ can be partitioned into countably many sets $P_n$ such that each antichain contained in $P_n$
              has cardinality $<2^n$
\end{itemize}
\begin{note} In (ii) we could have used any strictly increasing function instead of $2^n$ to get the same.
\end{note}

It is still an open problem (and one that is thought to be very hard) whether conditions (i) and (ii) are equivalent.
\end{definition}
\end{exercise}

\begin{exercise}{cohen_measure}
 Neither of the Cohen and Random algebras can be embedded into the other as complete subalgebras. This implies, that
 the Cohen algebra does not add a Random real and the Random algebra does not add a Cohen real. Cohen can, however,
 be embedded into Random as a normal subalgebra, Random, on the other hand, cannot be embedded into Cohen,
\end{exercise}

\begin{exercise}{sigma-centered}
 The class of $\sigma$-centered algebras consists exactly of subalgebras of $\pomega$. In particular any such algebra
 must be of cardinality $<\cont$.
\end{exercise}

\begin{exercise}{sigma-centered-cohen}
%Ka�d� atomless sigma-centered forcing p�id�v� cohena iff neex. nwd. ulf. na omeze.
%Blazczyk-Shelah
\end{exercise}

\startexercise{} <a,\leq> is isomorphic to <\alpha,\in> in H(k). In V(k) only if |V(k)|=|k|. There is a class of k's such that |V(k)|=|k|.
\stopexercise

\begin{exercise}
 A set is an ordinal iff it is transitive and each of its elements is transitive.
\end{exercise}

\begin{exercise}
 DC implies AC for countable sets.
\end{exercise}

\begin{exercise}
 Cohen makes ground model reals meagre, Measure makes them null.
\end{exercise}

\begin{exercise}
 Prikry forcing
\end{exercise}

\begin{exercise}
Full product omega cohenů je kolapsovka
\end{exercise}

\begin{exercise}
 Side by side Sacks funguje, ale side by side produkt millera, laver,  ... je kolapsovka (side by side tri uz prida cohena)
\end{exercise}

\begin{exercise} (Martin's axiom)
Je konsistentni, ze neex. kappa>omega, ze kappa^<kappa = kappa.
\end{exercise}



