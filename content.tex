\cfoot{}\rhead{\thepage}
\lhead{{\scshape Content} $\qquad$ {\tiny \today } }

\noindent{\large{\scshape\bfseries %Entre{\' e}
An Exposition of Generic Extensions and Forcing in Set Theory}} \\[0.1cm]

\noindent {\scshape Bohuslav Balcar}, %{\small CTS, J{\' \i}lsk{\' a} 1, Praha 1,
	%Czech Republic, {\ttfamily
	%\href{mailto:balcar@cts.cuni.cz}
	{balcar@math.cas.cz}\footnotemark[1]\footnotetext[1]{Supported in part by
	the GAAV Grant IAA100190509 and by
	the Research Program CTS MSM 0021620845} \\[0.1cm]
\noindent {\scshape Tom{\' a}{\v s} Paz{\' a}k}, %{\small CTS, J{\' \i}lsk{\' a} 1, Praha 1,
	%Czech Republic, {\ttfamily
	%\href{mailto:pazak@cts.cuni.cz}
	{pazak@math.cas.cz}\footnotemark[1] \\[0.1cm]
\noindent {\scshape Jonathan Verner}, %{\small KTIML MFF UK, Praha,
	%Czech Republic {\ttfamily
	%\href{mailto:jonathan.verner@matfyz.cz}
	{jonathan.verner@matfyz.cz}\footnotemark[2]\footnotetext[2]{Supported in part by the GAČR Grant no. 401/09/H007 Logical foundation of semantics}
	\\[0.1cm]
{\tiny \today } \\[0.5cm]

%\maketitle

\thispagestyle{empty}

% \noindent{\Large{\scshape\bfseries Entreé to Generic Extensions and Forcing in Set Theory}} \\[0.1cm]
%
% \noindent {\scshape Bohuslav Balcar}, {\small CTS, J{\' \i}lsk{\' a} 1, Praha 1,
% 	Czech Republic, {\ttfamily balcar@cts.cuni.cz} } \\[0.1cm]
% \noindent {\scshape Jonathan Verner}, {\small KTIML MFF UK, {\ttfamily jonathan.verner@matfyz.cz}\\[0.1cm]
% {\tiny \today } \\[0.5cm]

%\maketitle

%\thispagestyle{empty}

%%%%%%%%%%%%%%%%%%%%%%%%%%%%%%%%%%%%%%%%%%%%%%%%%%%%%%%%%%%%%%%%%%%%%%%%%%%
%%%%%%%%%%%%%%%%%%%%%%%%%%%%%%%%%%%%%%%%%%%%%%%%%%%%%%%%%%%%%%%%%%%%%%%%%%%
\section*{Annotated Content}

\pagenumbering{roman}

${}$\\[-0.1cm] \hrule\hrule
${}$\\[0.2cm]
{\scshape 1. Set Theory Prerequisites \hfill 1 -- 10 } \\[-0.3cm]
\hrule
${}$\\[-0.1cm]

\noindent We mention the axioms of Zermelo~-~Fraenkel Set theory with
axiom of choice (ZFC), transitive sets and ordinal numbers. \ We define
the $V_\alpha$ and $H_\kappa$ hierarchies to illustrate the possibilities of
working with fragments of ZFC. \ We review well-founded relations, induction
and recursion. \ Some basics concerning trees are introduced
together with an example of a well-founded $\omega$-ary tree of rank
$\alpha$, for arbitrary $\alpha < \omega_1$. \ We introduce
the Mostowski collapse and elementary substructure techniques. \ We conclude
the chapter with the definition of a semiset and some motivation for this
concept.

\smallskip

\noindent This chapter is not intended to be self-contained but is rather
meant as a quick summary of what is needed in the sequel.

${}$\\[-0.1cm] \hrule\hrule
${}$\\[0.2cm]
{\scshape 2. Orderings and Boolean Algebras \hfill 11 -- 26 } \\[-0.3cm]
\hrule
${}$\\[-0.1cm]

\noindent In this chapter we introduce many notions that are required to
understand basic forcing combinatorics. We start with the definition of an ordering,
its separative quotient and with some basic combinatorial properties of orderings
e.g. $\kappa$-$cc$, $\kappa$-closed, $\sigma$-centred. These
properties are illustrated on important orderings from forcing
theory such as Cohen's ordering, Collapsing orderings and Levi collapse. \
Rasiowa~-~Sikorski theorem. \ $\Delta$-system lemma. \ We review the
concept of a (M)AD family on $\omega$.

\smallskip

\noindent In the second half we define Boolean algebras and thoroughly investigate
the connection betwixt orderings, Boolean algebras and zero-dimensional
compact Hausdorff spaces. We conclude this chapter with a preparation for
forcing techniques. \ Work with matrices and hierarchy of names. \
Use of elementary submodels.

${}$\\[-0.1cm] \hrule\hrule
${}$\\[0.2cm]
{\scshape 3. Forcing Notion and Generic Filters \hfill 27 -- 33 } \\[-0.3cm]
\hrule
${}$\\[-0.1cm]

First we define when orderings are forcing equivalent and
introduce two criteria. \ We define the generic object and discuss
its existence. \  Some motivation for generic extensions and generic objects
is given. \ We conclude with the Balcar~-~Vopěnka theorem about the canonical
form of a generic object. \ Glimpse of a description of the generic extension.

${}$\\[-0.1cm] \hrule\hrule
${}$\\[0.2cm]
{\scshape 4. Classical Forcing Notions \hfill 34 -- 60 } \\[-0.3cm]
\hrule
${}$\\[-0.1cm]

\noindent We start with a description of basic techniques that will be needed to
introduce classical forcing notions. \ More on working with names, Boolean names,
forcing relation. \ We bring out many phenomena known
from forcing theory together with examples that illustrate these.

\smallskip

\noindent We start with $ccc$ forcings that add new reals. \ Cohen forcing. \
Random forcing, measure algebra. \ We show how to violate CH. \ Proof that $ccc$ forcing
preserves cardinalities and cofinalities. \ Hechler forcing and adding
a dominating real. \ Adding a filter pseudointersection. \ Non $ccc$ forcings.
Mathias forcing, infinite Ramsey theory. \ Three parameter distributivity. \
Base tree techniques.

${}$\\[-0.1cm]
\hrule\hrule
${}$\\[0.2cm]
{\scshape Forcing Relations \hfill 61 -- 70 } \\[-0.3cm]
\hrule
${}$\\[-0.1cm]

\noindent Chapter in progress.

${}$\\[-0.1cm] \hrule\hrule
${}$\\[0.2cm]
{\scshape Generic Extension, Iteration \hfill 70 -- 77 } \\[-0.3cm]
\hrule
${}$\\[-0.1cm]

\noindent Chapter in progress.

${}$\\[-0.1cm] \hrule\hrule
${}$\\[0.2cm]
{\scshape Solovay's Model, Where All Sets are Measurable \hfill 78 -- 102 } \\[-0.3cm]
\hrule
${}$\\[-0.1cm]

\noindent Chapter in progress.

${}$\\[-0.1cm] \hrule\hrule
${}$\\[0.2cm]
{\scshape Index \hfill 103 -- 105 } \\[-0.3cm]
\hrule
${}$\\[-0.1cm]


${}$\\[-0.1cm] \hrule\hrule
${}$\\[0.2cm]
{\scshape References \hfill 106 } \\[-0.3cm]
\hrule
${}$\\[-0.1cm]

\noindent List of references.

%{\tiny \today } \\[0.5cm]

%%%%%%%%%%%%%%%%%%%%%%%%%%%%%%%%%%%%%%%%%%%%%%%%%%%%%%%%%%%%%%%%%%%%%%%
%%%                          END                                    %%%
%%%%%%%%%%%%%%%%%%%%%%%%%%%%%%%%%%%%%%%%%%%%%%%%%%%%%%%%%%%%%%%%%%%%%%%
