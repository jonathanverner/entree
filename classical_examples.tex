\cfoot{}\rhead{\thepage}
\lhead{{\scshape Forcing} $\qquad$ {\tiny \today } }

% \noindent{\Large{\scshape\bfseries Entreé to Generic Extensions and Forcing in Set Theory}} \\[0.1cm]
%
% \noindent {\scshape Bohuslav Balcar}, {\small CTS, J{\' \i}lsk{\' a} 1, Praha 1,
% 	Czech Republic, {\ttfamily balcar@cts.cuni.cz} } \\[0.1cm]
% \noindent {\scshape Jonathan Verner}, {\small KTIML MFF UK, {\ttfamily jonathan.verner@matfyz.cz}\\[0.1cm]
% {\tiny \today } \\[0.5cm]

%\maketitle

\thispagestyle{empty}

%%%%%%%%%%%%%%%%%%%%%%%%%%%%%%%%%%%%%%%%%%%%%%%%%%%%%%%%%%%%%%%%%%%%%%%%%%%
%%%%%%%%%%%%%%%%%%%%%%%%%%%%%%%%%%%%%%%%%%%%%%%%%%%%%%%%%%%%%%%%%%%%%%%%%%%
\section{Classical forcing notions.}

{\tiny \today } \\[0.5cm]

We introduce basic examples of forcings, which were studied
together with the extensions they determine,
as one of the first. Namely Cohen, Random (Solovay), Hechler, Mathias,
Laver, Miller and Sacks forcing. We show some phenomena that
could arise in particular generic extensions, we focus
on those phenomena that can be expressed only using
semisets, i.e. we work only in $(\mathcal P^{V[G]} (V)) \subset V[G]$.

We bring out the standard definition of particular forcing notion
and if possible we also give alternative description using
subsets of the Cantor space $\mathcal C$ or Baire space $\mathcal N$.

First we start with notation and remind well known elementary facts
concerning trees, $\sigma$-fields of Borel sets and sets with
Baire property in Cantor $\mathcal C$ and Baire $\mathcal N$ spaces
and their quotients and some basics of measure theory.

\begin{notation}
Symbols $\exists^\infty$ and $\forall^\infty$ are abbreviations
of 'there is infinitely many' and 'for all except finitely many'
respectively.

\medskip

 Let $r$ be a binary relation, instead of $r''\{x\} = \{ y \in rng(r) : \< x , y \> \in r \}$,
 we shall use $(r)_x$. Similarly we denote
 $(r)^y = r^{-1}{}''\{y\} = \{x \in dom(r) : \< x , y \> \in r \}$.
\end{notation}


\subsection{${}$ \hspace{-1em}Properties of trees.}
We call a tree $T$ a \emph{binary tree} if it is a subtree
of  $(Seq(2),\subseteq)$, or \emph{$\omega$-ary}
is it is a subtree of $(Seq,\subseteq)$, note that always $\emptyset \in T$.
We are mainly interested in trees that have all branches infinite.
\begin{itemize}
 \item[(i)] $t \in T$ is splitting node if there exists $i \not = j$
	such that $t^\smallfrown i, t^\smallfrown j \in T$, for
	a splitting node we define splitting set
	$splt(t) = \{i \in \omega : t^\smallfrown i \in T\}$.
 \item[(ii)] Stem of tree $T$ is the splitting node of smallest rank.
 \item[(iii)] If $s \in T$ then $T(s) = \{t \in T : t \subseteq s \ \mbox{or} \
	 t \supseteq s \}$
	is a subtree of $T$.
 \item[(iv)] $[T]$ is the set of all infinite branches of $T$.
\end{itemize}

% \begin{example}
% There is a closed correspondence between Cantor space $\mathcal C$ and the binary
% tree $(Seq(2),\subseteq)$ and between Baire space $\mathcal N$ and $\omega$-ary tree $(Seq,\subseteq)$.
%  \begin{eqnarray*}
% {} [Seq(2),\subseteq] & \ = \ & \mathcal C = {}^\omega \{0,1\}, \\
% {} [Seq,\subseteq] & \ = \ & \mathcal N = {}^\omega \omega.
%   \end{eqnarray*}
% \end{example}

There is a close correspondence between trees and closed sets.

\begin{fact}
\begin{itemize}
 \item[(i)] It $T$ is a binary tree, then $[T]$ is closed subset of Cantor space $\mathcal C$.
 \item[(ii)] If $\emptyset \not = U \subset \mathcal C$ is a closed set, then
	$T_U = \{f \upharpoonright n : f \in U, \ n \in \omega \}$
	is a tree and $[T_U] = U$.
\end{itemize}
The similar statement holds for Baire space $\mathcal N$.
%  Let $U \not = \emptyset$ be closed subset of Cantor $\mathcal C$, or Baire space  $\mathcal N$,
% then  $T_U = \{f \upharpoonright n : f \in U, \ n \in \omega \}$ is a tree and
% $[T_U] = U$. On the other hand when $T$ is a tree then $[T] = U$ is a closed
% sub and moreover the tree $T_U$ is the smallest subtree of $T$ that does not
% contain finite branches, i.e. all its branches are infinite.
\end{fact}

Perfect set in a topological space is a nonempty, closed subset without isolated
points. The following definition reveals the interplay between trees and
subsets of the Cantor space.

\begin{definition}
Tree $T$ is called
\begin{itemize}
 \item[(i)]%{\bf Perfect tree.}
	\emph{perfect} if satisfies
	$$
	(\forall f \in T) \ (\exists g \in T) \ (g \supseteq f) \
	\mbox{and} \ g \ \mbox{is a splitting node of} \ T.
	$$
	Perfect trees correspond to non-empty closed subsets of
	$\mathcal C$ without isolated points.
	%Tree $T \subseteq (Seq,\subseteq)$ is
	\item[(ii)]%{\bf Nowhere dense.} $T$ is a
	\emph{nowhere dense} if satisfies
	$(\forall f \in T)\ \exists g \in T, \ g \supset f$,
	such that for each $h \in T$, $h \supseteq g$ $h$
	is not a splitting node of $T$.
	Nowhere dense trees correspond to nowhere dense
	subsets of $\mathcal C$.
	\item[(iii)] \emph{superperfect} if $T$ is $\omega$-ary
	tree and satisfies
	$$
	(\forall s \in T)\ \exists t \in T, \ t \supset s \ \&
	\ \mbox{split}(t) = \omega.
	$$
	We say that $[T]$ is \emph{superperfect} subset of $\mathcal N$.
\end{itemize}
\end{definition}

\noindent{\scshape Notice} that any superperfect subset of $\mathcal N$ cannot be covered
by countably many compact subsets. Moreover, if $\emptyset \not = U \subset \mathcal N$
is closed set then $U$ either contains superperfect subset or is
covered by countably many compact subsets.

\smallskip
\noindent{\scshape Hint.} Step by step prune the superperfect tree $T_U$
by the nodes with only finitely many successors. After finitely many steps
we obtain either empty set, or superperfect subtree.


% $(Seq \nearrow,\subseteq )$ is a subtree of $(Seq,\subseteq )$ consisting of strictly
% increasing finite sequences of natural numbers. $[(Seq \nearrow,\subseteq )]$ is a
% closed subset on $\mathcal N$, hence a Polish space.



\subsection{${}$ \hspace{-1em}Bases of the space $\mathcal C$.}


For $t \in Fn(\omega,\{0,1\};\omega)$ denote
$[t] = \{f \in \mathcal C : f \supset t\}$.
\begin{itemize}
 \item[(a)] Let
	$$
	\mathcal B_1 \ = \ \{[t] : t \in Fn(\omega,\{0,1\};\omega) \}.
	$$
	$\mathcal B_1$ is a base for topology of $\mathcal C$ and mapping
	$t \mapsto [t]$ is an isomorphism of
	$(Fn(\omega,\{0,1\};\omega), \supseteq)$ onto $(\mathcal B_1,\subseteq)$.
 \item[(b)] Let $\mathcal B_2 \ = \ \{[s] : s \in {}^n \{0,1 \} \}$.
	$\mathcal B_2$ is also a base for topology of $\mathcal C$ and
	$\mathcal B_1 = \{ [ k \oplus s ] : s \in {}^n \{0,1 \} \ \& \ k \in \omega \}$,
	where
 \item[(c)] $
	k \oplus s \in Fn(\omega,\{0,1\};\omega), \ \mbox{dom}(k \oplus s) = [k,k+|s|) \
	\mbox{and} \ (k \oplus s)(k+i) = s(i), \ \mbox{for each} \ i < |s|.
	$ is a \emph{shift} of $s$.
\end{itemize}

Define a mapping $m: \mathcal B_1 \to [0,1]$ by
$$
m[t] = \dfrac{1}{2^|t|}.
$$

This mapping has a unique extension to algebra of clopen subsets
of Cantor space. If $U \in Clop(\mathcal C)$ then $U$
can be expressed as a finite disjoint union of  base sets
$U = \cup \{[t_i] : i < k  \}$ and one can put
$m(U) = \sum_{i < k} m[t_i]$.

This value does not depend on the representation on $U$ as
a union of finite base sets. Mapping $m$ is a normalised
$m(\mathcal C) = m[\emptyset] = 1$, finitely additive measure
on the algebra of clopen sets.

This measure can be uniquely extended to a
$\sigma$-additive measure on the field of Borel subsets of $\mathcal C$,
which we denote $m$ as well. Such measure is continuous,
i.e. $m(\{x\}) = 0$ for each $x \in \mathcal C$. So $m$ is
an example of of probability continuous Borel measure.

% $$ %\startformula
%  R=\{T:\mbox{\ T is a binary tree without finite branches such that \ } \lim \frac{|T_n|}{2^n} > 0\},
% $$ %\stopformula
%
% Then $R$ ordered by inclusion is called the \emph{random} forcing. Note that $|T_n|/2^n$ is a (not necessarily strictly) decreasing sequence.
% It can be shown that $RO(R)\approx \mbox{Borel}(2^\omega)/{\mathcal N}$ where ${\mathcal N}$ is the $\sigma$-ideal of subsets of $2^\omega$ which have
% measure zero when considering the standard Haar measure on $2^\omega$. Random forcing is ccc since its completion is isomorphic to a Boolean algebra
% carrying a strictly positive non-atomic $\sigma$-additive probability measure. In contrast to the Cohen forcing notion, it is \emph{not} $\sigma$-centred,
% which follows from the following important property of measure algebras:

We denote $Null$ the ideal of subsets of measure zero, if measure
is $\sigma$-additive then $Null$ is $\sigma$-complete ideal.



$(\mbox{Borel}(\mathcal C) / Null,\subseteq)$ is particular but important example of so
called \emph{measure algebra}.

\begin{definition}
 A complete Boolean algebra $B$ is called \emph{measure algebra} if $B$ carries
a strictly positive probabilistic measure $\mu$, i.e.
 \begin{itemize}
  \item[(i)] $\mu : B \to [0,1]$,
  \item[(ii)] $\mu(a) = 0$ iff $a = \mathbf 0_B$ ($\mu$ is \emph{strictly positive})
  \item[(iii)] for any disjoint family $\<a_n : n \in \omega \>$
	$$
 	\mu(\bigvee \{a_n : n \in \omega \}) \ = \ \sum_0^\infty \mu(a_n),
	$$
	i.e. $\mu$ is $\sigma$-additive.
  \item[(iv)] $\mu(\mathbf 1_B) = 1$, i.e. $\mu$ is \emph{normalised}.
 \end{itemize}
\end{definition}

\begin{fact}
 Any measure algebra satisfies $ccc$.
\end{fact}

\begin{proof}
 Let $\mu$ be a measure on $B$ witnessing that $B$ is a measure algebra.
 Let there be uncountably many disjoint elements in $B$. Then since $\mu$ is
 strictly positive, there are uncountably
 many elements among them with measure greater or equal to $\frac{1}{n_0}$, for some
 $n_0 \in \omega$. Using $\sigma$-additivity of measure we contradict the fact
 that $\mu$ is normalised.
\end{proof}

\begin{example}
 $(\mathcal P(\omega),\subseteq)$ is a measure algebra. It is even complete field
 of sets - so it is a complete and atomic Boolean algebra. Let
$$
\nu(A) = \sum \{ \frac{1}{2^{n+1}} : n \in A \}, \ \mbox{for each} \ A \subset \omega.
$$
$\nu$ is probabilistic measure on $\mathcal P(\omega)$. This algebra is
diametrically different from $\mbox{Borel}(\mathcal C) / Null$.
\end{example}


\subsection{${}$ \hspace{-1em}Measure algebra.}
Our Borel measure $m$ on the Cantor space corresponds to Lebesgue measure
on the unit interval $[0,1]$ of real line. There is a continuous mapping
\begin{eqnarray*}
 \varphi : \mathcal C & \longrightarrow & [0,1] \\
		f & \longmapsto & \varphi(f) = \sum_{i=0}^\infty \dfrac{f(i)}{2^{i+1}},
\end{eqnarray*}
which is onto and one-to-one with exception for dyadic rational numbers,
where $\varphi$ is 'two-to-one'. Nevertheless, for any $U \in \mbox{Borel}(\mathcal C)$
is the set $\varphi[U]$ Borel in $[0,1]$ and $m(U)$ is equal to Lebesgue
measure of $\varphi[U]$.

\medskip

This can be generasized to the following theorem, see \cite{Ke:1994}

\begin{theorem}
 Let $\mu$ be a probability continuous Borel measure on a Polish space $S$. Then
there is  a Borel isomorphism $\varphi : \mathcal C \to S$, i.e. $\varphi$ is
one-to-one onto mapping that induces a measure preserving isomorphisms of
the fields Borel$(\mathcal C)$ and Borel$(S)$.
\end{theorem}

\begin{theorem}
 Any continuous probability Borel measure $\mu$ on a Polish space $S$ is regular
(even Radon), i.e.
\begin{itemize}
 \item[{}] $\mu(A)  = \inf \{\mu(U) : U \supseteq A,\ U \ \mbox{open} \ \}$,
 \item[{}] $\mu(A)  = \sup \{\mu(K) : K \subseteq A, \ K \ \mbox{compact} \}$.
\end{itemize}
\end{theorem}



\begin{fact}
 Let $\emptyset \not = U \subset \mathcal C$ is closed, then for a tree
 $T(U) = \{ f \upharpoonright n : f \in U, \ n \in \omega \}$,
$$
m(U) \ = \ \lim_{n \in \omega} \dfrac{1}{2^n} \cdot |T_n(U)|.
$$
\end{fact}


% Let $s \in Seq(2)$ then
% $\langle s \rangle = \{f \in {}^\omega 2 : s \subseteq f\}$
% and $\mathcal B_{\mathcal C} = \{\langle s \rangle : s \in Seq(2) \}$
% is a topology base of $\mathcal C$,
% we call such sets \emph{intervals}. Similarly we obtain a base
% $\mathcal B_{\mathcal C}$ for the Baire space $\mathcal N$.
%
% % \begin{remark}
% %  Every open subset can be expressed as at most countable union of
% %  intervals. Each clopen subset can be expressed as a union of
% %  finitely many disjoint intervals. Let $|s|$ denote the length of
% %  a finite sequence, so each clopen subset  can be \emph{uniquely} expressed
% %  as a union of finitely many disjoint intervals of minimal possible length.
% % \end{remark}
%
% Now there is a simple path that leads to a measure. Let $|s|$ denote the
% length of a finite sequence $s \in Seq(2)$. For the set $\langle s \rangle$
% let
% $$
% m(\langle s \rangle) = \dfrac{1}{2^{|s|}}.
% $$
%
% By the preceding fact each clopen $U$ set is expressed as $U = \bigcup_{i \leq k} \langle s_i \rangle$,
% hence $m(U) = \sum_{i \leq k} m(s_i)$. The function $m$ is a finitely additive
% normalised, i.e. $m(\langle \emptyset \rangle) = m(\mathcal C) = 1$, measure on
% algebra $Clop(\mathcal C)$.
%
%
%
% \medskip
%
%
%
% %
% % If there are new reals in $V[G]$ then
%
%
% % \subsection{${}$ \hspace{-1em}Trees on $\omega$.}
% %
% % We continue with a list of some of the standard forcing notions which are named after authors who first used them. We shall be looking at them
% % mostly as forcing notions arising in the context of trees so we first sum up some standard definitions:
% %
% % \begin{definition}\label{Baire-space}{\bf Baire space.}
% %  Recall that the Baire space, denoted by $\N$, consists of infinite sequences of natural numbers, so:
% % $$ %\startformula
% %  \N={}^\omega\omega,\quad d(f,g)=\frac{1}{min\{k+1:f(k)\neq g(k)\}},\ f\neq g\in\N,
% % $$ %\stopformula
% % where $d$ is a complete metric on $\N$. $\N$ with the topology generated by this metric is a \emph{Polish space}, i.e. completely metrizable,
% % separable topological space. Moreover $\N$ is homeomorphic to the irrational numbers on the real line.
% % \end{definition}
% %
% %
% % \begin{definition}
% % Given a tree $T$, elements of $T$ are called \emph{nodes}. Given a node $t$ we abuse the notation slightly and write
% % $|t|=lh(t)=|dom(t)|$. A $t\in T$ is a \emph{stem} of $T$ if it is maximal, such that for any $s\in T$ with
% %  $|s|\leq|t|$ we have $s\subseteq t$. A node $t\in T$ is \emph{splitting} if there are incompatible $s,u$ extending $t$
% %  with $|s|=|u|=|t|+1$. A \emph{branch} of $T$ is any maximal linearly ordered subset of $T$, the set of branches of $T$ is denoted by $[T]$.
% %  By $T_n$ we denote the $n^{th}$ level of $T$, i.e.$T_n=\{t\in T:|t|=n\}$. Given a tree $T$ and $s\in T$ we write $T_s=\{t\in T:s\subseteq t\ \vee\ t\subseteq s\}$.
% % \end{definition}
% %
% % \begin{notation}
% %  $Seq$ will be the set of finite sequences of natural numbers whereas $Seq\uparrow$ will be the set of finite increasing sequences of natural numbers (not necessarily strictly increasing). In this notation an $\omega$-tree on $omega$ is a subset of $Seq$.
% % For a splitting node $s$ of such tree $T$, we write $spl(s)=\{n:s^\smallfrown n\in T\}$. A tree $T\subseteq Seq$ is \emph{pruned} if it has no finite branches.
% % \end{notation}
% %
% % \begin{fact}
% %  If $T\subseteq Seq$ is a pruned tree then $[T]$ is a closed subset of $\N$. Also, whenever $F$ is a closed subset of $\N$, then $\{f\upharpoonright n:n<\omega,f\in F\}$
% %  is a pruned tree.
% % \end{fact}
% %
% % \begin{fact}
% % A pruned tree $T$ determines a perfect subset of $\N$ (i.e. a closed subset without isolated points) iff every node of $T$ can be extended to
% %  a splitting node of $T$.
% % \end{fact}
% %
% %
% % \begin{fact}
% % $Seq\upharpoonright$ itself is a pruned tree and $[Seq\upharpoonright]$ consists of enumerations of all infinite subsets of $\omega$.
% % \end{fact}
% %
% % Note that the previous facts and definitions easily generalise to
% % $$ %\startformula
% % Seq(\kappa)=\N(\kappa)=\prod_{n<\omega}\kappa={}^\omega\kappa,
% % $$ %\stopformula
% % where $\kappa\geq 2$. $\N(2)$ is homeomorphic to the cantor space and $\N(\kappa)$ is a complete nonseparable metric space if $\kappa>\omega$.

\subsection{${}$ \hspace{-1em}Reals.}
Phrase 'Forcing notion \emph{adds a new real}' means that for
any generic filter $G$ over $V$, the generic extension $V[G]$
contains a new subset $\sigma \subset \omega$ of natural numbers
and hence $V[G]$ contains a function $\rho: \omega \to \omega$ which does
not belong to groundmodel $V$. In our terminology these are
semisets of $V$ that are obtained as $r''G$ for some
relation $r \in V$.

It is quite common in set theory that under the term 'real'
we mean subset of $\omega$. Hence elements of Cantor space
$\mathcal C = {}^\omega \{0,1\}$ are reals as well as the
function from $\omega$ to $\omega$, i.e. elements from
Baire space $\mathcal N$ are called reals.
The following definition reveals an interplay between
new reals and groundmodel reals.

\begin{definition}
Let $M$ denote an extension of $V$.
\begin{itemize}
\item[(i)] $X\subseteq \omega$ in
	the extension is said to be an \emph{independent}
	 (or \emph{splitting}) \emph{real}\index{real! independent}\index{real! splitting}
	over $V$ if for all $Y\in[\omega]^\omega\cap V$ both $X\cap Y$ and
	$Y-X$ are infinite.
\item[(ii)] A function $f\in M$, $f\in \omega^\omega$,
	is a \emph{dominating real} over $V$ if for all
	$g\in \omega^\omega \cap V$ for all but finitely many
	$n\in \omega$, $g(n) \leq f(n)$.\index{real! dominating}
\item[(iii)] A function $h \in \omega^\omega$ in the extension is said to be an
	\emph{unbounded real} over $V$ if for all $f \in \omega^\omega \cap
	V$   the set $\{n \in \omega : h(n) > f(n) \}$ is infinite.
	\index{real! unbounded}
\item[(iv)] A function $h \in \omega^\omega$ in the extension is said to be an
	\emph{eventually different} real over $V$ if for all $f \in \omega^\omega \cap
	V$   the set $\{n \in \omega : h(n) \not = f(n) \}$ is infinite.
	\index{real! eventually different}
\item[(v)] $M$ is an \emph{${}^\omega \omega$-bounding extension} of
	$V$ if every $f\in M$, $f\in \omega^\omega$ is bounded by a
	$g\in \omega^\omega\cap V$, i.e. $f(n)\leq g(n)$ for any $n$.
	\index{extension! ${}^\omega \omega$-bounding}
\end{itemize}
\end{definition}

\begin{remark}
 Each dominating real is eventually different, hence if forcing adds a dominating
 real it automatically adds an eventually different real.
\end{remark}


\subsection{${}$ \hspace{-1em}Examples of forcing notions.}

Each of the following forcing notions are atomless and are constructed
so that the generic extension will contain a new real. Note that the
conditions of each example are made so it is a finite approximation of
desired new subset of reals and generic filter picks up some of them
and assembles a new real.

All presented examples preserves cardinal number $\omega_1$ in $V$,
i.e. in no generic extension there is a mapping of $\omega$ onto
$\omega_1$. These examples are moreover 'proper' - notion that will
be defined later in \ref{}.

On a classical examples of forcings we illustrate
\begin{itemize}
 \item[(a)] some phenomena that should arise in generic extension,
 \item[(b)] a profitability of using equivalent descriptions of
	of a particular forcing notion,
 \item[(c)] some elementary techniques and ideas that should make
	the general statements easier to understand;
	the use of Boolean matrices; absoluteness
	of cardinal numbers in generic extensions; some forms of
	distributivity; Laver and Sacks property and some others
\end{itemize}

In the Definition \ref{compare} there were introduced a forcing
equivalence. What does this equivalence means to generic extension?
If orderings $P,Q$ are equivalent $P\sim Q$, then for each generic filter
$G$ on $P$ over $V$, there is generic filter $G'$ on $Q$ over $V$
such that $G$ and $G'$ are similar and so $V[G] = V[G']$: Since
$G$ and $G'$ are similar there are relations $r_1$ and $r_2$ in $V$
such that $r_1''G = G'$ and $r_2''G' = G$, so for each relation
$s \in V$ we get $s''G = (r_2 \circ s)''G'$, where
$r_2 \circ s = \{\langle x,y \rangle : (\exists z) \langle x,z \rangle \in r_2 \
	\& \ \langle z,y \rangle \in s \}$.

\noindent{\scshape Conclusion.} When investigating generic extension
one can use any equivalent of forcing notion and can skip from
one equivalent to another according the demand.

%%%%%%%%%%%%%%%%%%%%% COHEN %%%%%%%%%%%%%%%%%%%%%
\subsection{${}$ \hspace{-1em}Cohen forcing.}

Cohen forcing is the forcing used by P.~Cohen in his pioneer paper,
it is countable ordering and is equivalent to any of the following
order
\begin{itemize}
 \item[(a)]  $Seq = \bigcup\{ ^n\omega:n<\omega\}$,
 \item[(b)]  $Seq_2 = \bigcup\{ ^n2:n<\omega\}$,
 \item[(c)]  $Fn(\omega,2)=\{f;f:D\to\{0,1\},D\in[\omega]^{<\omega}\}$,
 \item[(d)] Clop$(2^\omega)^+$,
\end{itemize}
and their completions are isomorphic to a complete Boolean algebra
Borel$(2^\omega) - \mathcal M$, where $\mathcal M$ is the $\sigma$-ideal
of meagre sets.

\begin{definition}{\bf Cohen forcing.}\label{cohen}
 A complete, atomless $ccc$ Boolean algebra wit countable dense set is
called \emph{Cohen algebra}.
\end{definition}

We already know, that any Cohen algebras are isomorphic (they have
isomorphic dense subset) and they are homogeneous, i.e. if $B$ is
a Cohen algebra and $a \in B^+$ then $B\upharpoonright a = \{x \in B : x \leq a \}$
is isomorphic to $B$.

\begin{example}
 For any perfect Polish space $S$ algebras $RO(S)$ and
Borel$(S) / \mathcal M$ are Cohen algebras.
\end{example}

Cohen algebra is $\sigma$-centred by the definition, since it
contains a countable dense set. Hence Cohen algebra is $ccc$.

\subsubsection{${}$ \hspace{-1em}Absoluteness of cardinals.}

First task concerning any forcing notion is whether it preserves
cardinal numbers and cofinalities, i.e. whether cardinal numbers
of the groundmodel are cardinal numbers in extension.

\begin{proposition}\label{ABScardinals}
 Any $ccc$ forcing preserves cardinals and cofinalities.
\end{proposition}

This means that in any generic extension by $ccc$ forcing the
cardinal numbers are the same as in groundmodel and also
all cofinalities are preserved.

In what follow we will introduce the proof of the Proposition \ref{ABScardinals}.

What does it means to destroy cofinality or cardinal. It means
that in the extension there is a new mapping from $\lambda$ a groundmodel
cardinal $\kappa > \lambda$, which is onto, i.e. we add a new semiset
$\rho \subset \kappa \times \lambda$. Let us first show generally
the possible form of a name for such a mapping.

Let $B$ be a complete Boolean algebra, $G$ generic filter on $B$
and $r$ a relation. We can assume that dom$(r) \subseteq B$. Fix
a set $X \supset \mbox{rng}(r)$. Define a mapping
$c: X \to B$ determined by $r$ as follows
$$
c(x) \ = \ \bigvee \{ b \in \mbox{dom}(r) : \< b,x \> \in r \},
$$
for any $x \in X$. Function $c$ is somewhat generalised characteristic
function.

It easily follows from the properties of generic filter that
$r''G = (c^{-1})''G = \{ x \in X : c(x) \in G \}$, so the mapping
$c : X \to B$ is the name for a set $r''G$.

\smallskip

Let us assume that $\kappa,\lambda$ are cardinals in $V$ and
$\rho \in V[G]$ is a mapping $\rho : \kappa \to \lambda$.
Then $\rho \subset \kappa \times \lambda$ is a semiset and
so there is a relation $r \subset B \times (\kappa \times \lambda) \in V$
such that $r''G = \rho$.

Let $c : (\kappa \times \lambda) \to B $ be a generalised characteristic function
corresponding to $r$. So we have $\{ c(\alpha,\beta) : \alpha \in \kappa,\ \beta \in \lambda \}$,
but generally this is not a matrix, see \ref{}. We can modify each
row. Put
$a(\alpha,\beta) = c(\alpha,\beta) - \bigvee_{\gamma < \beta} c(\alpha,\gamma)$,
now rows are disjoint and if we add to say $a(\alpha,0)$ complement of
$\bigvee_{\beta < \lambda} c(\alpha,\beta)$, then the resulted family
$A = \{a(\alpha,\beta): \alpha \in \kappa,\ \beta \in \lambda \}$ is a matrix, see \ref{}.

Fix $\alpha \in \kappa$, among $\{ c(\alpha,\beta) : \beta \in \lambda \}$ there
is exactly one $\beta$ such that $c(\alpha,\beta) \in G$, since $\rho$ is
a mapping with dom$(\rho) = \kappa$. By the construction for the very same $\beta$
it holds that $a(\alpha,\beta) \in G$ and so $\rho = \{\< \alpha, \beta :
 a(\alpha,\beta) \in G \> \}$. Hence the matrix $A$ is a name for the mapping $\rho$.

\begin{proof}{\scshape of Proposition \ref{ABScardinals}.}
 Now assume $\omega \leq \kappa < \lambda$, and let matrix $A$ be a name
for the mapping $\rho: \kappa \to \lambda$.  Since $B$ satisfies $ccc$
 at most countably many elements are non-zero. That is for each row $\alpha$
there is some $\beta$ such that $a(\alpha,\gamma) = \mathbf 0$, for each $\gamma > \beta$. Since
$\kappa \times \omega < \lambda$, there is some $\beta < \lambda$ such that
each column in $A$ consists from zero elements. That means that the mapping
$\rho$ cannot by onto $\lambda$, its range cannot contain such $\beta$.
\end{proof}

Similarly one can show, that the cofinalities are preserved by $ccc$ forcing.
The following theorem is a straight forward generalisation of Proposition \ref{ABScardinals}.

\begin{theorem}
 Any $\kappa-cc$ forcing preserves all cardinal numbers $\lambda$, $\lambda \geq \kappa$.
\end{theorem}

It is convenient to know, that the least infinite cardinal $\kappa$ such that
a forcing $P$ is $\kappa-cc$ has to be regular cardinal.


\subsection{${}$ \hspace{-1em}Conclusion.}

Any real in an arbitrary generic extension via some forcing $P$
can be added by generic extension over a subalgebra $C \subset RO(P)$
with at most countable many complete generators.

\begin{proof}
 Let $B = RO(P)$, $G$ be a generic filter on $P$ over $V$. $G$ induces
generic filter $G^*$ on $B$.

 If $\tau \in V[G]$ is a real, then its matrix name
$A = \{ a(m,n) \in B : m,n \in \omega \}$ generate a complete subalgebra
$C \subseteq B$ and $G^* \cap C$ is a generic filter on
$C$ so that $\tau \in V[G^* \cap C] \subseteq V[G]$.
\end{proof}


In the following part we show that Cohen forcing
\begin{itemize}
 \item[(a)] adds a new real,
 \item[(b)] adds a splitting set,
 \item[(c)] adds unbounded real,
 \item[(d)] does not add an eventually different real, hence cannot add dominating reals.
\end{itemize}


Fix a generic filter $G$ on an ordering $\mathcal C = (Seq(2),\subseteq)$, then
for $s,t \in G$ is either $s \subset t$ or $s \supset t$.

\subsubsection{Cohen real.}
\begin{eqnarray*}
 \sigma_c & = & \bigcup \{s \in Seq(2) : \< s \> \in G \}, \\
 \rho_c & = & \{ i \in \omega : (\exists s \in Seq(2) ) \ \mbox{such that} \ s(i) = 1 \ \& \ \< s \> \in G \}.
\end{eqnarray*}


\begin{lemma}
 \begin{itemize}
  \item[(i)] $\sigma_r, \rho_r \in V[G]$,
  \item[(ii)] $\sigma_r : \omega \to \{ 0, 1 \}$,
  \item[(iii)] $\rho_r \subset \omega$ is infinite and splitting set.
 \end{itemize}
\end{lemma}

\begin{proof}
\noindent{\itshape (i)} Define relations
$r_1 \subseteq  Seq(2) \times (\omega \times \{0,1\})$ and
 $r_2 \subseteq Seq(2) \times \omega$.
\begin{eqnarray*}
 r_1 \ & = & \ \{ \< s, \< i, s(i) \>  \>  \ : \ s \in Seq(2) \ \& \ i < \mbox{dom}(s) \ \},  \\
 r_2 \ & = & \ \{ \< s, i \>  \ : \ s \in Seq(2) \ \& \ i < \mbox{dom}(s) \ \& \ s(i) = 1 \ \}
\end{eqnarray*}
Then $\sigma_r = r_1'' G$ and $\rho_r = r_2'' G$. Moreover $\sigma_r$ is a standard
characteristic function of $\rho_r$.

\smallskip

\noindent{\itshape (ii)} Let $k \in \omega$ and put $H_k = \{s \in Seq(2) : k < |s| \}$.
The set $H_k$ is dense and so $\sigma_r : \omega \to \{0,1\}$, i.e. dom$(\sigma_r) = \omega$.

\smallskip

\noindent{\itshape (iii)} Let $X \in [\omega]^\omega$. Since $X$ is infinite and $X \in V$,
following sets are dense
$$
H_j^k \ = \ \{\ s \in Seq(2) \ : \ (\exists i \in X-k) \ (i < \mbox{dom}(s)) \ \& \  s(i) = j \ \}.
$$
It follows that both $X\cap \rho_r$ and $X - \rho_r$ are both infinite.
\end{proof}

\begin{lemma}
 Cohen forcing adds unbounded real.
\end{lemma}

\begin{proof}
 The lemma claims that there is $\tau : \omega \to \omega$ in the
 extension over Cohen forcing so that $(\exists^\infty i)$ $\tau(i) > f(i)$,
for each function $f:\omega \to \omega$ from the groundmodel.

 Here we profit from the equivalent ordering $\mathcal C = (Seq,\subseteq)$.
 Let $G$ be a generic filter, let $\tau = \bigcup \{t \in Seq : t \in G \}$.
Then $\tau \in V[G]$ and by similar arguments as in previous it is a
function from $\omega$ to $\omega$.

 For any $f \in {}^\omega \omega \cap V$ and $k \in \omega$ let
$$
H_f^k \ = \ \{\ s \in Seq \ : \ (\exists i > k) \ (i < \mbox{dom}(s)) \ \& \  s(i) > f(i) \ \}.
$$
Each set $H_f^k$ is dense in $\mathcal C$, which guarantees that $\tau$
is an unbounded function.
\end{proof}



\begin{lemma}
 For any $\sigma \in V[G]$ $\sigma : \omega \to \omega$ there is $f \in {}^\omega \omega \cap V$
 such that $\exists^\infty i \in \omega \ \sigma(i) = f(i)$.
\end{lemma}

\begin{proof}
 Take a Cohen algebra $B$ and let $G$ be a generic filter on $B$ over $V$.
 Fix a countable dense subset $H \subset B^+$ and fix an enumeration
$H = \{h(n) : n \in \omega \}$. Let $\sigma \in V[G]$ $\sigma : \omega \to \omega$,
then there is a matrix name for $\sigma$ in $V$, i.e. a matrix
$A = \< a(m,n) \in B : m,n \in \omega \>$. Define $f:\omega \to \omega$
$$
f(m) = \min \{n \in \omega : a(m,n) \cap h(m) \not = \mathbf 0\}.
$$
Such mapping is well defined on $\omega$ since each $h(m) \not = \mathbf 0$
and $\bigvee_n a(m,n) = \mathbf 1$.

\smallskip

\noindent{\scshape Claim.} For any $k \in \omega$ $\bigvee_{m \geq k} a(m,f(m)) = \mathbf 1$.

\smallskip

Suppose not, then $b = \mathbf 1 - \bigvee_{m \geq k} a(m,f(m)) \not = \mathbf 0$
and since $B$ is atomless, there is infinitely many $m \in \omega$ so that
$h(m) \leq b$. But for such $m$ we get $a(m,f(m))$ is compatible with $b$, which
is in contradiction with the definition of $f$.

We get that for each $k \in \omega$ there is $m \geq k$ such that $a(m,f(m)) \in G$
and so $\tau(m) = f(m)$ for infinitely many $m$'s.
\end{proof}


\subsection{${}$ \hspace{-1em}Violating Continuum Hypothesis.}



% \begin{definition}{\bf Cohen forcing.}
%  Let $C=\bigcup{}^n2$ with ordering by extension, i.e $f\leq g$ iff $f\supseteq g$. The \emph{Cohen algebra} is the complete Boolean algebra $RO(C)$.
%  A standard argument shows that each countable atomless forcing notion is equivalent to $C$. For example, consider the cantor space $2^\omega$.
%  For any $f\in C$ we can define a corresponding clopen set $[f]=\{h\in 2^\omega:f\subseteq h\}$ in $2^\omega$. Then ${\mathcal B}=(\{[f]:f\in C\},\subseteq)$ is
%  isomorphic to $(C,\leq)$ and ${\mathcal B}$ is a basis of the Cantor space so $RO(2^\omega)\approx RO(C)$. It is known that $RO(2^\omega)$ is the completion
%  of the \emph{cantor algebra} $Clop(2^\omega)$. Moreover $RO(2^\omega)\approx \mbox{Borel}(2^\omega)/{\mathcal M}$ where $\mbox{Borel}(2^\omega)$ is the $\sigma$-field
%  of borel subsets of $2^\omega$ and ${\mathcal M}$ is the $\sigma$-ideal of meagre subsets of $2^\omega$.
%
%  Cohen forcing notion can also be described using the language of trees: For $f\in C$ define $T_f=\{h\in C:h\subseteq f\ \vee f\subseteq h\}$. Then $T_f$ is
%  a binary tree with stem $f$ where each node above the stem is splitting. Then $(C,\leq)$ is isomorphic to $(\{T_f:f\in C\},\subseteq)$.
% \end{definition}

%%%%%%%%%%%%%%%%%%%%% RANDOM %%%%%%%%%%%%%%%%%%%%%
\subsection{${}$ \hspace{-1em}Random forcing.}

One of the equivalents of Cohen forcing was $(\mbox{Borel}(2^\omega) - Meagre(2^\omega),\subseteq)$.
Meagre sets are negligible sets from topological point of view. From measure theory point of
view, negligible sets are those of measure zero. We consider standard probability
$\sigma$-additive measure defined on a $\sigma$-field Borel$(\mathcal C)$, which uniquely extends
function $m : \mathcal B_\mathcal C \to [0,1], \ m(\< s \>) = \frac{1}{2^{|s|}}$ and denote it
by $m$. Such measure is continuous, see \cite{Ke:1994}, i.e. all singletons are
of measure zero. Let $Null$ denote the $\sigma$-ideal of sets of measure zero.

\begin{definition}{\bf Standard definition.}
\begin{itemize}
 \item[(i)] $(\mbox{Borel}(\mathcal C) - Null,\subseteq)$ is \emph{Random forcing}. The
	ordering is not separative, its separative quotient is
 \item[(ii)] $(\mbox{Borel}(\mathcal C) / Null,\subseteq)$. This is $ccc$ complete Boolean
	algebra that carries strictly positive $\sigma$-additive continuous measure,
	$m[u] = m(U)$, for each $U \in \mbox{Borel}(\mathcal C)$.
\end{itemize}
\end{definition}

\begin{corollary}
 \begin{itemize}
  \item[(i)] The set $\{U \subset \mathcal C : U \ \mbox{closed and} \ m(U) > 0 \}$ is dense in Random forcing.
  \item[(ii)] The set $\{W \subset \mathcal C : W \ \mbox{open and} \ m(W) = 0 \}$ is a base of the Null ideal.
 \end{itemize}
\end{corollary}

\subsubsection{Measurable algebra as a completely metrizable space.}

Let $B$ be a measure algebra with a measure $\mu$. Put
$d_\mu(a,b) = \mu(a \vartriangle b)$, for $a,b \in B$. The proof
of the following Proposition is straightforward.

\begin{proposition}
 \begin{itemize}
  \item[(i)] $d_\mu$ is a metric on $B$ induced by $\mu$ and
  \item[(ii)] $(B,d_\mu)$ is a complete metric space.
 \end{itemize}
\end{proposition}

Hence instead of Cantor space $\mathcal C$ one can consider any Polish space $S$
and instead of $(\mbox{Borel}(\mathcal C) / Null,\subseteq)$ consider
appropriate $(\mbox{Borel}(S) / Null,\subseteq)$. We denote this measure
algebra $\mathbf B(\omega)$.

\subsubsection{Characterisation of $\mathbf B(\omega)$.}

$\mathbf B(\omega)$ is a measure algebra, which is
\begin{itemize}
 \item[(a)] atomless and $g(\mathbf B(\omega)) = \omega$, i.e. has
	countably many complete generators, equivalently
 \item[(b)] atomless and metric space $(\mathbf B(\omega),d_/mu)$
	is separable, so Polish space.
\end{itemize}


Fix a generic filter $G$ on $\mbox{Borel}(\mathcal C) - Null$ over $V$.

\subsubsection{Random real.}
\begin{eqnarray*}
 \sigma_r & = & \bigcup \{s \in Seq(2) : \< s \> \in G \}, \\
 \rho_r & = & \{ i \in \omega : (\exists s \in Seq(2) ) \ \mbox{such that} \ s(i) = 1 \ \& \ \< s \> \in G \}.
\end{eqnarray*}
Formally the very same definition as for Cohen reals, but the result will be
completely different. In the definition of $\sigma_r$ there are only generators
of $\mathbf B(\omega)$, while in $\sigma_c$ it was elements of the dense subset
of $\mathbf C(\omega)$.

\begin{lemma}
 \begin{itemize}
  \item[(i)] $\sigma_r, \rho_r \in V[G]$,
  \item[(ii)] $\sigma_r : \omega \to \{ 0, 1 \}$,
  \item[(iii)] $\rho_r \subset \omega$ is infinite and splitting set.
 \end{itemize}
\end{lemma}


\begin{proof}
 \noindent{\itshape (i)} Same as a proof for Cohen reals.

 \noindent{\itshape (ii)} follows from the {\itshape (iii)}.

 \noindent{\itshape (iii)} For any $n \in \omega$ the set
 $\{ \< s \> : s \in {}^n \{0,1\} \}$ is a maximal antichain, so
there is exactly one $s \in {}^n \{0,1\}$ such that  $\< s \>  \in G$.
Thus $\sigma : \omega \to \{0,1\}$.

Let $X \subset \omega$ be an infinite set. Assume that $X \cap \rho_r \subset k$,
for some $k \in \omega$. It means that for every finite $Y \subset X - k$,
the clopen set $O_Y = \{ f \in \mathcal C : (\forall i \in Y) \ f(i) = 0 \}$
belongs to G, since there is $s \in Seq(2)$, $\< s \> \in G$ and
$Y \subset \mbox{dom}(s)$, so $O_Y \supset \<s\>$. Now $m(O_Y) = \frac{1}{2^|Y|}$,
thus $Z = \bigcap \{ O_Y : Y \in [X - k]^\omega \} \in Null$.
This is a contradiction with set-completeness of $G$ in the algebra
$\mbox{Borel}(\mathcal C) / Null$ since $[Z] = \bigwedge \{[O_Y] : Y \in [X - k]^\omega \}$.

We proved that $(\forall k \in \omega)$ there is $\< s \> \in G$ so that
there exists $i \in X -k$ such that $s(i) = 1$.

Similarly we show that $(\forall k \in \omega)$ there is $\< s \> \in G$ so that
there exists $i \in X -k$ such that $s(i) = 0$. It follows that $\rho_r$ is
infinite and splitting set in $V[G]$.
\end{proof}


\subsubsection{Interrelationship of $\mathcal C$ and $\mathcal N$.}

Denote $I_r = \{ f \in \mathcal C : |f^{-1}\{1\}| = \omega\}$.
Since there are only countably many
elements $f \in \mathcal C$ with finitely many $1$'s
$m(I_r) = 1$. The set $I_r$ as a subspace of $\mathcal C$ is
homeomorphic to $\mathcal N$.

If $f \in I_r$, we can partition $f$ to blocks of zeroes, where
the block delimiters are $1$'s and put $\varphi(f) \in \mathcal N$
be a function, where $\varphi (f)(n)$ is a number of zeroes in the
$n$-th block.

$\varphi$ is one-to-one mapping of $I_r$ onto $\mathcal N$ and
it maps a measure $m$ onto a measure $\bar m$ on the space $\mathcal N$.

We know that random real $\sigma_r$ contains infinitely many $1$'s, so
$\sigma_r \in I_r$ and $\varphi(\sigma_r) : \omega \to \omega$ in $V[G]$.
Such mapping $\varphi(\sigma_r)$ is eventually different real.
% \begin{definition}
% A forcing notion $P$ is weakly distributive if for every countable family $\{A_n:n<\omega\}$ of maximal antichains in $P$, there is
%  a maximal antichain $A$ such that for every $a\in A$ and for every $n<\omega$, $a$ is compatible with only finitely many members of $A_n$.
% \end{definition}
%

\begin{lemma}
 Random forcing adds eventually different real.
\end{lemma}

\begin{proof}

\end{proof}

%%%%%%%%%%%%%%%%%%%%% HECHLER %%%%%%%%%%%%%%%%%%%%%
\subsection{${}$ \hspace{-1em}Hechler forcing.}

Let $p \subset {}^{<\omega} \omega$ be $\omega$-ary
tree. A tree is called \emph{Hechler tree} if
\begin{itemize}
 \item[(i)] it has a stem $s \in {}^{<\omega} \omega$ and
 \item[(ii)] $(\forall t \in p) \ (s \subset t \to split_p(t) =
	\{i \in \omega : t^\smallfrown i \in p \})$
	is a cofinal set.
\end{itemize}
\emph{Hechler forcing}, denoted by $H$, consists of all Hechler trees
ordered by inclusion.

\begin{fact}
 $(H,\subseteq)$ is a separative partial ordering that is $\sigma$-centred,
so $ccc$.
\end{fact}

\begin{proof}
 Hechler trees with the common stem form a centred family and there is
only finitely many different stems.
\end{proof}

It is quite frequent in literature, that by the Hechler forcing
is meant the following

\begin{definition}
\emph{Hechler forcing} is a set
 $H_0 = \{\<s,f\> : s \in {}^{<\omega} \omega,\ f : \omega \to \omega,\ s \subset f \}$, with
partial ordering
$\<s,f\> \leq \<t,g\>$ if and only if $t \subseteq s \ \& \ (\forall n \in \omega) \ f(n) \geq g(n)$.
\end{definition}

Forcings $H$ and $H_0$ are different but they have very similar
properties. By our opinion it is easier to work with $H$ instead of
$H_0$.

\begin{proposition}
 $(H_0,\leq)$ is a partial ordering that can be regularly embedded
into $(H,\subseteq)$, i.e. $H_0 \lessdot H$.
\end{proposition}

\begin{proof}
 Let $\<s,f\>$ be a condition from $H_0$.
 Define $\omega$-ary tree $p(s,f) \in H$ such that
 $t \in p(s,f)$ iff $t \subseteq s$ or $(\ t \supseteq s$ and
$(\forall i \in \dom(t))\ (i \geq |s| \to t(i) \geq f(i)) \ )$.
 $p(s,f)$ is Hechler tree with stem $s$ and with uniformly
 determined splitting sets, i.e. $split(t) = \omega - f(n)$ for
each $t \in {}^n \omega$, $n \geq |s|$. It is easy to check
that the mapping $\<s,f\> \mapsto p(s,f)$ is isomorphism
of $(H_0,\leq)$ onto $(\{ p(s,f)\},\subseteq)$ and it is
regular embedding.
\end{proof}


\begin{lemma}
 Hechler forcing adds dominating reals and so it adds independent reals.
\end{lemma}

\begin{proof}
 Let $G$ be a generic filter on $H$ over $V$, then
$$
\tau = \bigcup \{ stem(p) : p \in G \}
$$
is a dominating real.Let $f : \omega \to \omega$ be a function from groundmodel.
For $p \in H$ consider $p' \subset p$, where
$$
p' \ = \ \{\ s \in p \ : \ s \subset stem(p) \ \mbox{or} \ (stem(p) \subset p \ \& \ (\forall i \geq |s|) \ p'(i) > f(i) \ )  \}.
$$
The set $\{p':p \in H \}$ is dense in $H$ thus there is some $p' \in G$. We immediately
get that $\tau(i) > f(i)$ for each $i \geq stem(p')$.

\smallskip

If there is a dominating real in the extension, then there is also an independent
real. It is enough to consider a partition of $\omega$ into intervals
$I_0 = [0,a_0) \cup I_1 = [a_0,a_1) \cup I_2 = [a_1,a_2), \cup \dots$ so that
$X \cap I_k \cup I_{k+1} \not = \emptyset$. Then $\bigcup_k I_{4k} \cup I_{4k+1}$
is a splitting set.
\end{proof}

\begin{fact}
 Hechler forcing adds a Cohen real.
\end{fact}

\begin{proof}
 We use Criterion 2, to the mapping $h : H \to \mathbf C$,
 $h(p) : n \to \{ 0,1 \}$ and $h(i) = s(i)_{mod(2)}$, for each $i < n = |s|$,
 where  $s$ is a stem of $p$.
\end{proof}



\subsubsection{Luzin sets.}

Interesting properties of Hechler forcing concerning Luzin sets.

\begin{definition}
 A set $L \subset \mathcal C$ is called \emph{Luzin set}
(or $(2^{\aleph_0},\omega_1)$-Luzin set), if $|L| = 2^\omega$
and $|L\cap M| < \omega_1$ for each meagre set $M$.
\end{definition}

We have to omit a proof of the following theorem.

\begin{theorem}
 Let $G$ be a generic filter on $H$ over $V$. Then there is
a Luzin set in the extension $V[G]$.
\end{theorem}

Now look at what happen to Cihon diagram (see \ref{cihon}) by
adding one Hechler real.

There are just two values for cardinal invariant in diagram,
namely $\omega_1$ and $2^\omega$ and those values are independent
on the values in the groundmodel.

\begin{itemize}
 \item[] $\non (\mathcal M) = \omega_1$, since subset
	$X \subset L$ of Luzin set of size $\omega_1$ cannot be meagre. On the
	other hand
 \item[] $\cov (\mathcal M) = 2^\omega$, i.e. is of full size,
	since we have to also cover a Luzin set $L$ and each meagre
	set covers only a countable part of $L$.
\end{itemize}

\subsubsection{Complete Boolean algebra.}

We focus on the description of the complete Boolean
algebra $RO(H)$ determined by the Hechler forcing $H$.

We work in the space $\mathcal N$. For $p \in H$, $[p]$ is
the set of all branches of tree $p$ and it is closed subset
of $\mathcal N$. Such sets satisfy the crucial property
to form a topology base:
$$
(\forall f \in [p_1] \cap [p_2]) \ (\exists q \in H) \ f \in [q] \subset [p_1] \cap [p_2].
$$
We denote $\tau_H$ the topology on ${}^\omega \omega$ given by this open base.
Topology $\tau_H$ extends standard topology of $\mathcal N$.

\begin{claim}
 The topological space $({}^\omega \omega, \tau_H)$ satisfies
 the Baire category theorem, i.e. any nonempty open set is
 not meagre.
\end{claim}

\begin{proof}
 We show that base set $[p]$ cannot be covered by meagre set,
 i.e. the set $[p] \cap \bigcap_{n \in \omega} \bigcup\{[q] : q \in P_n\}$
 is not empty for each countable collection of open dense sets
 $\{P_n: n \in \omega\}$
 Let $p \in H$, let $Q_n$ be arbitrary open dense
 (in the sense of ordering) set in $H$, for each $n \in \omega$; note
 that the set $\{ [q] : q \in Q_n\}$ is dense in $\tau_H$. Since
 $Q_0$ is dense, then there is a Hechler tree $q_0 \leq p$,
 $q_O \in Q_0$ and since $Q_0$ is open, one can assume, that
 $|stem q_0| >  0$. We proceed by induction and for $n+1 \in \omega$
 we fix $q_{n+1} \leq q_n$, $q_{n+1} \in Q_{n+1}$ and
 $|stem q_{n+1}| >  n+1$. The branch $f = \bigcup_n stem q_n$ is
 in $[p] \cap \bigcap_{n \in \omega} \bigcup\{[q] : q \in Q_n\}$,
 which completes the proof.
\end{proof}



\begin{corollary}
We immediately obtain that $Borel({}^\omega \omega, \tau_H) = Borel(\mathcal N)$
so
 $$
RO(H) \approx Borel(N) / Meagre(\tau_H) \approx BP(\tau_H) / Meagre (\tau_H).
 $$
\end{corollary}








%%%%%%%%%%%%%%%%%%%%% PROSTRELOVANI %%%%%%%%%%%%%%%%%%%%%
\subsection{${}$ \hspace{-1em}Adding pseudointersection to a filter.}


\newpage
%%%%%%%%%%%%%%%%%%%%% non CCC %%%%%%%%%%%%%%%%%%%%%


The following forcing notions does not satisfy $ccc$, in fact are
$(2^\omega)^+$-$cc$. They collapse the  cardinal $2^\omega$ in
groundmodel to smaller cardinal depending on the other properties
of groundmodel, but the first uncountable cardinal $(\omega_1)^V$
is preserved by all of them.


%%%%%%%%%%%%%%%%%%%%% MATHIAS %%%%%%%%%%%%%%%%%%%%%
\subsection{${}$ \hspace{-1em}Mathias forcing.}

This ordering apart from being used to produce interesting
generic extension, influenced a development of infinite
Ramsey theory.

Conditions in Mathias forcing have the same structure as
conditions in $P(F)$, see \ref{}. Instead of a filter $F$
we consider all infinite subsets of $\omega$.

\subsubsection{Standard definition.} \emph{Mathias forcing} consists
of the sets
$$
M \ = \ \{ \< s,A \> : s \in [\omega]^{<\omega}, \ A \in [\omega]^\omega \ \&
 \max s < \min A \ \}, \ \mbox{ordered by}
$$
$\<s,A\> \leq \< t, B\>$ iff $s \supseteq t, \ A \subseteq B$ and $s - t \subseteq B$.

\begin{fact}
 $(M,\subseteq)$ is a separative partial ordering of size $2^\omega$
 with the largest element  $\<\emptyset, \omega$, satisfying $(2^\omega)^+$-$cc$.
\end{fact}

\begin{proposition}
 Non separative partial ordering $([\omega]^\omega, \subseteq)$ can be
 regularly embedded into $(M,\subseteq)$.
\begin{eqnarray*}
 ([\omega]^\omega, \subseteq) \ & \hooklongrightarrow & \ (M,\subseteq) \\
 			A     \ & \longmapsto	  & \ \< \emptyset, A\>.
\end{eqnarray*}
\end{proposition}

\subsection{${}$ \hspace{-1em}Isomorphic representation of Mathias forcing
and infinite Ramsey theory.} Any condition $\< s,A \> \in M$ determines
so called \emph{Mathias set}
$$
E(s,A) \ = \ \{ X \in [\omega]^\omega \ : \ s \sqsubset X \ \& \ X-s \subset A \}.
$$

Notice that if  $S \subseteqq [\omega]^\omega$, then $S$ is a Mathias set
if \begin{itemize}
    \item[] $s = \bigcap \{X : X \in S \}$ is finite and
    \item[] $\{ X - s : X \in S \} \ = \ [A]^\omega$ for some
	infinite set $A$.
   \end{itemize}

\begin{fact}
 $(M,\subseteq)$ is isomorphic to $\{\< s,A\> : \<s,A\> \in M \}$ ordered
by the usual inclusion.
\end{fact}

\subsection{${}$Ellentuck topology.}

When we speak about $\mathcal P(\omega)$ as a topological space,
we usually mean the topology of Cantor discontinuum obtained
via identification of subset of $\omega$ with their characteristic
functions. $[\omega]^\omega \subseteq \mathcal P(\omega)$,
hence we have an inherited topology and denote it topology $\tau_c$.
Notice that $([\omega]^\omega, \tau_c )$ is completely metrizable
space, since $[\omega]^\omega$ is $G_\delta$ subset of $\mathcal P(\omega)$.


Ellentuck topology extends $\tau_c$ and its open  base is formed
by all Mathias sets. We denote this topologycal space by $\E$.

\begin{proposition}
 \begin{itemize}
  \item[(i)] $\E$ is well defined topological space,
  \item[(ii)] $Borel(\E) = Borel([\omega]^\omega,\tau_c)$ and
  \item[(iii)] $\E$ satisfies the Baire category theorem.
 \end{itemize}
\end{proposition}

\begin{proof}{\scshape Hint.}
\begin{itemize}
 \item[(i)] If $x \in S_1 \cap S_2$ for some Mathias sets $S_1$ and $S_2$
	then there is a Mathias set $S$ such that $x \in S \subseteq S_1 \cap S_2$.
 \item[(ii)] Follows from the fact that any Mathias set is closed in
	$\tau_c$ and for each $t \in {}^n \{0,1\}, \ [t] \cap [\omega]^\omega$
	is a Mathias set, namely $\<t^{-1}\{1\},\omega - n\>$.
 \item[(iii)] similar argument as for the Hechler forcing, see \ref{}.
\end{itemize}
\end{proof}

Now, it is easy to describe complete Boolean algebra that is determined
by Mathias forcing. All the following algebras are isomorphic:
$$
RO(M) \simeq RO(\E) \simeq Compl\ \bigl( Borel(\E) / Meagre(\E) \bigr ),
$$
where $RO(M)$ is complete Boolean algebra with dense set isomorphic to
$(M,\leq)$ and $RO(\E)$ is the algebra of regular open sets of the space $\E$
ordered by $\subseteq$. Since the algebra $Borel(\E) / Meagre(\E)$ is
only $\sigma$-complete, we have to take its completion here.


\subsection{${}$Infinite Ramsey theory.}
The well known classical Ramsey theorem stated in arrow notation
$$
(\forall k > 0) \ \omega \longrightarrow (\omega)^k_2,
$$
has a natural generalisation $\omega \rightarrow (\omega)^\omega_2$
that does not hold in ZFC. In other words, there is a colouring
$c: [\omega]^\omega \rightarrow \{0,1\}$ such that for each
infinite subset $A \subset \omega$ there are $A_1,A_2 \in [A]^\omega$
of different colour, $c(A_1) \not = c(A_2)$.

\smallskip

Each subset $S \subset [\omega]^\omega$ determines a colouring
$c_S:[\omega]^\omega \rightarrow \{0,1\}$ so that $c(X)=1$
iff $X \in S$.

\begin{definition}
 \begin{itemize}
  \item[(i)] $S \subset [\omega]^\omega$ is called \emph{Ramsey}
	if for each infinite $A \subset \omega$ there is its
	infinite subset $B \in [A]^\omega$ so that
	either $[B]^\omega \subset S$, or $[B]^\omega \cap S = \emptyset$.
  \item[(ii)]  $S \subset [\omega]^\omega$ is called \emph{completely Ramsey}
	if for each Mathias condition $\<s,A\>$ there is $B \in [A]^\omega$
	so that either $\dl s,B \dr \subset S$, or $\dl s,B \dr \cap S = \emptyset$.
 \end{itemize}
\end{definition}

We state here without proof.

\begin{theorem}\label{ellentuck}{\rm (Ellentuck Theorem)}
Subsets of the space $\E$ has the Baire property if and only if
it is completely Ramsey.
\end{theorem}

The following theorem and techniques used in its proof substantially influenced the
development of infinite Ramsey theory. Now it is an easy corollary
to more general Ellentuck theorem.

\begin{theorem}\label{galvin-prikry}{\rm (Galvin - Prikry Theorem)}
Each Borel set in classical topology $\tau_c$ is completely Ramsey.
\end{theorem}

We later use this theorem in a proof of a Laver property of Mathias forcing.

\subsection{${}$Historical note.}
For any topological space $(X,\tau)$ we can consider the space
of its non-empty closed subsets $Closed^+(\tau)$. Vietoris defined
natural topology on this space, now known as Vietoris topology.
Reader familiar with this concept may note that when starting
from the discrete space $\omega$ the Vietoris topology is a topology
on $\mathcal P(\omega)$ and its trace on $[\omega]^\omega$ is
exactly the Ellentuck topology.


\subsection{${}$Back to Mathias forcing.}\label{M-order}

We will define an   auxiliary sequence $\< \leq_n : n \in \omega \>$
of orderings on $M$ by $\< s,A \> \leq_n \<t,B\>$ if $s=t$, $A \subset B$
and first $n$ elements of $B$ are in $A$.

\smallskip

It is clear that
$$
\leq \ \supseteq \ \leq_0 \ \supseteq \ \leq_1 \ \supseteq \ \dots \ \supseteq \ \leq_n \ \supseteq \ \dots
$$
and if $\< s,A\> \leq_n \<t,B\>$ then both conditions have the same stem.

\smallskip

A sequence of conditions $\<s,A_0\> \geq_1 \<s,A_1\> \geq_2 \dots \geq_n \<s,A_n\>
	\geq_{n+1} \dots$ is called a \emph{fussion sequence} for $M$
and $p=\<s,\bigcap_i A_i\>$ is again a Mathias condition and
$p \leq_n \<s,A_n\>$, for each $n \in \omega$.

Our aim is to show that Mathias forcing with orderings $\leq_n (n \in \omega)$
satisfies the following general principle introduced by J.~Baumgartner.
At first sight it seems little odd, but it appears to be useful.

\begin{definition}
 {\bf Axiom A.} A forcing notion $(P,\leq)$ satisfies \emph{axiom A}
if there is a sequence $\< \leq_n, n \in \omega \>$ of orderings
of $P$ such that
\begin{itemize}
 \item[(i)] $\leq \ \supseteq \ \leq_0$ and $\leq_n \ \supseteq \ \leq_{n+1}$,
	i.e. $p \leq_{n+1} q$ implies that $p \leq_n q$ and also $p \leq q$,
 \item[(ii)] if $\<p_n : n \in \omega \>$ is a sequence of conditions
	such that $p_{n+1} \leq_n p_n$ then there is $p \in P$ such
	that $p \leq_n p+n$ for each $n \in \omega$,
 \item[(iii)] for any $p \in P$, any maximal antichain $A$ and $n \in \omega$
	there is $q \leq_n p$ and $q$ is compatible with at most countably
	many $a \in A$; $|\{a \in A : a || q \}|\leq\omega$.
\end{itemize}
\end{definition}

\begin{example}
 If forcing notion $(P,\leq)$ is $ccc$ or $\sigma$-closed then it
satisfies the axiom A. It suffices to put $\leq_n = \leq$ for
each $n \in \omega$. For $ccc$ forcing, there is nothing to check.

If $P$ is $\sigma$-closed then property {\itshape (ii)} follows
from '$\sigma$-closedness'.
\end{example}


\begin{proposition}
 Mathias forcing satisfies axiom A, with respect to orderings
 $\{\leq_n : n \in \omega \}$, see \ref{M-order}.
\end{proposition}

\begin{proof}
 Fussion argument guarantees the  property {\itshape (ii)} of axiom A.

\smallskip

To show {\itshape (iii)}, let $p = \<s,A\>$, $n\in\omega$ and
$\mathcal A$ be given antichain. Let $K_0$ be the set of first $n$-elements
of $A$ and $\{ s_i : i < 2^n \}$ be an enumeration of all
subsets of $K_0$, we may suppose that $s_0 = \emptyset$.

By recursion on $i < 2^n$ we construct non-increasing sequence
$\< S_i : i \in \omega\>$ of infinite subsets of $A$ and simultaneously
non-decreasing sequence $\<{\mathcal A'}_i : i < 2^n \>$ of finite subsets
of $\mathcal A$.

We start with condition $\<s,S\>$ where $S = A - K_0$, note that
$s \cup s_0 =s$.

Either there is a couple $r \in \mathcal A$ and $S_0 \in [S]^\omega$
such that $\< s, S_0 \>$ is below $r$. If so, choose one, say $r_0$, $S_0$
and set ${\mathcal A'}_0 = \{r_0\}$. So we have the set $S_0$.

Or such couple does not exist. Then put ${\mathcal A'}_0 = \emptyset$
and $S_0 = S$. Assume we know $S_k,\ {\mathcal A'}_k$ and $k+1 < 2^n$.
Now we test condition $\<s \cup s_{k+1}, S_k \>$. Again either
we can choose a couple $r_{k+1}, \ S_{k+1} \in [S_k]^\omega$
that fulfils $\< s \cup s_{k+1}, S_{k+1}\> \leq r_{k+1}$. Then
we have $S_{k+1}$ and ${\mathcal A'}_{k+1} = {\mathcal A'}_k \cup \{r_{k+1}\}$.
Or no such couple exists, then $S_{k+1} = S_k$ and
${\mathcal A'}_{k+1} = {\mathcal A'}_k$.

When $k = 2^{n-1}$, first step is finished and we denote $A_1=S_{2^{n-1}}$
and ${\mathcal A}_1 = {\mathcal A'}_{2^{n-1}}$. Notice that
$|{\mathcal A'}_1| \leq 2^n$.

So $p_1 = \<s , K_0 \cup A_1 \> \leq_n \<s,A\>$, but we do not know that
$p_1$ is compatible with only countably many conditions from $\mathcal A$.

We have to continue  other $\omega$-many steps. Take a set $K_1$
consisting of first $n+1$ elements of ${\mathcal A}_1$ and enumerate
$\{s_i : i < 2^{2n+1} \}$ of subsets of $K_0 \cup K_1$. We continue
with the same process, starting with condition $\<s,S\>$, where
$S = A_1 - k_1$. By recursion choose appropriate couples $r,S_i$,
i.e. $\<s \cup s_i , S_i \> \leq r$ provided it is possible and
enrich the starting set $\mathcal A_1$. Note that it is sufficient
to consider $s_i$ for which $s_i \cap K_1 \not = \emptyset$,
other cases are covered by $\mathcal A_1$.

We obtain a finite $\mathcal A_2 \supseteq \mathcal A_1$,
and $A_2 = S_{2^{2n}}$. Take $K_2$ first $n+1$ elements of
$A_2$ and continue up to $\omega$.

We get pairwise disjoint sets $K_j \ (j \in \omega)$, with $|K_0|=n$,
$|K_j|=n+1$, for $j > 0$, and each $K_j$ consists of first $n+1$
elements of $A_j$ for $j > 0$.

It remains to put $q = \< s, \bigcup_j K_j\>$ and it is clear that
$q \leq_n p$. We need to show that $q$ is compatible with at most
countably many  elements of $\mathcal A$. To this end, let us show
that for $r \in \mathcal A$, $r \not \in \bigcup_j \mathcal A_j$,
$q \perp r$.

For if $q || r$, then there is some $t$, $t \leq q$, $t \leq r$.
$t = \< s,B\>$, with finite $L$, so $s -L \subseteq \bigcup_{j < m_0} K_j$
for some $m_0 \in \omega$. But our construction guarantees that
$L$ is compatible with some condition from $\mathcal A_{m_0}$,
which contradicts the disjointness of $\mathcal A$.
\end{proof}

The following assertion holds true for any forcing notion satisfying axiom A,
the only specific but important for Mathias frocing  is the fact that
stems are preserved.

\begin{theorem}
 Let $\{\mathcal A_n : n \in \omega \}$ be a countable family of maximal antichains
in $M$. Then for any condition $p = \<s,A\>$ there is a condition
$q = \<s,B\>$ such that
\begin{itemize}
 \item[(i)] $q \leq p$ and
 \item[(ii)] $q$ is compatible with at most countably many elements of $\mathcal A_n$,
 for each $n \in \omega$.
\end{itemize}
\end{theorem}

\begin{proof}

\end{proof}

Mathias forcing has moreover a property, called \emph{Laver property}, that is not
generally a consequence of axiom A.

\begin{proposition}
 Let $B$ be a complete Boolean algebra in which Mathias ordering $(M,\leq)$ is dense,
 i.e. $B \approx RO(M)$. Let $\{\mathcal A_n : n \in \omega \}$ be a countable
family of maximal antichains in $B$. Then for any condition $p = \<s,A\>$ there is a stronger condition
 $q = \<s,B\>$ with the same stem such that $q$ is compatible with at most $2^n$ elements of $\mathcal A_n$,
for each $n \in \omega$. Particulary for any element $b \in B$ and any condition  $\<s,A\>$
there is a condition $\<s,B\> \leq \<s,A\>$ such that either $\<s,B\> \leq b$ or $\<s,B\> \perp b$.
\end{proposition}

\begin{proof}

\end{proof}

\subsection{${}$Different forms of distributivity laws.}\label{M-order}

Let $(P,\leq)$ be a forcing notion and $B$ be a complete Boolean algebra, $B = RP(P)$.

If $\{\mathcal A_i : i < n \}$ is a finitely many maximal antichains then there is
a maximal antichain $\mathcal A$ refining all $\mathcal A_i$'s, i.e.
for every $p \in \mathcal A$ and  for each $i < n$ there is $q \in \mathcal A_i$ so that $p \leq q$.

The common refinement for infinitely many antichains need not exist.

\begin{definition}
 Let $\kappa$ be an infinite  cardinal. An ordering $(P,\leq)$ is called
\emph{$\kappa$-distributive} if there is a common refinement for each collection of $\kappa$ many
maximal antichains.

\emph{Non distributivity number}, denoted as $\mathfrak h(P)$ is the minimal cardinal
such that $(P,\leq)$ is not $\kappa$-distributive.
\end{definition}

\begin{fact}
\begin{itemize}
 \item[(i)] $(P,\leq)$ is $\kappa$-distributive if and only if an intersection of at most $\kappa$ many open
 dense subsets is again an open dense subset.
 \item[(ii)] $B$ is $\kappa$ distributive iff some dense subset $(P,\leq)$ is $\kappa$ distributive.
 \item[(iii)] Non distributivity number is well defined provided that $(P,\leq)$ has no atoms, in this
	case $\mathfrak h(P) = \mathfrak h(RO(P))$ and it is regular cardinal.
 \item[(iv)] If $(P,\leq)$ is atomless and homogeneous, i.e. $(P,\leq)$ and $((leftarrow,p],\leq)$
	have isomorphic dense sets, then
	$
	\mathfrak h(P) \ = \ \min \{\kappa : (\exists \{\mathcal A_i : i < n \})$ maximal
	antichains such that for a dense subset $H \subseteq P \ (\forall p \in H) \
	(\exists \alpha < \kappa) \ (\exists q_1,q_2 \in \mathcal A_\alpha) \ q_1 \not = q_2 \ \&
	\ p \parallel q_1 \ \& \ p \parallel q_2 \ \}.
	$
\end{itemize}
\end{fact}

\begin{definition}{\bf Three parametric distributivity.}
\begin{itemize}
 \item[(i)] Let $\kappa$ be infinite, $\lambda,\mu \geq 2$ cardinals.
	We say that a complete Boolean algebra $B$ is
	\emph{$(\kappa,\lambda,\mu)$-distributive} if for any $\kappa$ many
	maximal antichains $\{\mathcal A_\alpha : \alpha < \kappa \}$ each
	of size at most $\lambda$ there is a dense set $H$ such that for any
	$p \in H$ and every $\alpha < \kappa$ $p$ is compatible with less then
	$\mu$ elements of each $\mathcal A_\alpha$.
 \item[(ii)] $(\kappa,\cdot,\mu)$ distributivity means that the antichains
	are not limited in size, i.e. $(\kappa,\cdot,2)$-distributivity is
	$\kappa$-distributivity.
 \item[(iii)] A Boolean algebra $B$ is \emph{weakly distributive} if it is
	$(\omega,\cdot,\omega)$-distributive.
 \item[(iv)] A forcing $(P,\leq)$ is called \emph{${}^\omega \omega$-bounding}
	if the algebra $RO(P)$ is $(\omega,\omega,\omega)$-distributive.
\end{itemize}
\end{definition}

Some of these notions we already encountered in Random forcing section. Measure
algebra is weakly distributive and Random forcing is ${}^\omega \omega$-bounding.

\begin{definition}
 Let $B$ be a complete Boolean algebra.
\begin{itemize}
 \item[(i)] $B$ has \emph{Laver property} if for any countable family
	$\{\mathcal A_n : n \in \omega \}$ of finite maximal antichains there is a dense
	subset $H \subseteq B^+$ such that
	$$
	(*) \qquad (\forall p \in H) \ (\forall n \in \omega) \quad
	|\{\ q \in \mathcal A_n \  : \ q \parallel  p \ \}| \ \leq \ 2^n.
	$$
  \item[(ii)] $B$ has \emph{Sacks property} if for any countable family
	$\{\mathcal A_n : n \in \omega \}$ of countable maximal antichains there is a dense
	subset $H \subseteq B^+$ such that $(*)$.

\end{itemize}
\end{definition}

Sack property is clearly stronger then Laver property, since maximal antichains need not
to be finite, but at most countable.

\subsubsection*{Influence of distributivity properties of forcing to extension.}

\begin{proposition}
 A forcing $(P,\leq)$ is $\kappa$-distributive if and only if in any generic
extension $V[G]$ via $P$ any mapping of form $\sigma : \kappa \rightarrow V$,
$\sigma \in V[G]$ is a set in groundmodel; i.e. $\sigma \in V$.
\end{proposition}

\begin{fact}
 Every forcing notion satisfying axiom A is $(\omega,\cdot,\omega)$-distributive.
\end{fact}

\begin{proposition}
 If forcing $(P,\leq)$ satisfies $(\omega,\cdot,\omega_1)$-distributivity, then
in any generic extension $V[G]$ via $P$ for any mapping $\sigma \in V[G]$,
$\sigma : \omega \rightarrow V$ there is a relation $S \in V$ such that
$S \subseteq \omega \times V$ and $|S(n)| \leq \omega$ for each $n \in \omega$
and $\sigma(n) \in S(n)$.

So $P$ preserves cardinal $\omega_1$.
\end{proposition}

\begin{proposition}
 If forcing $(P,\leq)$ has a Laver property then in any generic extension $V[G]$
via $P$ the following hold:
Let $\sigma : \omega \rightarrow \omega$, $\sigma \in V[G]$, if there is
$f \in V$ such that $f:\omega \rightarrow \omega$ and $\sigma \leq f$ then
there is $S : \omega \rightarrow [\omega]^{< \omega}$, $S \in V$ such that
$(\forall n \in \omega)$ $\sigma(n) \in S(n)$ and $|S(n)| \leq 2^n$.
\end{proposition}

The previous proposition holds for Sacks forcing in a bit stronger way,
to enclose the function $f$ from the extension into a relation $S$ we
do not need to check that $f$ is be bounded by the groundmodel function.

\smallskip

Important consequence of Laver property.

\begin{proposition}
 If forcing has the Laver property then it does not add neither Cohen nor Random reals.
\end{proposition}

\subsection{${}$Summary of properties of Mathias forcing.}

Mathias forcing
\begin{itemize}
 \item[(o)] is homogenous, ${2^\omega}^+-cc$ partial ordering.
 \item[(i)] adds a dominating real, so also adds an independet real.
 \item[(ii)] has laver property and so does not add neither Cohen
	nor Random reals.
 \item[(iii)] satisfies axiom A, so does not collapses $\omega_1$.
 \item[(iv)] collapses all cardinal $\kappa$ $\mathfrak h < \kappa \leq 2^\omega$
	onto $\mathfrak h$, provided $\mathfrak h < 2^\omega$ in $V$;
	i.e. there is a one-to-one mapping $\sigma: \mathfrak h \to (2^\omega)^V$
	and either smaller cardinals then $\mathfrak H$ or strictly
	larger cardinals then $2^\omega$ are preserved.
\end{itemize}

\begin{proof} {\scshape Hint.} For any  partition $\{I_n : n \in \omega \}$, to
intervals $I_n = [a_n , a_{n+1} )$ the set of conditions $\{ \< s, A \> : \forall^\infty n |I_n \cap s| \leq 1 \}$
is dense in Mathias forcing. So increasing enumeration of Mathias real is
a dominating real.

\smallskip

The fact that Mathias forcing collapses cardinal follows from that $(\omega,\subseteq^*) \lessdot M$
and that $(\omega,\subseteq^*)$ is a forcing for such collaps.

\end{proof}



The following two forcings do not add new reals. This means, among others,
that every Polish space in the groundmodel remains Polish in the extension.

\subsection{${}$ $([\omega]^\omega,\subseteq)$.}

The ordering $([\omega]^\omega,\subseteq)$ and its separative modification
$([\omega]^\omega,\subseteq^*)$ were mentioned in chapter \ref{ch2,kde}.

\smallskip

Relevant properties of $([\omega]^\omega,\subseteq^*)$.
\begin{itemize}
 \item[(i)] atomless, homogeneous, $(2^\omega)^+$-$cc$ and $\sigma$-closed.
 \item[(ii)] Cardinal number $\mathfrak h$ denotes the non-distributivity
	number of this forcing, it is one  of basic cardinal characteristics
	of the continuum.
 \item[(iii)] $\mathfrak h$ is the minimal cardinal $\kappa \in V$ so that
	this forcing adds a new subset of $\kappa$.
 \item[(iv)] Generaly this forcing adds a selective ultrafilter on $\omega$.
 \item[(v)] It adds a mapping from $\mathfrak h$ onto $(2^\omega)^V$, so
	it collapses all cardinals in $V$ that are less orequal to $2^\omega$
	and larger then $\mathfrak h$.
 \end{itemize}

Complete Boolean algebra $RO([\omega]^\omega,\subseteq^*)$ is the completion
of $\mathcal P(\omega) / fin$ and this is isomorphic to the algebra of regular
open sets of \v Cech-Stone reminder $\beta \omega - \omega$.

\medskip

Let us first recall the definitions of some types of ultrafilters on $\omega$.

\begin{definition}
 A non principal ultrafilter on $\omega$ is
\begin{itemize}
 \item[(i)] \emph{selective (Ramsey)} ultrafilter if for any partition
	$R$ of $\omega$ either
	\begin{itemize}
	\item[(a)] $R \cap U \not = \emptyset$ or
	\item[(b)] there is $X \in U$ such that $(\forall r \in R) \ |r \cap X| \leq 1$,
		i.e. $X$ is a selector of $R$.
	\end{itemize}
 \item[(ii)] \emph{$P$-ultrafilter} if for any partition $R$ of $\omega$
	either $R \cap U \not = \emptyset$  or there is $X \in U$  such that
	$(\forall r \ in R) \ |X \cap r | < \omega$.
 \item[(iii)] \emph{$Q$-ultrafilter} if for any partition $R$ of $\omega$
	into finite pieces there is $X \in U $ which is a selector for $R$.
\end{itemize}
\end{definition}

It is easy to see that an ultrafilter $U$ is selective if and only if it is
simultaneously $P$ and $Q$ ultrafilter.

\smallskip

It should be noted that the existence of any such ultrafilter is not provable
in ZFC. Selective ultrafilters exist e.g. when MA (Martin axiom) holds.


\begin{proof}
\noindent (i) Clear, (ii), (iii) follows from definition.

\smallskip

\noindent (iv) Let $U$ be a generic filter, $U$ is a filter on $\omega$
containing all co-finite sets. Since in $V[U]$ there is no subset of
$\omega$, there is no new partition of $\omega$. Let $R$   be a partition
of $\omega$. If $R$ is finite, then $R \cap [\omega]^\omega$ is predense
and so $R \cap U \not = \emptyset$ , therefore $U$ is ultrafilter.

If $R$ is infinite then a set $R \cup \{ X : (\forall r \in R) (|X \cap r| \leq 1) \}$
is predense, so $U$ contains either a member of $R$ or some of
its selectors.
\end{proof}


We show that in $([\omega]^\omega,\subseteq^*)$ there is so called base tree.
It is a particular case of more general forcing notions.

\begin{proposition}{\rm (Base tree.)}\label{basetree}
 Let $(P,\leq)$ be an ordering that is atomless, homogeneous, $\sigma$-closed
 and $|P| \leq 2^\omega$. Then there is a family
 $\{ \mathcal A_\alpha : \alpha < \mathfrak (P) \}$ of maximal antichains
 such that
 \begin{itemize}
  \item[(i)] if $\alpha < \beta$ then $\mathcal A_\alpha$ refines  $\mathcal A_\beta$,
  \item[(ii)] $(\forall x \in P) \ (\exists \alpha < \mathfrak h(P)) \
	(\exists y \in \mathcal A_\alpha) \ y \leq x$,
  \item[(iii)] $\forall  y \in \mathcal A_\alpha \
	|\{z \in \mathcal A_{\alpha + 1} : z \leq y \}| = 2^\omega$.
 \end{itemize}
\end{proposition}

The structure $(\bigcup_{\alpha < \mathfrak h(P)} \mathcal A_\alpha, \leq)$
is a tree by the general definition in chapter \ref{I,kde}, and the reason why
this structure is caled base tree is that the undelrying set
$\bigcup_{\alpha < \mathfrak h(P)}$ is dense in $(P,\leq)$.

\begin{proof}
 Since $(P,\leq)$ is an atomless and $\sigma$-closed ordering, its distributivity
number $\kappa = \mathfrak h(P)$ is well defined and any family of maximal
antichains of size strictly less then $\kappa$ has a common refinement and
moreover $\kappa$ is uncountable regular cardinal, $\kappa \leq 2^\omega$ for
$|P| \leq 2^\omega$.

Take a family $\{Q_\alpha : \alpha < \kappa\}$ of refining maximal
antichains. From homogenity of $P$ we can suppose that every $p \in P$ is
compatible with at least two elements of some $Q_\alpha$.

We claim
$$
(\forall p \in P) \ (\exists \alpha < \kappa) \quad
	|\ \{\ q \in Q_\alpha \ : \ p \parallel q \  \} \ | \ = \ 2^\omega.
$$
Let $p \in P$ be given. By recursion for nodes $\varphi$ of binary tree
${}^\omega \{ 0,1 \}$ we choose labels $\< \alpha_\varphi, p_\varphi \>$,
where $\alpha_\varphi < \kappa$ and $p_\varphi \in P$. Fro $\emptyset$,
put $\alpha_\emptyset = 0$ and $p_\emptyset = p$. If we already know
$\alpha_\varphi, p_\varphi$, then for $\varphi_0 = \varphi^\smallfrown \{0\}$
and $\varphi_1 = \varphi^\smallfrown \{1\}$ set $\alpha_{\varphi_i}$
the least $\alpha < \kappa$  such that there are $q_0, q_1 \in Q_\alpha$,
$q_0 \not = q_1$ such that $p_\varphi \parallel q_0$ and $p_\varphi \parallel q_1$
and choose $p_{\varphi_0}$ as an element which is below $p_\varphi$ and $q_0$
and $p_{\varphi_1}$ below $p_\varphi$ and $q_1$.

For arbitrary branch $f : \omega \to \{0,1\}$ we have descending chain
$\{p_{f \upharpoonright n} : n \in \omega \}$, since $P$ is $\sigma$-closed
there is an element $p_f$ which is below all $p_{f \upharpoonright n}$.

The set $\{ p_f : f \in {}^\omega \{0,1 \} \ \}$  has size $2^\omega$ is
an antichain, all $p_f$'s are below $p$.  Denote
$\beta(p) = \sup \{\alpha_\varphi : \varphi \in {}^{< \omega} \{0,1 \} \}$.
Then $\beta < \kappa$ and $p$ is compatible with $2^\omega$ many elements
of $Q_\beta$, since $p_f, p_g$ are compatible with different elements of $Q_\beta$
provided $f \not = g$.

\smallskip

The following is an easy corollary of the claim:
\begin{itemize}
 \item[(a)] for any $p \in P$ there is a maximal antichain in $(\leftarrow,p]$ of
	size $2^\omega$,
 \item[(b)] there is an antichain $\mathcal X_\beta$ such that
	$(\forall p \in P) \ (\beta_p = \beta) \rightarrow
	(\exists x \in \mathcal X_\beta) \ x \leq p$.
\end{itemize}

For $p \in P$ pick up a maximal antichain $\mathcal Y(p)$ in $(\leftarrow,p]$
of size $2^\omega$. By recursion, define a base tree: Put $\mathcal A_0 = Q_0$,
for limit ordinal $\alpha$  let $\mathcal A_\alpha$ be a common refinement of
$\{ \mathcal A_\beta : \beta < \alpha \}$ and
let $\mathcal A_{\alpha +1}$ be a maximal antichain refining
$\bigcup \{\mathcal Y_p : p \in \mathcal A_\alpha \}$, $Q_\alpha$ and $\mathcal X_\alpha$
\end{proof}

\begin{corollary}
 Suppose that $(P,\leq)$ and $(Q,\leq)$ satisfy conditions of previous proposition
and $\mathfrak h(P) = \mathfrak h(Q) = \omega_1$. Then $P$ and $Q$ have dense
isomorphic subsets.
\end{corollary}

\begin{proof}{\scshape Hint.}
 Let $\{\mathcal A_\alpha : \alpha < \omega_1 \}$ and
$\{\mathcal B_\alpha : \alpha < \omega_1 \}$ be corresponding base trees, we
can assume that $| \mathcal A_0 | = | \mathcal B_0 | = 2^\omega$. Let
$\varphi_0$   be one-to-one mapping of $\mathcal A_0$ onto
 $\mathcal B_0$, $\varphi_0$ can be extended to isomorphism of
$(\bigcup \{\mathcal A_\alpha : \alpha \in \omega_1 \},\leq_1)$ onto
$(\bigcup \{\mathcal B_\alpha : \alpha \in \omega_1 \},\leq_2)$.
\end{proof}

Another corollary of base tree Proposition \ref{basetree} is the
following lemma concerning forcing notion $(P,\leq)$ satisfying
$cc$-properties \eqref{cc-prop}. The particular case of this
lemma proves remaining claims concerning $([\omega]^\omega \subseteq^*)$

\begin{lemma}
 Let $\kappa = \mathfrak h(P)$. Then
\begin{itemize}
 \item[(i)] $P$ does not add new subset of $\lambda$ for $\lambda < \kappa$ and
 \item[(ii)] $P$ adds a  mapping of $\kappa$ onto $(2^\omega)^V$, so if
	$\kappa < (2^\omega)^V$ then $P$ collapses groundmodel continuum onto
	$\kappa$ in any generic extension via $P$.
\end{itemize}
\end{lemma}

\begin{proof}
\noindent {\itshape (i)} for any $\lambda < \kappa$, $(P,\leq)$
	is $\lambda$-distributive, see \ref{}.

\noindent {\itshape (ii)} Let $\{\mathcal A_\alpha : \alpha < \kappa \}$
be a base tree of $P$. For given $\alpha < \kappa$ choose a mapping
$f_\alpha : \\mathcal A_{\alpha +1} \to 2^\omega$ such that for every
$a \in \mathcal A_\alpha$ $f_\alpha \upharpoonright \{b \leq a\}$
is one-to-one mapping of the set $\{b \in \mathcal A_{\alpha +1} : b < a \}$
onto $2^\omega$. if $G$ is a generic filter on $P$ over $V$ then
$\tau = \{ \< \alpha, p \> : \alpha < \kappa, \ p < 2^\omega; \
	(\exists a \in \mathcal A_{\alpha+1} \cap G) \ f_\alpha(a) = p \}$.
is the desired mapping, that belongs to $V[G]$.
\end{proof}


\subsection{${}$ Cardinal numbers $\mathfrak h_n$, for $n \geq 2$.}

Let $N \in \{n \in \omega: n \geq 2\} \cup \{ \omega \}$, then the
product $\prod_{n \in N} ([\omega]^\omega,\subseteq^*)$ also satisfy
$cc$-property. The distributivity number of such product is denoted
$\mathfrak h_N$.

\begin{lemma}
 Let $P,Q$ are orderings satisfying $cc$-properties and there is a regular
embedding of $P$ into $Q$. Then $\mathfrak h(Q) \leq \mathfrak h(P)$.
\end{lemma}

\begin{proof}
 We proceed toward a contradiction. Suppose $\mathfrak h(P) < \mathfrak h(Q)$
and we can assume that $P \subseteq Q$.

Let $\{\mathcal A_\alpha : \alpha < \mathfrak h(P) \}$ be a base matrix for $P$.
Then $\mathcal A_\alpha$ are also maximal antichains in $Q$. Since
$\mathfrak h(P) < \mathfrak h(Q)$, there is maximal antichain $\mathcal A$ in $Q$
refining all $\mathcal A_\alpha$'s. Since the embedding of $P$ into $Q$ is
regular, the complete Bolean algebra $RO(P)$ is a complete subalgebra
of complete Boolean algebra $RO(Q)$. Let $S = \{ upw(a) : a \in \mathcal A \}$,
then $S \subset RO(P)^+$ and no element of $P$
is below any element of $S$ which is a contradiction since $P$ is dense in $RO(P)$.
\end{proof}

The preceding lemma and the fact that the caninical embedding
of $\prod_{i < n} ([\omega]^\omega,\subseteq^*)$ into $\prod_{i < m} ([\omega]^\omega,\subseteq^*)$
for $n \leq m$ is regular give us the basic inequality
$$
\mathfrak h \geq \mathfrak h_2 \geq \mathfrak h_3 \geq \dots \geq \mathfrak h_\omega.
$$
In this generaly infinite sequence there are clearly only finitely many jumps; i.e.
only finitely many inequalities can be strict. It is still an open problem if the
jumps can ocur on arbitrarily prescribed places. It is known (Shelah and Spinas)
that it is consistent that $\omega_2 = \mathfrak h > \mathfrak h_2 = \omega_1$.


\subsection{${}$ Continuum hypothesis.}

For every model of ZFC there is a generic extension that satisfy the continuum hypothesis
and moreover the prediction principle $\Diamond$ holds true.

Standard forcing notions for producing extension that satisfy CH are the following
\begin{eqnarray*}
& (\ Fn(\omega_1, \{0,1\};\ \omega_1), \ \supseteq \ ), \\
& (\ Fn(\omega_1, \mathbb R;\ \omega_1), \ \supseteq \ ), \\
& (\ Fn(2^\omega, \{0,1\};\ \omega_1), \ \supseteq \ ),
\end{eqnarray*}
those forcings are equivalent, separatice partial orderings, satisfying $cc$-condition
with distributivity number $\mathfrak h(P) = \omega_1$.

\smallskip

Let $G$ be a geric filter on such ordering. By the previous it is clear that there are
no new reals in $V[G]$ and that there is is one-to-one mapping $\tau$ of $\omega_1$
on $(2^\omega)^V$. Hence, if $\{r_\alpha : \alpha < (2^\omega)^V \}$ is the numbering
of all subsets of $\omega$ in $V$ then  $\{r_{\tau(\alpha)} : \alpha < (2^\omega)^V \}$
is the ordering of $\mathcal P(\omega)$ in $V[G]$. So $V[G] \vDash CH$.

\begin{definition}
 \emph{$\Diamond$-sequence}. A sequence $\< A_\alpha   : \alpha < \omega_1 \>$ is called
$\Diamond$-sequence if
\begin{itemize}
 \item[(i)] $A_\alpha \subset \alpha$ for every $\alpha \in \omega_1$ and
 \item[(ii)] for any $X \subseteq \omega_1$ the set
$\{ \alpha \in \omega_1 :  X \cap \alpha = A_\alpha \}$ is stationary in $\omega_1$.
\end{itemize}
\end{definition}

Gussing principle $\Diamond$ that postulates the existence of a $\Diamond$-sequence
is due to R.~Jensen. It is easy to see, that $\Diamond$ implies CH.

\smallskip

Let us mention two well known consequences of $\Diamond$ without proofs:
\begin{itemize}
 \item[(a)] there is a Souslin tree, so the negation of Souslin hypothesis holds true.
 \item[(b)] (A.~Ostaszewski) There is an Ostaszewski space; i.e. a topological space
which is completely regular, hereditarily separable, countable compact, perfectly normal
 but non-compact space.
\end{itemize}


%
% \begin{theorem}
% Suppose $B$ is a complete Boolean algebra carrying a $\sigma$-complete
% strictly positive probability measure (or a Maharam submeasure).
%  Then $(B,\leq)$ is weakly distributive.
% \end{theorem}
% %\begin{proof}{}
% %\end{proof}
%
% %%%%%%%%%%%%%%%%%%%%% HECHLER %%%%%%%%%%%%%%%%%%%%%
% \begin{definition}{\bf Hechler.}
% The standard definition of Hechler forcing is the following:
% $$ %\startformula
% H=\{(s,f):s\in Seq,f\in{}^\omega\omega,f\ \hbox{is nondecreasing}\}
% $$ %\stopformula
%  where $(s,f)\leq(t,g)$ iff $s\supseteq t$, $f\geq g$, i.e. $(\forall n<\omega)(f(n)\geq g(n))$, and $(\forall i\in dom(s)\setminus dom(t))(s(i)\geq f(i))$
% \end{definition}
%
% We shall now introduce a variant of this definition in the language of trees.
%
% \begin{definition}
% $H=\{T\subseteq Seq\uparrow:T\ \hbox{is a tree}\ \&\ (\forall s\in T)(\{n:s\conc n\in T)\}\ \hbox{is cofinite})\}$ where $H$ is
% ordered by inclusion.
% \end{definition}
%
% The standard Hechler can be regularly embedded into the 'tree' Hechler and both forcings behave very similarly, although they are probably
% not forcing equivalent.
%
% \begin{proposition}
% Cohen forcing $C$ can be regularly embedded into $H$.
% \end{proposition}
%
% The following forcing notions are not ccc:
%
% \begin{definition}{\bf Sacks.}
% $S=\{T\subseteq Seq_2:(\forall s\in T)(\exists t\in T)(s\subseteq t\ \&\ \{t\conc0,t\conc1\}\subseteq T)\}$% borel/countable
% \end{definition}
% iii
% \begin{fact}
% $S\sim \mbox{Borel}(2^\omega)/[2^\omega]^{\leq\omega}$.
% \end{fact}
%
% %%%%%%%%%%%%%%%%%%%%% MILLER %%%%%%%%%%%%%%%%%%%%%
% \begin{definition}{\bf Miller.}
% $M=\{T\subseteq Seq:(\forall s\in T)(\exists t\in T)(s\subseteq t\ \&\ |\{n:t\conc n\in T\}|=\omega\}$% borel/K_sigma (spocetne sjednoceni kompaktu)
% \end{definition}
%
%
% \begin{fact}
% $M\sim \mbox{Borel}(\N)/K_\sigma$ where $K_\sigma$ is the $\sigma$-ideal generated by compact subsets of $\N$.
% \end{fact}
%
% %%%%%%%%%%%%%%%%%%%%% LAVER %%%%%%%%%%%%%%%%%%%%%
% \begin{definition}{\bf Laver.}
% $L=\{T\subseteq Seq:(\forall s\in T,stem(T)\subseteq s)(|\{n:s\conc n\in T\}|=\omega\}$
% \end{definition}
%
% \begin{definition}
% We define the \emph{Laver ideal} as follows. For any $g\in{}^{Seq}\omega$ let $A_g=\{h\in{}^\omega\omega:(\forall n<\omega)(h(n)\leq g(h\upharpoonright n))\}$. Then the Laver ideal $I_L$ is the $\sigma$-ideal generated by $\{A_g:g\in{}^{Seq}\omega$.
% \end{definition}
%
% \begin{fact}
%  $L\sim \mbox{Borel}({}^\omega\omega)/I_L$.
% \end{fact}
%
%
% %%%%%%%%%%%%%%%%%%%%% MATHIAS %%%%%%%%%%%%%%%%%%%%%
% \begin{definition}{\bf Mathias forcing.}
% The classical definition of Mathias forcing is as follows
% $$ %\startformula
% M=\{(s,A):s\in Seq\uparrow,A\in[\omega]^\omega\ \&\ min\ A> max\ s\}
% $$ %\stopformula
%  where
% $$ %\startformula
% (s,A)\leq(t,B)\equiv s\supseteq t\ \&\ A\subseteq B
% $$ %\stopformula
% $M$ can be represented in tree language as
% $$ %\startformula
% M=(\{T\subseteq Seq\uparrow:T\ \hbox{is a tree}\ \&\ (\exists A\in[\omega]^\omega)(\forall t\in Seq\uparrow, t\supseteq stem(T))(
% rng(t)\subseteq A\rightarrow t\in T)\},\subseteq)
% $$ %\stopformula
% \end{definition}
%
% \begin{definition}
% For $(s,A)\in M$ we let $[s,A]=\{B\in[A]^\omega:s\sqsubseteq B\}$. The ideal $I_M$ of \emph{Ramsey-null} sets is the $\sigma$-ideal generated by
% $\{X\subseteq[\omega]^\omega:(\forall (s,A)\in M)(\exists (t,B)\in M, (t,B)\leq (s,A))([t,B]\cap X=\emptyset)\}$.
% \end{definition}
%
% \begin{fact}
% $M\sim \mbox{Borel}([\omega]^\omega)/I_M$.
% \end{fact}
%
% %%%%%%%%%%%%%%%%%%%%% P(omega)/fin %%%%%%%%%%%%%%%%%%%%%
% \begin{definition}
% ($\pomega/fin$)
% \end{definition}
%
% \begin{definition}{\bf Prikry.} % prostřelování filtru
% \end{definition}

%%%%%%%%%%%%%%%%%%%%%%%%%%%%%%%%%%%%%%%%%%%%%%%%%%%%%%%%%%%%%%%%%%%%%%%
%%%                          END                                    %%%
%%%%%%%%%%%%%%%%%%%%%%%%%%%%%%%%%%%%%%%%%%%%%%%%%%%%%%%%%%%%%%%%%%%%%%%


% \subsection{I DON'T KNOW WITCH}\label{first-translation}
%
% We know that a given ordering $P=(P,\leq)$ determines a complete Boolean algebra $B=RO(P)$, i.e. that there is an order homomorphism
% $h:(P,\leq)\to (B^+,\leq)$ which preserves disjointness and its range is dense in $B$ (it is an embedding iff $P$ is a separative partial order).
% We also know that a generic filter $G$ on $P$ determines a generic filter $\bar{G}=\{b\in B:(\exists p\in P\cap G)(h(p)\leq b)\}$ on $B$
% and vice versa. These two filters are similar (\ref{similarity}).
%
% We shall admit, that groundmodel relations with domains subsets of $P$ are actually names for semisets and the image $r[G]$ of a generic filter via
% such a relation $r$ is in fact a semiset --- the `interpretation' of $r$ `via' the generic filter $G$. It shouldn't be surprising that
% different names can have identical interpretations via a particular generic filter. The question is whether we can tell that two names are potentially
% names for the same semiset based just on the knowledge of the names themselves. The answer is yes. Let us first show that there is a natural
% correspondence between names---relations and names---Boolean functions:
%
% For a relation $\dot{r}$ and $a\in V$ such that $rng(\dot{r})\subseteq a$ we define a function $\dot{f}:a\to B$ as follows:
% $$
% \dot{f}_r(x)=\bigvee\{h(p):p\in P\ \&\ \langle p,x\rangle\in r\}.
% $$
%
% \begin{fact}
% $$
% r[G] = \{x\in a:f(x)\in\bar{G}\}.
% $$
% \end{fact}
% \begin{proof} The inclusion $\subseteq$ is clear. On the other hand if $f(x)\in\bar{G}$ then the set $\{h(p):p\in P\ \&\ \langle p,x\rangle\in r\}$ is
% predense below $f(x)$. Since $f(x)\in\bar{G}$ there must be a $p\in G$ such that $\langle p,x\rangle\in r$
% \end{proof}
%
% On the other hand, given a Boolean function---name $\dot{f}:a\to B$ we define the corresponding relation---name $\dot{r}$:
% $$
% \dot{r}_f=\{\langle p,x\rangle: p\in P\ \&\ h(p)\leq f(x)\}.
% $$
%
% Note that if we start with $\dot{r}$ then transfer to $\dot{f}_r$ and go back to $\bar{r}=\dot{r}_{\dot{f}_r}$, we get $\bar{r}\supseteq \dot{r}$ and
% clearly $\bar{r}[G]=\dot{r}[G]$. The relation $\bar{r}$ can be defined directly from $\dot{r}$:
% $$
% \langle q,x\rangle\in\bar{r}\equiv \{p\in P:\langle p,x\rangle\in r\}\ \mbox{is predense below}\ q.
% $$
% It follows that the fibers of $\bar{r}$ are downward closed, i.e. if $\langle q,x\rangle\in\bar{r}$ and $q^\prime\leq q$ then
% $\langle q^\prime,x\rangle\in\bar{r}$.
%
% We now introduce a new notation in line with our view of the relations as names and elements of $P$ as conditions. Given $p\in P$ a relation $\dot{r}$
% and $x$ we write
% $$
% p\force x\in\dot{r}
% $$
% to mean $\langle p,x\rangle\in\dot{r}$ and we read it as: ``$p$ forces that $x$ is an element of the set named by $\dot{r}$''. From the observation
% that the fibers of $\bar{r}$ are downward closed we immediately get that if a condition $p$ forces $x$ to be an element of $\dot{r}$ then any stronger
% condition $q\leq p$ also forces $x$ to be an element of $\dot{r}$. We also define the meaning of
% $$
% p\force x\not\in\dot{r}
% $$
% as $(\forall q\leq p)(q\not\force x\in\dot{r})$, i.e. $p\force x\not\in\dot{r}$ iff no stronger condition forces $x$ into $\dot{r}$. Note that if
% $\dot{r}=\bar{r}$ then this is the same as saying $p\not\force x\in\dot{r}$ (because $\bar{r}$ is downward closed) but in general it need not be.
% Thus, given a general $\dot{r}$ and $x$, $P$ is split into three parts: conditions which force $x$ into $\dot{r}$, condition which force $x$ out of $\dot{r}$
% and the remaining conditions, which ``are not sure''. Below each of these third conditions there must be two disjoint conditions from the first and second part.



